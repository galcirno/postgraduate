\begin{center}
{\erhao\heiti 平行相移深度神经网络\\\vspace{0.2cm}模型及算法研究}\\
\end{center}
\vspace{0.7cm}
\centerline{\normalsize{\songti\xiaosihao\textbf{\heiti 研究生姓名:}
\hspace{10pt}邢康}\hspace{20pt} \normalsize{ \songti\xiaosihao\textbf{\heiti 导师姓名:}
\hspace{10pt}孙小军~~~教授}}
\centerline{\normalsize{ \songti\textbf{\heiti 学科:}系统科学\quad}\normalsize{ \songti\textbf{\heiti 研究方向: }复杂网络理论与应用\quad}\normalsize{ \songti\textbf{\heiti 年级: } 2020 级} }
\vspace{0.5cm}
\centerline{\heiti\sanhao 中文摘要}
\vspace{0.1cm}
\par
% 随着实验核物理的发展,核数据的类别和数量越来越多,这就使得机器学习方法在研究核物理中具备了可能性,而使用机器学习研究中子共振截面还没有人涉及。当靶核比较重时,中子共振截面具有大量共振峰,研究这些共振峰对基础核物理,核技术有着重大作用。
随着实验核物理的发展,核数据的类别及数量越来越多,机器学习方法在核物理领域中的应用也越来越广,但在中子共振截面领域至今仍无人涉及。中子诱发燃料核$^{235}$U裂变反应的共振截面,存在低频、高频和不可分辨共振区,这些共振裂变截面大于快中子能区裂变截面几个数量级,这在新一代核能技术、国防战略武器、基础核物理研究等领域中有着非常重要的实际应用和学术价值。

本文以$^{235}\text{U}(n,f)$裂变共振截面为研究对象,主要的工作有以下三个方面。首先通过理论分析和算法验证了传统深度神经网络方法(DNN)在低频振荡数据情况下的适用性,但在高频振荡数据情况下具有很大的局限性。其次,借鉴平行相移理论(PPS),在DNN基础上发展能同时适用于低频和高频变化的共振裂变截面的平行相移深度神经网络(PPSDNN)模型。该模型主要包含以下几方面的内容:(1)傅立叶变换和尼奎斯特采样定理,傅立叶变换可以将目标函数从时域空间变换到频域空间,尼奎斯特采样定理揭示了在不失真的情况下需要的最小采样频率;(2)单位分解定理,它可以将目标函数分解成若干部分;(3)B-样条函数,它可以用来构造满足单位分解的函数;(4)平行相移在频率空间中的实现,这是PPSDNN算法核心,可以将目标函数从高频平移到低频。最后,为应用该模型,本文通过构建算法,使用python语言研发出程序代码。算法主要思路:第一步把目标函数通过单位分解分解成若干部分,对低频部分直接使用DNN训练并给出最佳结果;对高频部分通过傅立叶变换将目标函数在频域空间上使用平行相移将高频平移到低频,第二步使用DNN训练这些低频函数并将训练结果再还原到高频振荡数据;第三步将训练得到的若干目标函数叠加得到最终对目标函数的训练结果。通过$^{235}\text{U}(n,f)$裂变共振截面评价库数据的学习,发现PPSDNN模型可以得到较好的效果,为进一步针对实验数据和获得工程所需的中子共振参数奠定了坚实的基础,具有很大的应用前景。
\par
\vspace{0.5cm}
\noindent{\heiti 关键词:}
$^{235}\text{U}(n,f)$;中子共振截面;深度神经网络方法;平行相移深度神经网络模型;算法