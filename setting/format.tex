%%%%%%%%%%%字体大小
\newcommand{\chuhao}{\fontsize{42pt}{\baselineskip}\selectfont}
\newcommand{\xiaochuhao}{\fontsize{36pt}{\baselineskip}\selectfont}
\newcommand{\yihao}{\fontsize{26pt}{\baselineskip}\selectfont}
\newcommand{\xiaoyihao}{\fontsize{24pt}{\baselineskip}\selectfont}
\newcommand{\erhao}{\fontsize{22pt}{\baselineskip}\selectfont}
\newcommand{\xiaoerhao}{\fontsize{18pt}{\baselineskip}\selectfont}
\newcommand{\sanhao}{\fontsize{16pt}{\baselineskip}\selectfont}
\newcommand{\xiaosanhao}{\fontsize{15pt}{\baselineskip}\selectfont}
\newcommand{\sihao}{\fontsize{14pt}{\baselineskip}\selectfont}
\newcommand{\xiaosihao}{\fontsize{12pt}{\baselineskip}\selectfont}
\newcommand{\wuhao}{\fontsize{10.5pt}{\baselineskip}\selectfont}
\newcommand{\xiaowuhao}{\fontsize{9pt}{\baselineskip}\selectfont}
\newcommand{\liuhao}{\fontsize{7.5pt}{\baselineskip}\selectfont}
\newcommand{\xiaoliuhao}{\fontsize{6.5pt}{\baselineskip}\selectfont}
\newcommand{\qihao}{\fontsize{5.5pt}{\baselineskip}\selectfont}
\newcommand{\xiaoqihao}{\fontsize{5pt}{\baselineskip}\selectfont}
%%%%%%%%%%%%目录
% 定义章节标题格式
% 带编号的章节标题
% 带编号的章节标题


% 定义目录内容格式
% \titlecontents{chapter}[0pt]
%   {\vspace{1ex}\bfseries\sffamily\sihao}
%   {\thecontentslabel\quad}
%   {}{\hfill\contentspage}
% \titleformat{\chapter}[display]
%   {\sffamily\xiaosanhao\bfseries}{\chaptertitlename\ \thechapter}{1em}{\vspace{1.5ex}}
% % 不带编号的章节标题
% \titleformat{name=\chapter,numberless}[display]
%   {\sffamily\xiaosanhao\bfseries}{\titlerule[1pt]\vspace{1ex}\filcenter}{}{}


% 定义目录内容格式
% \titlecontents{chapter}[0pt]
%   {\vspace{1ex}\bfseries\sffamily\sihao}
%   {\thecontentslabel\quad}
%   {}{\hfill\contentspage}
% \titlecontents{chapter}[0pt]{\vspace{0.5\baselineskip}\bfseries\sffamily\sihao}
% {\thecontentslabel\quad}{}{\hfill\contentspage}
%\setcounter{tocdepth}{3}
%\setcounter{secnumdepth}{3}
\dottedcontents{chapter}[1.8cm]{\fontsize{14pt}{14pt}\selectfont\heiti\vspace{0.5cm}}{4.0em}{10pt}
%\dottedcontents{section}[0.8cm]{}{1.8em}{5pt}
%\dottedcontents{subsection}[1.9cm]{}{2.7em}{5pt}
%\dottedcontents{subsubsection}[2.86cm]{}{3.4em}{5pt}
%====================== 定制图形和表格标题样式 =====================%
%---------------------- 定制图形和表格标题格式 ---------------------%
\renewcommand{\captionlabeldelim}{} %定义如  "图(表)2: 示例" 中的间隔符号,如 ":" ,这里定义为空
\renewcommand{\captionlabelsep}{\hspace{1em}} %定义图表编号与标题间的间隔距离
\renewcommand{\captionlabelfont}{\small \heiti\bf} %定义图表标签的字体
\renewcommand{\captionfont}{\small \songti\rmfamily} %定义图表标题内容的字体
%% \scriptsize \footnotesize \small \large \Large  %图形标签字体大小
%\renewcommand\arraystretch{1.5}
%--------------------- 定义图、表、公式的编号格式 -------------------%
\renewcommand{\thetable}{\arabic{chapter}-\arabic{table}}
\renewcommand{\theequation}{\arabic{chapter}-\arabic{equation}}
\renewcommand{\thefigure}{\arabic{chapter}-\arabic{figure}}
%\renewcommand{\thetable}{\arabic{chapter}-\arabic{tabular}}

%%%%%%%页眉页脚
% 定义页眉与正文间双隔线
\newcommand{\makeheadrule}{%
    \makebox[0pt][l]{\rule[.7\baselineskip]{\headwidth}{0.3pt}}%0.4
    \rule[0.85\baselineskip]{\headwidth}{1.5pt}\vskip-.8\baselineskip}%1.5 0.4->0.5
\makeatletter
\renewcommand{\headrule}{%
    {\if@fancyplain\let\headrulewidth\plainheadrulewidth\fi
     \makeheadrule}}
\makeatother
%如果需要画单隔线,则需要
%\iffalse%-------------------------------%
%\renewcommand{\headrulewidth}{0.5pt}    %在页眉下画一个0.5pt宽的分隔线
%\renewcommand{\footrulewidth}{0pt}      % 在页脚不画分隔线。
%\fi%-------------------------------------%

\pagestyle{fancyplain}

%\renewcommand{\chaptermark}[1]%
%{\markboth{\CTEXthechapter \ #1}{}}            % \chaptermark 去掉章节标题中的数字
%\renewcommand{\sectionmark}[1]%
%{\markright{\thesection \ #1}{}}            % \sectionmark 去掉章节标题中的数字
\fancyhf{}  %清除以前对页眉页脚的设置

%\fancyhead[LE]{\songti\rightmark}     % 在book文件类别下,
\fancyhead[RO,LE]{\songti\leftmark}% \leftmark自动存录各章之章名,

\fancyfoot[C]%[RE][LO]
%{\CJKfamily{hei} 第\thepage页,共\pageref{LastPage}页}
{-\,\thepage\,-}

%如果要要改变封面和章首页的页眉和页脚,则需要
%\iffalse%----------------------------------------%
%\fancypagestyle{plain}
%{
%\fancyhead{}                                    % clear all header fields
%\fancyhead[LE,RO]{\leftmark}
%\renewcommand{\headrulewidth}{0pt}
%\fancyfoot{}                                    % clear all footer fields
%\fancyfoot[CE,CO]{\thepage}
%}
%\fi%--------------------------------------------%

% 增加 \upcite 命令使显示的引用为上标形式
% \newcommand{\upcite}[1]{$^{\mbox{\scriptsize \cite{#1}}}$}             % 方法1
%\newcommand{\upcite}[1]{\textsuperscript{\textsuperscript{\cite{#1}}}}  % 方法2
%=============================== 脚注 =============================%
%\renewcommand{\thefootnote}{\arabic{footnote}}
%detcounter{footnote}{0}

%==================== 定义题头格言的格式 ==========================%
% 用法 \begin{Aphorism}{author}
%         aphorism
%      \end{Aphorism}
%\newsavebox{\AphorismAuthor}
%\newenvironment{Aphorism}[1]
%{\vspace{0.5cm}\begin{sloppypar} \slshape
%\sbox{\AphorismAuthor}{#1}
%\begin{quote}\small\itshape }
%{\\ \hspace*{\fill}------\hspace{0.2cm} \usebox{\AphorismAuthor}
%\end{quote}
%\end{sloppypar}\vspace{0.5cm}}

%============================== 控制表格线宽 ==========================%
%\makeatletter
%\def\hlinewd#1{%
%  \noalign{\ifnum0=`}\fi\hrule \@height #1 \futurelet
%   \reserved@a\@xhline}
%\makeatother
%\newcommand\vlinewd[1][1pt]{\vrule width #1}

%========================== 其它自定义 ==============================%
%====================================================================%
% 下面定义的命令(\alpheqn \reseteqn)可以使公式编号变为 4-a,4-b
% 使用说明:\alpheqn 为开始产生处,\reseteqn为恢复原来公式编号形式处
% 这两个命令为自定义,使用时应注意:不可放于 数学环境中!!!
% 在公式开始前和结束后使用!!!
%====================================================================%
%\newcounter{saveeqn}%
%
%\newcommand{\alpheqn}{%
%\setcounter{saveeqn}{\value{equation}}%
%\stepcounter{saveeqn}%
%\setcounter{equation}{0}%
%%\renewcommand{\theequation}{\arabic{saveeqn}-\alph{equation}}}%%article 中的定义
%\renewcommand{\theequation}{\arabic{chapter}-\arabic{saveeqn}\alph{equation}}}%book %中的定义
%%{\mbox{\arabic{equation}-\alph{equation}}}}%
%
%\newcommand{\reseteqn}{%
%\setcounter{equation}{\value{saveeqn}}%
%%%\renewcommand{\theequation}{\arabic{equation}}}    %article 中的定义
%\renewcommand{\theequation}{\arabic{chapter}-\arabic{equation}}}  %book 中的定义
%


%定义新的命令\rmnum,\Rmnum,用来显示大小写罗马数字
%用法:\rmnum{数字},\Rmnum{数字}
%\makeatletter
%\newcommand{\rmnum}[1]{\romannumeral #1}
%\newcommand{\Rmnum}[1]{\expandafter\@slowromancap\romannumeral #1@}
%\makeatother
%\CTEXsetup[format={\Large\bfseries}]{section}
%%%%%%
\ctexset{
	  %ontset = founder ,
      chapter={
	    format = {\raggedright\zihao{-3}\songti},
%	    name = {,、},
	    beforeskip ={-20pt},
	    afterskip = {11pt},
	    number=\arabic{chapter},
	    %tocline = {\CTEXnumberline{#1}#2},
	    },
	section = {
	format = {\raggedright\zihao{4}\songti},
%	name ={(,)},
%	number = \chinese{section},
	},
	subsection ={
	format ={\zihao{-4}\songti},
%	name ={,.},
%	number=\arabic{subsection},
	},
  subsubsection ={
  	format ={\zihao{-4}\songti},
  %	name ={,.},
  %	number=\arabic{subsection},
  	},
}
% \titleformat{\subsection}{\left\heiti}{\thesubsection}{1em}
%%%%%%%align公式跨页
\allowdisplaybreaks[4]
\def\urlprefix{url:}

% % 设置页面底部对齐
\raggedbottom
% \flushbottom