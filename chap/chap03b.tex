%----------------------------------------------------------------------------------------------------------------
\chapter[$\Xi^{+}_{c}\to\Xi^{-}\pi^{+}\pi^{+}$衰变及\\$\Xi(1620)$和$\Xi(1690)$的分子态结构特性检验]{$\Xi^{+}_{c}\to\Xi^{-}\pi^{+}\pi^{+}$衰变及$\Xi(1620)$和$\Xi(1690)$的分子态结构特性检验}
\label{chap3}
\section{Belle合作组关于$\Xi_{c}^{+}\to\Xi^{-}\pi^{+}\pi^{+}$衰变$\Xi^{-}\pi^{+}$不变质量谱的实验结果}
2019年Belle合作组发表了关于$\Xi_{c}^{+}\to\pi^{+}\pi^{+}\Xi$的衰变过程的实验结果\cite{PhysRevLett.122.072501}。依据两个$\pi^{+}$介子的动量大小,Belle合作组将末态的两个$\pi^{+}$介子区分为动量较低的$\pi_{L}^{+}$和动量较高的$\pi^{+}_{H}$,测量得到的$\Xi^{-}\pi^{+}_{L}$不变质量谱如图\ref{ex}所示。图中,带误差棒的黑点为实验数据,较平缓的黑色虚线为本底贡献(非共振贡献),$\SI{1.53}{GeV}$附近较高且窄的红色点虚线是$\Xi(1530)(\frac{3}{2}^{+})$共振态的贡献,峰值在$\SI{1.61}{GeV}$附近宽度较大的黑色点虚线是$\Xi(1620)$共振态的贡献,$\SI{1.69}{GeV}$附近较矮的粉色点虚线是$\Xi(1690)$共振态的贡献,蓝色实线是考虑了以上所有贡献后的拟合结果。可以看到,$\Xi^{-}\pi^{+}_{L}$不变质量谱中出现了较强的$\Xi(1530)(\frac{3}{2}^{+})$和$\Xi(1620)$共振峰信号,而$\Xi(1690)$共振峰的信号较弱。\par
\begin{figure}[h]
	\centering
	\includegraphics[width=.496\linewidth]{ex}
	\caption[实验上$\Xi^{-}\pi^{+}_{L}$不变质量谱]{Belle合作组测量得到的$\Xi_{c}^{+}\to\Xi^{-}\pi^{+}\pi^{+}$衰变的$\Xi^{-}\pi^{+}_{L}$不变质量谱。(取自文献\cite{PhysRevLett.122.072501})}
	\label{ex}
\end{figure}\newpage
Belle合作组的拟合得到:$\Xi(1530)(\frac{3}{2}^{+})$的质量和宽度分别为$1533.4\pm\SI{0.4}{MeV}/c^2$和$11.2\pm\SI{1.5}{MeV}$;$\Xi(1620)$的质量和宽度分别为$1610.4\pm\SI{6.0}{MeV}/c^2$和$60.0\pm\SI{4.8}{MeV}$;$\Xi(1690)$的质量和宽度分别为$\SI{1686}{MeV}/c^2$和$\SI{10}{MeV}$。
\par
在本论文的工作中,我们将手动加入$\Xi(1530)(\frac{3}{2}^{+})$共振态的贡献并视之为本底,而$\Xi(1620)$和$\Xi(1690)$将作为$\pi\Xi$(及其耦合道)末态相互作用动力学产生的共振态而自然出现(详细的计算过程和结果见第二章)。
\section{$\Xi_{c}^{+}\to\Xi^{-}\pi^{+}\pi^{+}$衰变过程的反应机制}
弱作用衰变过程$\Xi_{c}^{+}\to\Xi^{-}\pi^{+}\pi^{+}$可通过如下的两步反应机制进行\cite{PhysRevC.95.035212}:第一步,$\Xi_{c}^{+}$中的$c$夸克发生弱衰变,继而终态的所有夸克强子化为$\pi^{+}MB$,其中$M$指赝标介子,$B$指$J^{P}=\frac{1}{2}^{+}$的基态重子;第二步,强子化形成的$MB$通过强相互作用的重散射过程产生$\pi^{+}\Xi^{-}$末态,如图\ref{xi}所示。\par
我们首先考虑$\Xi_{c}^{+}\to\Xi^{-}\pi^{+}\pi^{+}$衰变的第一步反应机制。由于Cabibbo允许的衰变为 $c\to su\bar{d}$和 $cd\to su$,因此存在如图\ref{qqbar}所示几种反应机制。对于图\ref{qqbar}\subref{qq0}而言,这个反应机制与图\ref{qqbar}\subref{qq}相比是受到压低的。首先图\ref{qqbar}\subref{qq}的机制比图\ref{qqbar}\subref{qq0}多了一个颜色增强的因子。这是因为在图\ref{qqbar}\subref{qq0}中要构造色单态 $\pi^{+}$,$W$玻色子 $-u\bar{d}$的顶点中的颜色由$\Xi_{c}^{+}$的 $u$夸克的颜色决定,而对于图\ref{qqbar}\subref{qq}而言, $W$玻色子 $-u\bar{d}$的顶点可以是任意的颜色。另一方面,在 $\Xi_{c}^{+}$中的 $s$夸克和 $u$夸克是一对强耦合反对称的双夸克构型,因此将它们分开是很难的。最后图\ref{qqbar}\subref{qq0}中的 $\Xi_{c}^{+}$中的 $u$夸克和 $W$衰变产生的 $d$夸克难以产生高动量的 $\pi^{+}$。综上,我们将不考虑图\ref{qqbar}\subref{qq0}的机制,图\ref{qqbar}中的\subref{qq3}和\subref{qq4}同样因为颜色重组因子将会受到压低,所以不予考虑。接下来考虑图\ref{qqbar}\subref{qq1}和\subref{qq2},这两种机制是被允许的,但是它们的强相互作用的初态是$\ket{\eta\Xi^{0}}$,而这个道主要的贡献是在$\Xi(1690)$处,再加上要生成高动量的$\pi^{+}$,所以存在一定的运动学压低,它们的贡献不大,稍后再做讨论。\par
接下来我们讨论图\ref{qqbar}\subref{qq}的机制。
由文献\cite{ROBERTS_2008}可知,$\Xi_{c}^{+}$重子的味道波函数为
\begin{equation}
	\ket{\Xi_{c}^{+}}=\frac{1}{\sqrt{2}}\ket{c(su-us)},
\end{equation}
在图\ref{qqbar}\subref{qq}中,$\Xi_{c}^{+}$中的$c$夸克通过$W^{+}$顶点弱衰变为一个$\pi^{+}$和一个$s$夸克,而$\Xi_{c}^{+}$中的$s$和$u$夸克作为旁观者,不参与该弱作用过程,因而该弱作用衰变之后,净余的夸克为图\ref{qqbar}\subref{qq}中右侧所示的$ssu$,相应的味道波函数
\begin{equation}
	\label{eqfw}
	\frac{1}{\sqrt{2}}\ket{s(su-us)}.
\end{equation}
\begin{figure}[t]
	\centering
	\includegraphics[width=.6\linewidth]{xi}
	\caption[$\Xi_{c}^{+}\to\pi^{+}\pi^{+}\Xi^{-}$衰变过程的重散射机制]{$\Xi_{c}^{+}\to\pi^{+}\pi^{+}\Xi^{-}$衰变过程的重散射机制。}
	\label{xi}
\end{figure}
\begin{figure}[h]
	\subfigure[]{
	\begin{minipage}[t]{.5\textwidth}
	\label{qq}
	\centering
	\includegraphics[width=.8\linewidth]{qq}
	\end{minipage}}
	\subfigure[]{
	\begin{minipage}[t]{.5\textwidth}
	\label{qq0}
	\centering
	\vspace{0.7cm}
	\includegraphics[width=.8\linewidth]{qq0}
	\end{minipage}}
	\subfigure[]{
	\label{qq1}
	\begin{minipage}[t]{.5\textwidth}
	\centering
	\includegraphics[width=.8\linewidth]{qq1}
	\end{minipage}}
	\subfigure[]{
	\label{qq2}
	\begin{minipage}[t]{.5\textwidth}
	\centering
	\includegraphics[width=.8\linewidth]{qq2}
	\end{minipage}}
	\subfigure[]{
	\label{qq3}
	\begin{minipage}[t]{.5\textwidth}
	\centering
	\includegraphics[width=.8\linewidth]{qq3}
	\end{minipage}}
	\subfigure[]{
	\label{qq4}
	\begin{minipage}[t]{.5\textwidth}
	\centering
	\includegraphics[width=.8\linewidth]{qq4}
	\end{minipage}}
	\caption[夸克层次Feynman图]{$\Xi_{c}^{+}\to\Xi^{-}\pi^{+}\pi^{+}$衰变第一步过程的夸克层次Feynman图。}
\label{qqbar}
\end{figure}\par
为使式\eqref{eqfw}的夸克成分在强子化之后能够产生$MB$强子对,我们通过强子化的$^3P_{0}$机制引入一对具有真空量子数的正反夸克对$\bar{q}q$。若只考虑轻夸克,则$\bar{q}q$可写为
\begin{equation}
	\bar{u}u+\bar{d}d+\bar{s}s=\sum^{3}_{i=1}\bar{q}_{i}q_{i}.
\end{equation}
于是,经图\ref{qqbar}\subref{qq}所示的强子化过程之后,我们得到的强子态可写为
\vspace{-0.2cm}
\begin{equation}
\label{fds}
	H=\sum_{i=1}^{3}\frac{1}{\sqrt{2}}\ket{s\bar{q}_{i}q_{i}(su-us)}.
\end{equation}
上式所示的$H$为包含一个轻赝标介子和一个$J^{P}=\frac{1}{2}^{+}$的基态重子的强子对。\par
定义由轻夸克构成的$q\bar{q}$介子矩阵$M'$,
\begin{equation}
	M'=
	\begin{pmatrix}
		u\bar{u} & u\bar{d} & u\bar{s} \\
		d\bar{u} & d\bar{d} & d\bar{s} \\
		s\bar{u} & s\bar{d} & s\bar{s} \\
	\end{pmatrix},
\end{equation}
于是式\eqref{fds}所示的强子态$H$可表示为
\begin{equation}
	\label{eq3a1}
	H=\frac{1}{\sqrt{2}}\sum^{3}_{i=1}M'_{3i}q_{i}(su-us),
\end{equation}
\newpage
将$M'$矩阵取为以轻赝标介子九重态表示的SU(3)矩阵,
\begin{equation}
\label{pseudoscalar meson}
	M'\to\phi=
	\begin{pmatrix}
		\frac{\pi^{0}}{\sqrt{2}}+\frac{\eta_{1}}{\sqrt{3}}+\frac{\eta_{8}}{\sqrt{6}} & \pi^{+} & K^{+} \\
		\pi^{-} & -\frac{\pi^{0}}{\sqrt{2}}+\frac{\eta_{1}}{\sqrt{3}}+\frac{\eta_{8}}{\sqrt{6}} & K^{0} \\
		K^{-} & \bar{K}^{0} & \frac{\eta_{1}}{\sqrt{3}}-\sqrt{\frac{2}{3}}\eta_{8} \\
	\end{pmatrix},
\end{equation}
采用$\eta$和$\eta'$的Bramon混合\cite{BRAMON1992416},
\begin{equation}
	\eta=\frac{\eta_{1}}{3}+\frac{2\sqrt{2}}{3}\eta_{8}\qquad\quad\eta'=\frac{2\sqrt{2}}{3}\eta_{1}-\frac{\eta_{8}}{3},
\end{equation}
\newpage
则赝标介子矩阵可以写为
\begin{equation}
\label{pseudoscalar meson2}
	\phi=
	\begin{pmatrix}
		\frac{\pi^{0}}{\sqrt{2}}+\frac{\eta}{\sqrt{3}}+\frac{\eta'}{\sqrt{6}} & \pi^{+} & K^{+} \\
		\pi^{-} & -\frac{\pi^{0}}{\sqrt{2}}+\frac{\eta}{\sqrt{3}}+\frac{\eta'}{\sqrt{6}} & K^{0} \\
		K^{-} & \bar{K}^{0} & -\frac{\eta}{\sqrt{3}}+\sqrt{\frac{2}{3}}\eta' \\
	\end{pmatrix}.
\end{equation}
于是式\eqref{fds}中的强子态$H$可写为
\begin{equation}
	\label{eq37}
	H=K^{-}\left[\frac{1}{\sqrt{2}}u(su-us)\right]+\bar{K}^{0}\left[\frac{1}{\sqrt{2}}d(su-us)\right]+\left(-\frac{\eta}{\sqrt{3}}+\sqrt{\frac{2}{3}}\eta'\right)\left[\frac{1}{\sqrt{2}}s(su-us)\right].
\end{equation}
式\eqref{eq37}中以具体夸克组分表示的是$3q$重子态。下面我们基于基态重子的味道波函数,来看看式\eqref{eq37}中的$3q$态对应于哪些基态重子。由式\eqref{baryon quark}可知如下重子的味道波函数,
\begin{equation}
\begin{split}
	\Sigma^{+}&=\frac{1}{\sqrt{2}}u(su-us),\qquad\quad\Xi^{0}=\frac{1}{\sqrt{2}}s(su-us),\\[1ex]
	\Sigma^{0}&=\frac{1}{2}(dus-dsu+uds-usd),\\[1ex]
	\Lambda^{0}&=\frac{1}{\sqrt{12}}\big(u(ds-sd)+d(su-us)-2s(ud-du)\big).
\end{split}
\end{equation}
可以看出,式\eqref{eq37}中第一项的$3q$态对应于$\Sigma^{+}$,第三项的$3q$态对应于$\Xi^{0}$,第二项的$3q$态对应于$\Lambda$和$\Sigma^{0}$的混合,
\vspace{-0.3cm}
\begin{equation}
	\label{eq39}
	\frac{1}{\sqrt{2}}d(su-us)=a_{1}\Sigma^{0}+a_{2}\Lambda^{0},
\end{equation}
\vspace{-0.2cm}
其中$a_{1}$和$a_{2}$为组合系数。现在通过波函数的内积来确定组合系数$a_{1}$和$a_{2}$,
\begin{equation}
\begin{aligned}[b]
	a_{1}=&\braket{\frac{1}{2}(dus-dsu+uds-usd)}{\frac{1}{\sqrt{2}}d(su-us)}\\[1ex]
	=&\frac{1}{2}\frac{1}{\sqrt{2}}(-1-1)\\[1ex]
	=&-\frac{1}{\sqrt{2}},
\end{aligned}
\end{equation}
\begin{equation}
\begin{aligned}[b]
	a_{2}=&\braket{\frac{1}{\sqrt{12}}\big(u(ds-sd)+d(su-us)-2s(ud-du)\big)}{\frac{1}{\sqrt{2}}d(su-us)}\\[1ex]
	=&\frac{1}{\sqrt{12}}\frac{1}{\sqrt{2}}(1+1)\\[1ex]
	=&\frac{1}{\sqrt{6}}.
\end{aligned}
\end{equation}
于是式\eqref{eq39}可写为
\begin{equation}
	\frac{1}{\sqrt{2}}d(su-us)=-\frac{1}{\sqrt{2}}\Sigma^{0}+\frac{1}{\sqrt{6}}\Lambda.
\end{equation}
因此式\eqref{eq37}所示的强子态$H$可以用赝标介子-重子对表示为
\begin{equation}
	\label{eq3aa4}
	H=K^{-}\Sigma^{+}+\bar{K}^{0}\big(-\frac{1}{\sqrt{2}}\Sigma^{0}+\frac{1}{\sqrt{6}}\Lambda\big)+\big(-\frac{\eta}{\sqrt{3}}+\sqrt{\frac{2}{3}}\eta'\big)\Xi^{0}.
\end{equation}
由于$\eta'$的质量较大,$m_{\eta'}=\SI{958}{MeV}$,上式中出现的$\eta'\Xi^0$的阈值大于$\SI{2270}{MeV}$,比本文工作中我们感兴趣的$\Xi(1620)$、$\Xi(1690)$的能量要高得多,因此我们略去式\eqref{eq3aa4}中的$\eta'\Xi^0$态,取
\begin{equation}
\label{first state}
	H=K^{-}\Sigma^{+}-\frac{1}{\sqrt{2}}\bar{k}^{0}\Sigma^{0}+\frac{1}{\sqrt{6}}\bar{K}^{0}\Lambda-\frac{1}{\sqrt{3}}\eta\Xi^{0}.
\end{equation}
由上式可见,经历图\ref{qqbar}\subref{qq}所示的强子化过程之后,并没有产生$\pi\Xi$成分。这表明,该成分来自于图\ref{xi}所示的重散射机制。\par
将式\eqref{first state}中各组分的权重因子记为
\begin{equation}
	\label{eqehi}
	h_{K^{-}\Sigma^{+}}=1,\quad h_{\bar{K}^{0}\Sigma^{0}}=-\frac{1}{\sqrt{2}},\quad h_{\bar{K}^{0}\Lambda}=\frac{1}{\sqrt{6}},\quad h_{\eta\Xi^{0}}=-\frac{1}{\sqrt{3}}.
\end{equation}
%\section{同位旋基}
%现在用同位旋基来表示,首先先要进行同位旋约定。采用文献\cite{OSET199899}的同位旋约定即
%\begin{equation}
%\begin{split}
%	\ket{\pi^{+}}=&-\ket{I=1,I_{z}=1}\quad
%	\ket{K^{-}}=-\ket{I=\frac{1}{2},I_{z}=-\frac{1}{2}}\\
%	\ket{\Sigma^{+}}=&-\ket{I=1,I_{z}=1}\quad
%	\ket{\Xi^{-}}=-\ket{I=\frac{1}{2},I_{z}=-\frac{1}{2}}.
%\end{split}
%\end{equation}
%通过CG系数可以得到
%\begin{equation}
%\label{ib1}
%\begin{split}
%	\ket{\pi^{+}\Xi^{-}}=&\ket{-(I_{1}=1,I_{1z}=1);-(I_{2}=\frac{1}{2},I_{2z}=-\frac{1}{2})}\\
%	=&\sqrt{\frac{1}{3}}\ket{I=\frac{3}{2},I_{z}=\frac{1}{2}}+\sqrt{\frac{2}{3}}\ket{I=\frac{1}{2},I_{z}=\frac{1}{2}}\\
%	\ket{K^{-}\Sigma^{+}}=&\ket{-(I_{1}=\frac{1}{2},I_{1z}=-\frac{1}{2});-(I_{2}=1,I_{2z}=1)}\\
%	=&(-1)^{\frac{3}{2}-\frac{1}{2}-1}\sqrt{\frac{1}{3}}\ket{I=\frac{3}{2},I_{z}=\frac{1}{2}}\\
%	 &+(-1)^{\frac{3}{2}-\frac{1}{2}-1}\sqrt{\frac{2}{3}}\ket{I=\frac{1}{2},I_{z}=\frac{1}{2}}\\
%	=&\sqrt{\frac{1}{3}}\ket{I=\frac{3}{2},I_{z}=\frac{1}{2}}-\sqrt{\frac{2}{3}}\ket{I=\frac{1}{2},I_{z}=\frac{1}{2}}\\
%	\ket{\bar{K}^{0},\Sigma^{0}}=&\ket{I_{1}=\frac{1}{2},I_{1z}=\frac{1}{2};I_{2}=1,I_{2z}=0}\\
%	=&(-1)^{\frac{3}{2}-\frac{1}{2}-1}\sqrt{\frac{2}{3}}\ket{I=\frac{3}{2},I_{z}=\frac{1}{2}}\\
%	 &+(-1)^{\frac{1}{2}-\frac{1}{2}-1}(-\sqrt{\frac{1}{3}})\ket{I=\frac{1}{2},I_{z}=\frac{1}{2}}\\
%	=&\sqrt{\frac{2}{3}}\ket{I=\frac{3}{2},I_{z}=\frac{1}{2}}+\sqrt{\frac{1}{3}}\ket{I=\frac{1}{2},I_{z}=\frac{1}{2}},
%\end{split}
%\end{equation}
%其中$\ket{K^{-}\Sigma^{+}}$和$\ket{\bar{K}^{0}\Sigma^{0}}$用到了关系式
%\begin{equation}
%\braket{j_{1}j_{2}m_{1}m_{2}}{j_{1}j_{2}JM}=(-1)^{J-j_{1}-_{2}}\braket{j_{2}j_{1}m_{2}m_{1}}{j_{2}j_{1}JM}.
%\end{equation}
%在$\Xi(1620)$和$\Xi(1690)$的过程只考虑$I=\frac{1}{2}$。同理可以从CG系数得到
%\begin{equation}
%\label{ib2}
%\begin{split}
%	\ket{\pi\Xi,I=\frac{1}{2},I_{3}=\frac{1}{2}}=&\sqrt{\frac{2}{3}}\ket{1,1;\frac{1}{2},-\frac{1}{2}}-\sqrt{\frac{1}{3}}\ket{1,0;\frac{1}{2},\frac{1}{2}}\\
%	=&\sqrt{\frac{2}{3}}\ket{-\pi^{+},-\Xi^{-}}-\sqrt{\frac{1}{3}}\ket{\pi^{0},\Xi^{0}}\\
%	\ket{\bar{K}\Sigma,I=\frac{1}{2},I_{3}=\frac{1}{2}}=&(-1)^{\frac{1}{2}-\frac{1}{2}-1}\sqrt{\frac{2}{3}}\ket{\frac{1}{2},-\frac{1}{2};1,1}\\
%							    &+(-1)^{\frac{1}{2}-\frac{1}{2}-1}\left(-\sqrt{\frac{1}{3}}\right)\ket{\frac{1}{2},\frac{1}{2}1,0}\\
%	=&\sqrt{\frac{1}{3}}\ket{\bar{K}^{0},\Sigma^{0}}-\sqrt{\frac{2}{3}}\ket{-K^{-},-\Sigma^{+}}\\
%	=&\sqrt{\frac{1}{3}}\ket{\bar{K}^{0},\Sigma^{0}}-\sqrt{\frac{2}{3}}\ket{K^{-},\Sigma^{+}}.
%\end{split}
%\end{equation}
%\section{BS方程的核$V_{ij}$}
%$V_{ij}$的定义为
%\begin{equation}
%\label{def v}
%	V_{ij}=\mel{i}{-\mathcal{L}_{2}}{j}.
%\end{equation}
%所以首先要讨论的拉氏量,在文献\cite{OSET199899,Pich_1995,ECKER19951,BERNARD_1995}中,可以知道赝标介子八重态和$\frac{1}{2}^{+}$八重态重子的最低阶手征拉氏量为
%\begin{equation}
%	\mathcal{L}=\langle\bar{B}i\gamma^{\mu}\grad_{\mu}B\rangle-M_{B}\langle\bar{B}B\rangle+\frac{1}{2}D\langle\bar{B}\gamma^{\mu}\gamma_{5}\{u_{\mu},B\}\rangle+\frac{1}{2}F\langle\bar{B}\gamma^{\mu}\gamma_{5}[u_{\mu},B]\rangle.
%\end{equation}
%其中
%\begin{equation}
%\begin{split}
%	&\grad_{\mu}B=\partial_{\mu}B+[\Gamma_{\mu},B]\\
%	&\Gamma_{\mu}=\frac{1}{2}(u^{\dagger}\partial_{\mu}u+u\partial_{\mu}u^{\dagger})\\
%	&U=u^2=\exp(i\sqrt{2}\phi/f)\\
%	&u_{\mu}=iu^{\dagger}\partial_{\mu}Uu^{\dagger},
%\end{split}
%\end{equation}
%$\phi$是赝标介子八重态,具体形式为式\eqref{meson8},$B$为$\frac{1}{2}^{+}$的重子八重态,表达式为式\eqref{baryon8}。相互作用拉氏量来自$\Gamma_{\mu}$的项,所以拉氏量为
%\begin{equation}
%	\mathcal{L}=\langle\bar{B}i\gamma^{\mu}[\Gamma_{\mu},B]\rangle.
%\end{equation}
%把拉氏量处理一下,其中$u$和$u^{\dagger}$为
%\begin{equation}
%\begin{aligned}
%	u=&\exp(\frac{i\sqrt{2}}{2f}\phi)&u^{\dagger}=&\exp(\frac{-i\sqrt{2}}{2f}\phi)\\
%	=&1+\frac{i\sqrt{2}}{2f}\phi-\frac{1}{4f^2}\phi\phi+\cdots&=&1-\frac{i\sqrt{2}}{2f}\phi-\frac{1}{4f^2}\phi\phi+\cdots,
%\end{aligned}
%\end{equation}
%其中$\phi^{\dagger}=\phi$,只保留到$\phi$的平方项,则$\partial_{\mu}u$和$\partial_{\mu}u^{\dagger}$为
%\begin{equation}
%\begin{split}
%	\partial_{\mu}u=&\frac{i\sqrt{2}}{2f}\partial_{\mu}\phi-\frac{1}{4f^2}\partial_{\mu}\phi\phi-\frac{1}{4f^2}\phi\partial_{\mu}\phi\\
%	\partial_{\mu}u^{\dagger}=&\frac{-i\sqrt{2}}{2f}\partial_{\mu}\phi-\frac{1}{4f^2}\partial_{\mu}\phi\phi-\frac{1}{4f^2}\phi\partial_{\mu}\phi.
%\end{split}
%\end{equation}
%现在可以把$\Gamma_{\mu}$表示出来,同样保留到$\phi$的平方项
%\begin{equation}
%\begin{split}
%	\Gamma_{\mu}=&\frac{1}{2}(u^{\dagger}\partial_{\mu}u+u\partial_{\mu}u^{\dagger})\\
%	=&\frac{1}{2}[(1-\frac{i\sqrt{2}}{2f}\phi)(\frac{i\sqrt{2}}{2f}\partial_{\mu}\phi-\frac{1}{4f^2}\partial_{\mu}\phi\phi-\frac{1}{4f^2}\phi\partial_{\mu}\phi)+(1+\frac{i\sqrt{2}}{2f}\phi)(\frac{-i\sqrt{2}}{2f}\partial_{\mu}\phi-\frac{1}{4f^2}\partial_{\mu}\phi\phi-\frac{1}{4f^2}\phi\partial_{\mu}\phi)]\\
%	=&\frac{1}{2}(-\frac{1}{4f^2}\partial_{\mu}\phi\phi-\frac{1}{4f^2}\phi\partial_{\mu}\phi+\frac{1}{2f^2}\phi\partial_{\mu}\phi-\frac{1}{4f^2}\partial_{\mu}\phi\phi-\frac{1}{4f^2}\phi\partial_{\mu}\phi+\frac{1}{2f^2}\phi\partial_{\mu}\phi)\\
%	=&\frac{1}{4f^2}(\phi\partial_{\mu}\phi-\partial_{\mu}\phi\phi)\\
%	=&\frac{1}{4f^2}[\phi,\partial_{\mu}\phi].
%\end{split}
%\end{equation}
%最终的拉氏量可以简化为
%\begin{equation}
%	\mathcal{L}=\frac{1}{4f^2}\langle\bar{B}i\gamma^{\mu}[[\phi,\partial_{\mu}\phi],B]\rangle.
%\end{equation}
%从拉氏量中可以看出$V_{ij}$具有类似的结构
%\begin{equation}
%\begin{split}
%	V_{ij}=&\mel{j}{\mathcal{-L}}{i}\\
%	=&-C_{ij}\frac{1}{4f^2}\bar{u}(p')\gamma^{\mu}u(p)(k_{\mu}+k'_{\mu}),
%\end{split}
%\end{equation}
%$V_{ij}$的具体计算过程详见附录\ref{appendices a},因此$BS$方程的核为
%\begin{equation}
%	V_{ij}=-C_{ij}\frac{1}{4f^2}\sqrt{\frac{E_{i}+M_{i}}{2M_{i}}}\sqrt{\frac{E_{j}+M_{j}}{2M_{j}}}(2\sqrt{s}-M_{i}-M_{j}).
%\end{equation}
%\begin{table}[h]
%\centering
%%\caption{Coefficients $C_{ij}$ of the meson baryon amplitudes for isospin $I=\frac{1}{2}(C_{ji}=C_{ij})$}
%\caption{同位旋$I=\frac{1}{2}$的介子-重子振幅系数$C_{ij}$($C_{ji}=C_{ij}$)}
%\label{Cij}
%\begin{spacing}{1.5}
%%\scalebox{1.5}{
%\setlength{\tabcolsep}{10mm}
%\begin{tabular}{c|rrrr}
%	\toprule
%	\hline
%	~ & $\pi\Xi$ & $\bar{K}\Lambda$ & $\bar{K}\Sigma$ & $\eta\Xi$\\\hline
%	$\pi\Xi$ & 2 & $-\frac{3}{2}$ & $-\frac{1}{2}$ &0 \\
%	$\bar{K}\Lambda$ & ~ & 0 & 0 &$-\frac{3}{2}$ \\
%	$\bar{K}\Sigma$ & ~ & ~ & 2 &$\frac{3}{2}$ \\
%	$\eta\Xi$ & ~ & ~ & ~ &0 \\
%	\Xhline{1pt}
%\end{tabular}
%%}
%\end{spacing}
%\end{table}
%式中$E$和$M$分别是重子的能量和质量,$C_{ij}$见表\ref{Cij}。
\section{重散射机制对衰变振幅的贡献}
\label{sec33}
衰变过程$\Xi_{c}^{+}\to\Xi^{-}\pi^{+}\pi^{+}$的重散射机制如图\ref{xi}所示,相应的衰变振幅可写为
\begin{equation}
	\label{eq3aa5}
	\mathcal{T}=V_{P}\sum_{i}h_{i}G_{i}(M_{\text{inv}})T_{i,\pi^{+}\Xi^{-}}(M_{\text{inv}})
\end{equation}
其中$M_{\text{inv}}=\sqrt{s}$为$\pi^+\Xi^-$的不变质量,$i$指式\eqref{first state}中的介子重子反应道,
\begin{equation}
	i=K^{-}\Sigma^{+},\quad \bar{K}^{0}\Sigma^{0},\quad \bar{K}^{0}\Lambda,\quad \eta\Xi^{0},
\end{equation}
$h_i$为式\eqref{eqehi}中的各权重因子;\;$G_i(M_{\text{inv}})$是第$i$道的介子-重子传播子圈函数,可取为式\eqref{eqcG}所示的三动量截断的$G$函数;\;$T_{i,\pi^+\Xi^-}$是第$i$道至$\pi^+\Xi^-$末态的跃迁振幅;$V_p$是包含了图\ref{xi}中第一个弱相互作用顶角贡献的整体因子,在本论文工作中作为一个可调参数。\par
由式\eqref{eqib1}-\eqref{eqib6}可得到,式\eqref{eq3aa5}中的介子-重子二体散射振幅$T_{i,\pi^{+}\Xi^{-}}$可用同位旋耦合道的散射振幅来表示出来,
\begin{equation}
\begin{split}
	T_{K^{-}\Sigma^{+},\pi^{+}\Xi^{-}}=&\sqrt{\frac{2}{3}}\left(-\sqrt{\frac{2}{3}}\right)T_{\bar{K}\Sigma,\pi\Xi}^{I={1}/{2}}\\[1ex]
	=&-\frac{2}{3}T_{\bar{K}\Sigma,\pi\Xi}^{I=1/2},\\[1ex]
	T_{\bar{K}^{0}\Sigma^{0},\pi^{+}\Xi^{-}}=&\sqrt{\frac{2}{3}}\sqrt{\frac{1}{3}}T_{\bar{K}\Sigma,\pi\Xi}^{I=1/2}\\[1ex]
	=&\frac{\sqrt{2}}{3}T_{\bar{K}\Sigma,\pi\Xi}^{I=1/2},\\[1ex]
	T_{\bar{K}^{0}\Lambda,\pi^{+}\Xi^{-}}=&\sqrt{\frac{2}{3}}T_{\bar{K}\Lambda,\pi\Xi}^{I=1/2},\\[1ex]
	T_{\eta\Xi^{0},\pi^{+}\Xi^{-}}=&\sqrt{\frac{2}{3}}T_{\eta\Xi,\pi\Xi}^{I=1/2}\ .
\end{split}
\end{equation}
在同位旋基下,式\eqref{eqehi}所示的各权重因子变为
\begin{equation}
	h_{\pi\Xi}=0,\quad h_{\bar{K}\Lambda}=\frac{1}{\sqrt{6}},\quad h_{\bar{K}\Sigma}=-\sqrt{\frac{3}{2}},\quad h_{\eta\Xi}=-\frac{1}{\sqrt{3}}.
\end{equation}
于是在同位旋基下,式\eqref{eq3aa5}所示的衰变振幅可改写为
\begin{equation}
	\label{eqiT}
\begin{split}
	\mathcal{T}=V_{P}\Bigg[&h_{\bar{K}\Lambda}G_{\bar{K}\Lambda}(M_{\text{inv}})\cdot\sqrt{\frac{2}{3}}T_{\bar{K}\Lambda,\pi\Xi}^{I=1/2}(M_{\text{inv}})\\[1ex]
			       &+h_{\bar{K}\Sigma}G_{\bar{K}\Sigma}(M_{\text{inv}})\cdot\Big(-\frac{2}{3}+\frac{\sqrt{2}}{3}\Big)T_{\bar{K}\Sigma,\pi\Xi}^{I=1/2}(M_{\text{inv}})\\[1ex]
			       &+h_{\eta\Xi}G_{\eta\Xi}(M_{\text{inv}})\cdot T_{\eta\Xi,\pi\Xi}^{I=1/2}(M_{\text{inv}})\Bigg].
\end{split}
\end{equation}\par
现在考虑图\ref{qqbar}中的\subref{qq1}和\subref{qq2},其中介子夸克成分为$u\bar{u}+d\bar{d}$,根据式\eqref{pseudoscalar meson2}可知$u\bar{u}+d\bar{d}=\frac{2}{\sqrt{3}}\eta$,所以在该机制下初态为$\frac{2}{\sqrt{3}}\ket{\eta\Xi^{0}}$。因为无法判断该机制与\ref{qqbar}\subref{qq}的相位关系,所以我们用一个未知数$x$来表示,因此只需要令$h_{\eta\Xi}=\frac{-1+2x}{\sqrt{3}}$。此时考虑了\ref{qqbar}\subref{qq1}和\subref{qq2}贡献的衰变振幅仍可取为式\eqref{eqiT}的形式,但在其中将$h_{\eta\Xi}$代换为$h_{\eta\Xi}=\frac{-1+2x}{\sqrt{3}}$。\par
\newpage
\section{背景道的贡献}
\label{sec34}
\subsection{$\Xi(1530)$($\frac{3}{2}^{+}$)共振态的贡献}
\label{subsec1530}
Belle合作组在$\Xi_{c}^{+}\to\Xi^{-}\pi^{+}\pi^{+}$衰变的$\Xi^{-}\pi^{+}$不变质量谱中观测到了一个很明显的$\Xi(1530)(\frac{3}{2}^{+})$窄共振峰信号,如图\ref{ex}所示。本论文工作的主要目的在于研究$\Xi(1620)$和$\Xi(1690)$共振态在$\Xi_{c}^{+}\to\Xi^{-}\pi^{+}\pi^{+}$衰变过程中的动力学产生机制及其结构特性,因而我们将$\Xi(1530)$共振态的贡献看作本底贡献,现在我们来计算这一本底贡献。\par
\begin{figure}[htbp]
	\centering
	\includegraphics[width=.6\linewidth]{1530}
	\caption[$\Xi(1530)(\frac{3}{2}^{+})$极点图]{$\Xi_{c}^{+}\to\Xi^{-}\pi^{+}\pi^{+}$衰变过程的$\Xi(1530)(\frac{3}{2}^{+})$极点图。}
	\label{1530}
\end{figure}
$\Xi(1530)$的$J^{P}$量子数为$\frac{3}{2}^{+}$,$\Xi_{c}^{+}$的$J^{P}$量子数为$\frac{1}{2}^{+}$。$\Xi_{c}^{+}\to\pi^{+}\pi^{+}\Xi^{-}$衰变过程的$\Xi(1530)(\frac{3}{2}^{+})$极点图如图\ref{1530}所示。我们先分析该图的第一个顶角,即$\Xi_{c}^{+}(\frac{1}{2}^{+})\to\Xi(1530)(\frac{3}{2}^{+})$ $\pi^{+}_{H}(0^{-})$弱衰变顶角。弱作用过程中宇称不是一个守恒的量子数,但角动量守恒。因而,角动量守恒要求$\Xi_{c}^{+}\to\Xi(1530)(\frac{3}{2}^{+})\pi^{+}_{H}(0^{-})$为$p$-波衰变,可将相关的衰变顶角写为如下形式,
\begin{equation}
	\label{eq3a1}
	V_{1}\propto \vec{S'}^{\dagger}\cdot \vec{p}_{\pi_H},
\end{equation}
其中$\vec{p}_{\pi_H}$是$\pi^{+}_{H}$介子在$\Xi_{c}^{+}$静止系中的三动量,$\vec{S'}$为自旋跃迁算符,其作用使自旋$\frac{1}{2}$的态跃迁至自旋$\frac{3}{2}$的态。\par
我们再来分析图\ref{1530}的第二个顶角,即$\Xi(1530)(\frac{3}{2}^{+})\to\Xi^{-}(\frac{1}{2}^{+})\pi_{L}^{+}(0^{-})$衰变顶角。此时,角动量和宇称均为守恒量子数,因此$\Xi(1530)(\frac{3}{2}^{+})\to\Xi^{-}(\frac{1}{2}^{+})\pi^{+}_{L}(0^{-})$是$p$-波衰变过程,衰变顶角也可写为
\begin{equation}
	\label{eq3a2}
	V_{2}\propto \vec{S'}\cdot \vec{\tilde{p}}_{\pi_L},
\end{equation}
其中$\vec{\tilde{p}}_{\pi_L}$为$\pi_{L}^{+}$介子在$\Xi(1530)$静止系中的三动量。\par
由式\eqref{eq3a1}和\eqref{eq3a2}所示的顶角形式,按照Feynman规则可写出图\ref{1530}所示衰变过程的振幅,
\begin{equation}
	\label{eq3a3}
	\mathcal{T}'=A\sum_{M}\mel{m}{\vec{S'}\cdot \vec{\tilde{p}}_{\pi_L}}{M}\frac{1}{M_{\text{inv}}-M_{\Xi(1530)}+i\frac{\Gamma_{\Xi(1530)}}{2}}\mel{M}{\vec{S'}^{\dagger}\cdot \vec{p}_{\pi_H}}{m'},
\end{equation}
上式中,$A$为常数因子,$m$、$M$和$m'$分别是$\Xi^{-}$、$\Xi(1530)$和$\Xi_{c}^{+}$的角动量第三分量(即磁量子数),也称为极化自由度;\;$\ket{m}$、$\ket{M}$和$\ket{m'}$表示相应的极化状态。由于$\Xi(1530)$是中间共振态,需要对其所有的极化自由度求和,即上式中的$\sum_{M}$。上式中的因子
\begin{equation*}
\dfrac{1}{M_{\text{inv}}-M_{\Xi(1530)}+i\Gamma_{\Xi(1530)}/2}
\end{equation*}
表示中间态$\Xi(1530)$传播子,其中$M_{\Xi(1530)}$和$\Gamma_{\Xi(1530)}$是$\Xi(1530)$的物理质量和宽度,$M_{\text{inv}}$是$\Xi^{-}\pi_{L}^{+}$体系的不变质量。\par
式\eqref{eq3a3}可进一步写为
\begin{equation}
	\label{eqn3-27}
\begin{aligned}[b]
	\mathcal{T}'=&A\sum_{M}\mel{m}{S'_{i}\tilde{p}_{\pi_L,i}}{M}\mel{M}{S'^{\dagger}_{j}p_{\pi_H,j}}{m'}\frac{1}{M_{\text{inv}}-M_{\Xi(1530)}+i\frac{\Gamma_{\Xi(1530)}}{2}}\\[1ex]
	=&A\mel{m}{S'_{i}\tilde{p}_{\pi_L,i}S'^{\dagger}_{j}p_{\pi_H,j}}{m'}\frac{1}{M_{\text{inv}}-M_{\Xi(1530)}+i\frac{\Gamma_{\Xi(1530)}}{2}},
\end{aligned}
\end{equation}
上式中利用了$\sum_{M}\dyad{M}{M}=1$。由于动量$p_{\pi_{H},j}$和$\tilde{p}_{\pi_{H},i}$与自旋无关,矩阵元
\begin{equation*}
\mel{m}{S'_{i}\tilde{p}_{\pi_{L},i}S'^{\dagger}_{j}p_{\pi_{H},j}}{m'}=\mel{m}{S'_{i}S'^{\dagger}_{j}}{m'}\tilde{p}_{\pi_{L},i}p_{\pi_{H},j}.
\end{equation*}
于是\eqref{eqn3-27}式可写为
\begin{equation}
\label{eq3a4}
\mathcal{T}=A\mel{m}{S'_{i}S'^{\dagger}_{j}}{m'}\tilde{p}_{\pi_{L},i}p_{\pi_{H},j}\frac{1}{M_{\text{inv}}-M_{\Xi(1530)}+i\frac{\Gamma_{\Xi(1530)}}{2}}.
\end{equation}\par
下面计算衰变振幅\eqref{eq3a4}式中自旋跃迁算符相关的矩阵元。为此,我们考虑$k$阶球面张量算符$T^{(k)}$,$T^{(k)}_{q}$为该算符的$q$分量。体系的状态由基矢$\ket{\tau jm}$描写,其中$jm$是系统的角动量量子数和磁量子数,$\tau$是其他量子数。根据Wigner-Eckart定理\cite{2001高等量子力学},$T^{(k)}_{q}$在角动量算符的本征态(角动量基)$\ket{jm}$和$\ket{j'm'}$之间的矩阵元可写为
\begin{equation}
\label{eq3a5}
\mel{\tau jm}{T^{(k)}_{q}}{\tau'j'm'}=\braket{j'm';kq}{jm}\mel**{\tau j}{\abs{T^{(k)}}}{\tau'j'},
\end{equation}
其中$j$为由$j'$和$k$合成的总角动量,$q$为$k$的第三分量(即磁量子数),$q=-k,-k+1,\cdots,k$;$\braket{j'm';kq}{jm}$是角动量耦合的C.G.系数;$\mel**{\tau j}{\abs{T^{(k)}}}{\tau'j'}$是一个与量子数$m$、$m'$和$q$无关的约化矩阵元。由\eqref{eq3a5}式可以看到,Wigner-Eckart定理将一个物理过程中只与对称性(在这里是角动量对称性)有关的部分分离出来,体现在C.G.系数中;与物理的相互作用相关的部分体现在约化矩阵元中。\par
现在,考虑自旋跃迁算符$S'^{\dagger}_{\mu}$在角动量基$\ket{\frac{3}{2}M}$和$\ket{\frac{1}{2}m}$之间的矩阵元。由式\eqref{eq3a5},有
\begin{equation}
	\label{eqwe}
\mel{\frac{3}{2}M}{S'^{\dagger}_{\mu}}{\frac{1}{2}m}=\braket{\frac{1}{2}m;1\mu}{\frac{3}{2}M}\mel{\frac{3}{2}}{\abs\bigg{S'^{\dagger}}}{\frac{1}{2}},
\end{equation}
其中约化矩阵元$\mel{\frac{3}{2}}{\abs\big{S'^{\dagger}}}{\frac{1}{2}m}=1$,$k=1$表示$\Xi_{c}^{+}(\frac{1}{2}^{+})\to\Xi(1530)(\frac{3}{2}^{+})\pi^{+}_{H}(0^{-})$是$p$-波衰变,相应的磁量子数$\mu=+1,0,-1$。$S'^{\dagger}_{\mu}$是$\vec{S'}^{\dagger}$算符在球坐标系中的三个分量。在笛卡尔坐标系(直角坐标系)中,$\vec{S'}^{\dagger}$的三个分量为$S'^{\dagger}_{x},S'^{\dagger}_{y}$和$S'^{\dagger}_{z}$,记为$S'^{\dagger}_{i}(i=1,2,3)$。$S'^{\dagger}_{\mu}$和$S'^{\dagger}_{i}$之间的关系为\cite{thompson2008angular}
%%%%%%%%%%%%%%%%
\begin{equation}
	S'^{\dagger}_{+1}=-\frac{1}{\sqrt{2}}(S'^{\dagger}_{1}+iS'^{\dagger}_{2}),\quad S'^{\dagger}_{-1}=\frac{1}{\sqrt{2}}(S'^{\dagger}_{1}-iS'^{\dagger}_{2}),\quad S'^{\dagger}_{0}=S'^{\dagger}_{3}.
\end{equation}
由附录\ref{wigner-eckart}可知,
\begin{equation}
	\label{eq323}
	S'_{i}S'^{\dagger}_{j}=\frac{2}{3}\delta_{ij}-\frac{i}{3}\epsilon_{ijk}\sigma_{k}\,,
\end{equation}
其中$\epsilon_{ijk}$为三阶全反对称张量,$\sigma_{k}$为Pauli矩阵。\par
将式\eqref{eq323}带入式\eqref{eq3a4}中,可将图\ref{1530}所示衰变过程的振幅写为
\begin{equation}
	\label{eq3a6}
	\mathcal{T}'=A\mel**{m}{\Big(\frac{2}{3}\delta_{ij}-\frac{i}{3}\epsilon_{ijk}\sigma_{k}\Big)}{m'}\frac{\tilde{p}_{\pi_{L},i}p_{\pi_{H},j}}{M_{\text{inv}}-M_{\Xi(1530)}+i\dfrac{\Gamma_{\Xi(1530)}}{2}}
\end{equation}
由式\eqref{eq3a6},可计算振幅$\mathcal{T}'$的模方,得到
\begin{equation}
	\label{eq3a7}
\begin{aligned}[b]
	\sum_{mm'}\abs{\mathcal{T}'}^2=&\sum_{mm'}\mathcal{T}'\mathcal{T}'^{\dagger}\\
	=&A^2 \frac{1}{\abs{M_{\text{inv}}-M_{\Xi(1530)}+i\frac{\Gamma_{\Xi(1530)}}{2}}^2}X,
\end{aligned}
\end{equation}
其中
\begin{equation}
\begin{split}
	X\equiv\sum_{mm'}&\mel**{m}{\Big(\frac{2}{3}\delta_{ij}-\frac{i}{3}\epsilon_{ijk}\sigma_{k}\Big)}{m'}\tilde{p}_{\pi_{L},i}p_{\pi_{H},j}\\[1ex]
			      &\times\mel**{m'}{\Big(\frac{2}{3}\delta_{i'j'}+\frac{i}{3}\epsilon_{i'j'k'}\sigma_{k'}^{\dagger}\Big)}{m}\tilde{p}_{\pi_{L},i'}p_{\pi_{H},j'},
\end{split}
\end{equation}
利用
\begin{equation}
	\sum_{m'}\ketbra{m'}{m'}=1.
\end{equation}
\newpage
$X$可写为
\begin{equation}
\label{eq3a8}
\begin{aligned}[b]
	X=&\sum_{m}\mel**{m}{\Big(\frac{2}{3}\delta_{ij}-\frac{i}{3}\epsilon_{ijk}\sigma_{k}\Big)\Big(\frac{2}{3}\delta_{i'j'}+\frac{i}{3}\epsilon_{i'j'k'}\sigma_{k'}\Big)}{m}\\
		    &\qquad\times\tilde{p}_{\pi_{L},i}p_{\pi_{H},j}\tilde{p}_{\pi_{L},i'}p_{\pi_{H},j'}\\[1ex]
	=&\sum_{m}\mel**{m}{\Big(\frac{4}{9}\delta_{ij}\delta_{i'j'}+i\frac{2}{9}\delta_{ij}\epsilon_{i'j'k'}\sigma_{k'}-i\frac{2}{9}\delta_{i'j'}\epsilon_{ijk}\sigma_{k}+\frac{1}{9}\epsilon_{ijk}\epsilon_{i'j'k'}\sigma_{k}\sigma_{k'}\Big)}{m}\\
	 &\qquad\times\tilde{p}_{\pi_{L}i}p_{\pi_{H}j}\tilde{p}_{\pi_{L}i'}p_{\pi_{H}j'}\\[1ex]
	=&\sum_{m}\mel**{m}{\Big[\frac{4}{9}\delta_{ij}\delta_{i'j'}+\frac{1}{9}\epsilon_{ijk}\epsilon_{i'j'k'}(\delta_{kk'}+i\epsilon_{kk'c}\sigma_{c})\Big]}{m}\\
	 &\qquad\times\tilde{p}_{\pi_{L}i}p_{\pi_{H}j}\tilde{p}_{\pi_{L}i'}p_{\pi_{H}j'}\\[1ex]
	=&\sum_{m}\mel**{m}{\Big(\frac{4}{9}\delta_{ij}\delta_{i'j'}+\frac{1}{9}\delta_{ii'}\delta_{jj'}-\frac{1}{9}\delta_{ij'}\delta_{i'j}\Big)}{m}\\
	 &\qquad\times\tilde{p}_{\pi_{L}i}p_{\pi_{H}j}\tilde{p}_{\pi_{L}i'}p_{\pi_{H}j'}\\[1ex]
	=&\frac{4}{9}(\vec{p}_{\pi_{H}}\cdot \vec{\tilde{p}}_{\pi_{L}})^2+\frac{1}{9}p_{\pi_{H}}^2\tilde{p}_{\pi_{L}}^2-\frac{1}{9}(\vec{p}_{\pi_{H}}\cdot \vec{\tilde{p}}_{\pi_{L}})^2\\[1ex]
	=&\frac{1}{3}(\vec{p}_{\pi_{H}}\cdot \vec{\tilde{p}}_{\pi_{L}})+\frac{1}{9}p_{\pi_{H}}^2\tilde{p}_{\pi_{L}}^2,
\end{aligned}
\end{equation}
上式的导出中利用了如下关系式,
\begin{align}
	&\sigma_{i}^{\dagger}=\sigma_{i},\\[1ex]
	&\sigma_{a}\sigma_{b}=i\epsilon_{abc}\sigma_{c}+\delta_{ab},\\[1ex]
	&\braket{m}{m}=1,\\[1ex]
	&\sum_{m}\mel{m}{\sigma_{i}}{m}=0,\\[1ex]
	&\epsilon_{ijk}\epsilon_{i'j'k}=\delta_{ii'}\delta_{jj'}-\delta_{ij'}\delta_{ji'}.
\end{align}
在式\eqref{eq3a8}中含有$(\vec{p}_{\pi_{H}}\cdot\vec{\tilde{p}}_{\pi_{L}})^2$项,可以写成$(\vec{p}_{\pi_{H}}\cdot\vec{\tilde{p}}_{\pi_{L}})^2=p_{\pi_{H}}^2\tilde{p}_{\pi_{L}}^2\cos^2\theta$,$\theta$为两个动量间的夹角。因为两个$\pi$的动量方向没有限制,所以应该是各个方向都有相同的可能性,这样就可以做一个平均,
\begin{equation}
	\frac{\int_{0}^{2\pi}\dd{\phi}\int_{0}^{\pi}\sin\theta\dd{\theta}\cos^2\theta}{\int_{0}^{2\pi}\dd{\phi}\int_{0}^{\pi}\sin\theta\dd{\theta}}=\frac{1}{3},
\end{equation}
这样$(\vec{p}_{\pi_{H}}\cdot\vec{\tilde{p}}_{\pi_{L}})^2$就可以写为$\frac{1}{3}p_{\pi_{H}}^2\tilde{p}_{\pi_{L}}^2$。最后式\eqref{eq3a8}就可以写为
\begin{equation}
\label{eq3a9}
\begin{aligned}[b]
	X=&\frac{1}{3}\frac{1}{3}p_{\pi_{H}}^2\tilde{p}_{\pi_{L}}^2+\frac{1}{9}p_{\pi_{H}}^2\tilde{p}_{\pi_{L}}^2\\[1ex]
	=&\frac{2}{9}p_{\pi_{H}}^2\tilde{p}_{\pi_{L}}^2.
\end{aligned}
\end{equation}\par
把式\eqref{eq3a9}代入\eqref{eq3a7},振幅$\mathcal{T}'$的模方可写为
\begin{equation}
\label{eq331}
\begin{aligned}[b]
	\sum_{mm'}\abs{\mathcal{T}'}^2=&A^2\frac{\frac{2}{9}p_{\pi_{H}}^2\tilde{p}_{\pi_{L}}^2 }{\abs{M_{\text{inv}}-M_{\Xi(1530)}+i\frac{\Gamma_{\Xi(1530)}}{2}}^2}\\[1ex]
	=&V_A^2\frac{p_{\pi_{H}}^2\tilde{p}_{\pi_{L}}^2}{\abs{M_{\text{inv}}-M_{\Xi(1530)}+i\frac{\Gamma_{\Xi(1530)}}{2}}^2}\ .
\end{aligned}
\end{equation}
上式即为来自$\Xi(1530)$共振态的贡献,式中$p_{\pi_{H}}$是动量较高的$\pi^{+}$介子在$\Xi_{c}^{+}$静止系中的三动量大小,$\tilde{p}_{\pi_{L}}$是动量较低的$\pi^{+}$介子在$\Xi(1530)$静止系(或$\Xi^{-}\pi_{L}^{+}$质心系)中的三动量大小,
\begin{align}
	p_{\pi_{H}}=&\frac{\lambda^{\frac{1}{2}}(M_{\Xi_{c}^{+}}^2,m_{\pi^{+}}^2,M_{\text{inv}}^2)}{2M_{\Xi_{c}^{+}}},\\[1ex]
	\tilde{p}_{\pi_{L}}=&\frac{\lambda^{\frac{1}{2}}(M_{\text{inv}}^2,m_{\pi^{+}}^2,M_{\Xi^{-}}^2)}{2M_{\text{inv}}},\\[1ex]
	\lambda(x,y,z)=&x^2+y^2+z^2-2xy-2yz-2xz.
\end{align}
\subsection{其它背景道的贡献}
对于$\Xi_{c}^{+}\to\Xi^{-}\pi^{+}\pi^{+}$衰变过程,当中间态$J=\frac{3}{2}$时,除了\ref{subsec1530}节所讨论的$\Xi(1530)(\frac{3}{2}^{+})$共振态的贡献外,还有非共振态的贡献,
\begin{equation}
\label{eq3a10}
\mathcal{\widetilde{T}}'\propto p^2_{\pi_{H}}\tilde{p}_{\pi_{L}}^2.
\end{equation}
综合式\eqref{eq331}和\eqref{eq3a10},可得到$J=\frac{3}{2}$的中间态对$\Xi_{c}^{+}\to\Xi^{-}\pi^{+}\pi^{+}$衰变振幅的贡献为
\begin{equation}
\label{eq332}
\abs{\mathcal{T}'}^2=V_A^2\frac{p_{\pi_{H}}^2\tilde{p}_{\pi_{L}}^2}{\abs{M_{\text{inv}}-M_{\Xi(1530)}+i\frac{\Gamma_{\Xi(1530)}}{2}}^2}+V_B^2p_{\pi_{H}}^2\tilde{p}_{\pi_{L}}^2,
\end{equation}
其中$V_A$和$V_B$为常数,在计算中作为可调参数。上式中第一项是来自$\Xi(1530)(\frac{3}{2}^{+})$共振态的贡献,第二项是非共振态贡献。\par
\begin{figure}[h]
	\centering
	\includegraphics[width=.6\linewidth]{other}
	\caption[$\Xi(\frac{1}{2}^{+})$极点图]{$\Xi_{c}^{+}\to\Xi^{-}\pi^{+}\pi^{+}$衰变过程的$\Xi(\frac{1}{2}^{+})$极点图。}
	\label{other}
\end{figure}
除此之外,还有另一种背景贡献,即$\Xi(\frac{1}{2}^{+})$极点图的贡献,如图\ref{other}所示。图中第一个顶角所对应的$\Xi_{c}^{+}\to\Xi(\frac{1}{2}^{+})\pi_{H}^{+}(0^{-})$弱衰变过程中宇称不守恒,因而$\Xi(\frac{1}{2}^{+})\pi_{H}^{+}(0^{-})$的相对轨道角动量$L=0$即可满足角动量守恒的要求,即$\Xi_{c}^{+}\to\Xi(\frac{1}{2}^{+})\pi_{H}^{+}(0^{-})$为$s$-波衰变。第二个顶角相对应的$\Xi(\frac{1}{2}^{+})\to\Xi^{-}(\frac{1}{2}^{+})\pi_{L}^{+}(0^{-})$强衰变过程中,宇称和角动量均为守恒量,要求该衰变为$p$-波衰变。由此,可写出图\ref{other}所示的$\Xi_{c}^{+}\to\Xi^{-}\pi^{+}\pi^{+}$的$\Xi(\frac{1}{2}^{+})$极点图的衰变振幅模方为
\begin{equation}
\label{eq333}
	\abs{\mathcal{T}''}^2=V_C^2\abs{\tilde{p}_{\pi_{L}}}^2,
\end{equation}
其中$V_C$为常数因子。
\section{$\Xi^{-}\pi_{L}^{+}$不变质量谱}
\subsection{$\Xi^{-}\pi_{L}^{+}$不变质量谱的理论计算公式}
\ref{sec33}节和\ref{sec34}节分别计算了$\Xi_{c}^{+}\to\Xi^{-}\pi^{+}\pi^{+}$衰变中来自$J^{P}=\frac{1}{2}^{-}$的$\Xi(1620)$和$\Xi(1690)$共振态的贡献以及其它的机制(背景道)的贡献。综合起来,$\Xi_{c}^{+}\to\Xi^{-}\pi^{+}\pi^{+}$衰变振幅模方可写为
\begin{equation}
	\abs{T}^2=\abs{\mathcal{T}}^2+\abs{\mathcal{T}'}^2+\abs{\mathcal{T}''}^2,
\end{equation}
其中$\abs{\mathcal{T}}^2$是来自$\Xi(1620)$和$\Xi(1690)$共振态的贡献,由式\eqref{eqiT}给出;$\abs{\mathcal{T}'}^2$和$\abs{\mathcal{T}''}^2$是背景道的贡献,由式\eqref{eq332}和\eqref{eq333}给出,
\begin{equation}
\label{eq3a11}
\abs{T}^2=\abs{\mathcal{T}}^2+V_A^2\frac{p_{\pi_{H}}^2\tilde{p}_{\pi_{L}}^2}{\abs{M_{\text{inv}}-M_{\Xi(1530)}+i\frac{\Gamma_{\Xi(1530)}}{2}}^2}+V_B^2p_{\pi_{H}}^2\tilde{p}_{\pi_{L}}^2+V_C^2\tilde{p}_{\pi_{L}}^2,
\end{equation}
上式包含了4个参数: $V_{P}^2$、$V_A^2$、$V_B^2$和$V_C^2$。\par
于是可得到$\Xi_{c}^{+}\to\Xi^{-}\pi^{+}_{L}\pi^{+}_{H}$衰变过程的末态$\Xi^{-}\pi_{L}^{+}$不变质量分布的表达式为
\begin{equation}
\label{invmass}
\dv{\Gamma}{M_{\text{inv}}}=\frac{1}{(2\pi)^{3}}\frac{1}{4M_{\Xi_{c}}^2}p_{\pi_{H}}\tilde{p}_{\pi_{L}}\abs{T}^2,
\end{equation}
其中$\abs{T}^2$如式\eqref{eq3a11}和\eqref{eq323}所示。
\subsection{计算结果及分析讨论}
由式\eqref{invmass}、\eqref{eq3a11}和\eqref{eq323}可以计算出$\dfrac{\dd\Gamma}{\dd M_{\text{inv}}}$,给出$\Xi_{c}^{+}\to\Xi^{-}\pi_{L}^{+}\pi_{H}^{+}$衰变的$\Xi^{-}\pi_{L}^{+}$不变质量谱的理论预言。理论计算中有4个可调参数:
\begin{equation}
\label{eq3a12}
	V_{P}^2,\quad V_A^2,\quad V_B^2,\quad V_C^2.
\end{equation}
通过式\eqref{eq3a11}和图\ref{ex}所示的$\Xi^{-}\pi_{L}^{+}$不变质量观测谱,我们可以粗略地作如下预测:式\eqref{eq3a11}的最后两项可能给出图\ref{ex}中较平缓的黑色虚线所描写的非共振本底贡献,因此调节参数$V_B^2$和$V_C^2$可改变较平缓的非共振本底贡献;式\eqref{eq3a11}中的第二项是来自$\Xi(1530)(\frac{3}{2}^{+})$共振态的贡献,调节参数$V_A^2$可改变$\dfrac{\dd\Gamma}{\dd M_{\text{inv}}}$中$\Xi(1530)$共振峰的高度,或者说可通过拟合图\ref{ex}中$\Xi(1530)$共振峰的高度来确定参数$V_A^2$;式\eqref{eq3a11}中的第一项是来自$\Xi(1620)$和$\Xi(1690)$共振态的贡献,这一项正比于参数$V_{P}^2$,调节$V_{P}^2$可改变$\dfrac{\dd\Gamma}{\dd M_{\text{inv}}}$中$\Xi(1620)$或$\Xi(1690)$共振峰的高度,因此可通过拟合图\ref{ex}中$\Xi(1620)$或$\Xi(1690)$共振峰的高度来确定参数$V_{P}^2$。\par
我们对$\Xi_{c}^{+}\to\Xi^{-}\pi^{+}_{L}\pi^{+}_{H}$衰变过程的$\Xi^{-}\pi^{+}_{L}$不变质量分布$\dfrac{\dd\Gamma}{\dd M_{\text{inv}}}$的理论计算结果将包含如下预言:
\begin{enumerate}[1)]
	\item 共振态$\Xi(1620)$和$\Xi(1690)$的位置和宽度;
	\item 该衰变过程中中间共振态$\Xi(1620)$和$\Xi(1690)$产生的相对强度。
\end{enumerate}
值得指出的是,尽管我们的理论计算中有式\eqref{eq3a12}所示的4个参数,但可以预期以上两个预言所包含的信息对参数$V_{P}^2$、$V_{A}^2$、$V_{B}^2$和$V_{C}^2$的依赖性不会很大,因为这些信息已由第二章中所讨论的$\pi\Xi$耦合道散射、$\Xi(1620)$和$\Xi(1690)$的动力学产生机制及它们的结构特性所决定。因此,将我们理论计算得到的$\Xi^{-}\pi_{L}^{+}$不变质量分布$\dfrac{\dd\Gamma}{\dd M_{\text{inv}}}$与Belle实验观测到的$\Xi^{-}\pi_{L}^{+}$不变质量谱进行比较,可检验$\Xi(1620)$和$\Xi(1690)$的分子态结构特性。\par
图\ref{7501}、图\ref{7502}和图\ref{7701}是采用参数$q_{\text{max}}$、$V_{P}^2$、$V_{A}^2$、$V_{B}^2$、$V_{C}^2$的三组不同取值计算得到的$\Xi^{-}\pi_{L}^{+}$不变质量分布与实验数据的比较。
\begin{figure}[!h]
	\centering
	\includegraphics[width=\linewidth]{newi7501}
	\begin{spacing}{1.5}
		\caption[$\Xi^{-}\pi^{+}$不变质量分布(第一组参数)]{$\Xi_{c}^{+}\to\pi^{+}_{H}\pi^{+}_{L}\Xi^{-}$衰变过程的末态$\Xi^{-}\pi_{L}^{+}$不变质量分布,带误差棒的蓝色点是Belle合作组的实验数据\cite{PhysRevLett.122.072501},黑色实线是基于式\eqref{invmass}的理论计算结果。[所取参数:$q_{\text{max}}=\SI{750}{MeV}$, $V_{P}^2=8.8\times10^7$, $V_A^2=6.2\times10^{-2}$, $V_B^2=2.5\times10^{-5}$, $V_C^2=10$]}
\end{spacing}
	\label{7501}
\end{figure}
\begin{figure}[!h]
	\centering
	\includegraphics[width=\linewidth]{newi7502}
	\begin{spacing}{1.5}
\caption[$\Xi^{-}\pi^{+}$不变质量分布(第二组参数)]{$\Xi_{c}^{+}\to\pi^{+}_{H}\pi^{+}_{L}\Xi^{-}$衰变过程的末态$\Xi^{-}\pi_{L}^{+}$不变质量分布,带误差棒的蓝色点是Belle合作组的实验数据\cite{PhysRevLett.122.072501},黑色实线是基于式\eqref{invmass}的理论计算结果。[所取参数:$q_{\text{max}}=\SI{750}{MeV}$, $V_{P}^2=8.5\times10^7$, $V_A^2=6.2\times10^{-2}$, $V_B^2=4\times10^{-5}$, $V_C^2=7$]}
\end{spacing}
	\label{7502}
\end{figure}
\begin{figure}[!h]
	\centering
	\includegraphics[width=\linewidth]{newi7701}
	\begin{spacing}{1.5}
\caption[$\Xi^{-}\pi^{+}$不变质量分布(第三组参数)]{$\Xi_{c}^{+}\to\pi^{+}_{H}\pi^{+}_{L}\Xi^{-}$衰变过程的末态$\Xi^{-}\pi_{L}^{+}$不变质量分布,带误差棒的蓝色点是Belle合作组的实验数据\cite{PhysRevLett.122.072501},黑色实线是基于式\eqref{invmass}的理论计算结果。[所取参数:$q_{\text{max}}=\SI{770}{MeV}$, $V_{P}^2=8.5\times10^7$, $V_A^2=6.2\times10^{-2}$, $V_B^2=2.5\times10^{-5}$, $V_C^2=7$]}
\end{spacing}
	\label{7701}
\end{figure}\par\newpage
图\ref{7501}和图\ref{7502}对应的$q_{\text{max}}$相同,但其它参数不同。可以看到,对于我们感兴趣的能区$[1550,1700]\,\si{MeV}$,这两种情况得到的$\Xi^{-}\pi_{L}^{+}$不变质量分布的计算结果都能与Belle的实验数据符合得相当好。因此,确实如我们所预测的,$\Xi(1620)$和$\Xi(1690)$共振态在$\Xi^{+}\to\pi^{+}\pi^{+}\Xi^{-}$的衰变过程中的相对产生强度对$V_{P}^2$、$V_{A}^2$、$V_{B}^2$和$V_{C}^2$等参数的依赖性不大。\par
图\ref{7701}对应于$q_{\text{max}}=\SI{770}{MeV}$。与$q_{\text{max}}=\SI{750}{MeV}$所得到的图\ref{7501}和\ref{7502}的$\Xi^{-}\pi_{L}^{+}$不变质量谱相比,图\ref{7701}所示的$\Xi^{-}\pi^{+}_{L}$不变质量分布中$\Xi(1620)$和$\Xi(1690)$共振峰的峰值位置均向能量低端移动了一点,两个峰的相对高度也有所偏差。相比之下,图\ref{7501}和图\ref{7502}所示的$\Xi^{-}\pi_{L}^{+}$不变质量分布的理论结果与实验结果符合度更好一些。\par
综上,我们在手征幺正法的理论框架下,考察了奇异数$S=-2$、同位旋$I=\frac{1}{2}$、自旋-宇称$J^{P}=\frac{1}{2}^{-}$的轻赝标介子-重子$s$-波相互作用体系,包括了$\pi\Xi$、$\bar{K}\Lambda$、$\bar{K}\Sigma$和$\eta\Xi$这四个相互耦合的同位旋反应道,通过求解耦合道的BS方程,得到$\pi\Xi$及其耦合道散射的全散射振幅$T$。$J^{P}=\frac{1}{2}^{-}$的超子激发态$\Xi(1620)$和$\Xi(1690)$作为$T$振幅的极点由$\pi\Xi$(及其耦合道)的$s$-波相互作用动力学地产生,具有介子-重子分子态的结构特性。由动力学产生态与各反应道的耦合常数可知,$\Xi(1620)$的主要成分是$\pi\Xi$和$\bar{K}\Lambda$,而$\Xi(1690)$的主要成分则为$\bar{K}\Sigma$。\par
为检验超子激发态$\Xi(1620)$和$\Xi(1690)$的分子态结构特性,我们研究了$\Xi(1620)$和$\Xi(1690)$共振态在$\Xi_{c}^{+}\to\Xi^{-}\pi^{+}\pi^{+}$衰变过程中的动力学产生机制,计算了这两个共振态对上述衰变过程末态$\Xi^{-}\pi^{+}$不变质量谱的贡献。为将理论计算所得到的$\Xi^{-}\pi^{+}$不变质量分布与Belle合作组对$\Xi_{c}^{+}\to\Xi^{-}\pi^{+}\pi^{+}$测量得到的$\Xi^{-}\pi^{+}$不变质量谱实验结果进行比较,我们还计算了背景道的贡献,包括$\Xi(1530)(\frac{3}{2}^{+})$共振态的贡献以及非共振贡献。\par
本工作的研究结果表明,将$\Xi(1620)$和$\Xi(1690)$作为自旋-宇称$J^{P}=\frac{1}{2}^{-}$的介子-重子分子态,计算得到的$\Xi_{c}^{+}\to\Xi^{-}\pi^{+}\pi^{+}$衰变过程$\Xi^{-}\pi^{+}$不变质量分布的理论结果与Belle的实验结果相符(特别是在$\Xi(1620)$和$\Xi(1690)$的能区范围$[1550,1700]\,\si{MeV}$内)。由于在理论计算所得到的$\Xi^{-}\pi^{+}$不变质量谱中,共振态$\Xi(1620)$和$\Xi(1690)$的位置和宽度,以及这两个共振态产生的相对强度是由第二章所讨论的$\pi\Xi$耦合道散射、$\Xi(1620)$和$\Xi(1690)$的动力学产生机制及它们的分子态结构特性所决定的,因此理论计算所得到的$\Xi^{-}\pi^{+}$不变质量分布与Belle实验结果的符合支持了$\Xi(1620)$和$\Xi(1690)$共振态的分子态结构图像。\par
为了确定$\Xi(1620)$和$\Xi(1690)$的自旋-宇称量子数和内部结构,还需要做更多的工作,包括实验上的和理论上的。实验上需要提供关于$\Xi(1620)$和$\Xi(1690)$共振态的更多、更精确的实验数据和信息。理论上,需要对比这两个共振态的不同结构图像在各种物理反应过程的可观测效应的差异。
