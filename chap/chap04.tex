 %--------------------------------------------------------------
\chapter{总结与展望}\label{chap4}
% 这是我毕业论文的总结和展望部分:

本文详细推导并复现了PPSDNN算法,实现了PPSDNN算法对toy model的学习,在toy model中,目标函数是连续可导的,其傅立叶变换也是解析的,根据其傅立叶变换得到的采样频率,通过平行相移,就可以很好的学习toy model。在实际应用中,训练数据是离散的,比如$^{235}\text{U}(n,f)$反应的评价数据,通过frequency sweep,也可以做到较好的学习这类离散数据。这表明该算法确实具有较高的预测精度和效率和较好高频数据学习能力。
这为今后从事中子共振截面的评价工作奠定了坚实的基础。

机器学习用于中子共振研究还有以下潜在应用:
\begin{enumerate}
    \item 加速计算\cite{verbraeken2020survey}。中子共振计算中需要进行大量的矩阵运算和复杂的数值计算,通常非常耗时。使用机器学习算法,可以通过对之前计算过的数据进行训练,从而加速计算过程并保持相同的计算精度。
    \item 提高计算精度。中子共振计算中存在着一些难以处理的误差和不确定性。使用机器学习算法可以通过对大量数据进行训练,从而减小误差和降低不确定性,提高计算精度。
    \item 数据分析。中子共振实验通常产生大量数据,如能谱和截面等。使用机器学习算法,可以对这些数据进行分析和分类,更好地理解中子共振现象和物理机制。
    \item 拟合物理模型参数。中子共振研究中需要对物理模型进行拟合,以更好地描述中子共振现象和物理机制。使用机器学习算法,可以对大量的实验数据进行训练,以获得更好的拟合效果和更准确的模型参数。随着中子共振核反应数据量的增加,机器学习方法也逐渐体现出其独有的优势。
\end{enumerate}

在今后的工作中,将继续发挥机器学习的优势,并将PPSDNN方法用于学习以$^{235}\text{U}(n,f)$反应为代表的中子共振实验数据中。并使用学习和预言得到的激发曲线给出核反应的共振参数。

目前得到共振参数的主流方法是使用R矩阵拟合实验数据。
尽管R矩阵方法已经被广泛应用,但仍然存在一些困难和限制。例如,该方法只能唯象地给出道半径R,无法清晰地解释反应机制,而且R矩阵程序的编写也较为困难。为此,在今后的科研中计划将PPSDNN方法和R矩阵相结合,以改进共振参数的计算方法。这一方法将利用PPSDNN算法的高预测精度和高效率,以及R矩阵的核反应共振的原理,提高共振参数的准确性和高效性。