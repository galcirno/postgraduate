%多行公式多个对齐点
%\begin{equation}
%\begin{aligned}
%	a&=x&&ysdfs\\
%	b&=cdsfsd&&z
%\end{aligned}
%\end{equation}
\chapter[$\Xi\pi$相互作用及\\超子$\Xi(1620)$和$\Xi(1690)$的动力学产生]{$\Xi\pi$相互作用及超子$\Xi(1620)$和$\Xi(1690)$的动力学产生}
\section{Bethe-Salpeter (BS)方程}
\subsection{李普曼-史温格方程(Lippmann-Schwinger equation)}
李普曼-史温格方程是散射理论中非常重要的方程,文献\cite{PhysRev.79.469}中给出了由薛定谔方程得到李普曼-史温格方程的详细过程。\par
在量子力学中,定态薛定谔方程为
\begin{equation}
\label{Sq}
\hat{H}\ket{\psi}=E_{i}\ket{\psi},
\end{equation}
其中,体系的哈密顿量可写为
\begin{equation}
\begin{split}
	\hat{H}=\hat{H}_{0}+\hat{V}&=\frac{1}{2m}\hat{P}^2+\hat{V},\\[2ex]
	E_{i}=&\frac{1}{2m}p_{i}^2.
\end{split}
\end{equation}
在散射过程中,散射定态是平面波与散射态的叠加,即$\ket{\psi}=\ket{k_{i}}+\ket{\psi'}$,$\ket{k_{i}}$是平面波,平面波所满足的薛定谔方程为
\begin{equation}
\label{pweq}
\hat{H}_{0}\ket{k_{i}}=E_{i}\ket{k_{i}},
\end{equation}
定态散射的波函数$\ket{\psi}$所满足的薛定谔方程为
\begin{equation}
\label{Scatter Sq}
\begin{split}
	\hat{H}\ket{\psi}=&(\hat{H}_{0}+\hat{V})\ket{\psi}\\
	=&E_{i}\ket{\psi}.
\end{split}
\end{equation}
把公式\eqref{Scatter Sq}做一下处理,再把$\ket{\psi}=\ket{k_{i}}+\ket{\psi'}$代入,得到
\begin{equation}
	\begin{aligned}[b]
	V\ket{\psi}=&(E_{i}-\hat{H}_{0})\ket{\psi}\\
	=&(E_{i}-\hat{H}_{0})(\ket{k_{i}}+\ket{\psi'})\\
	=&(E_{i}-\hat{H}_{0})\ket{\psi'}.
	\end{aligned}
\end{equation}
在方程两边乘以$E_{i}-\hat{H}_{0}$的逆算符,就可以得到$\ket{\psi'}$,
\begin{equation}
\ket{\psi'}=\frac{1}{E_{i}-\hat{H}_{0}\pm i\epsilon}\hat{V}\ket{\psi},
\end{equation}
式中$\epsilon$是一个一阶无穷小量,再计算到最后会做$\epsilon\to 0$的极限。最后可以得到李普曼-史温格方程
\begin{equation}
\label{LSeq}
\ket{\psi^{\pm}}=\ket{k}+\frac{1}{E_{i}-\hat{H}_{0}\pm i\epsilon}\hat{V}\ket{\psi^{\pm}}.
\end{equation}
把公式\eqref{LSeq}做以下变换
\begin{equation}
\begin{split}
	&(E_{i}-\hat{H}_{0}\pm i\epsilon)\ket{\psi^{\pm}}=(E_{i}-\hat{H}_{0}\pm i\epsilon)\ket{k}+\hat{V}\ket{\psi^{\pm}},\\
	&(E_{i}-\hat{H}_{0}\pm i\epsilon)\ket{\psi^{\pm}}-\hat{V}\ket{\psi^{\pm}}=(E_{i}-\hat{H}_{0}\pm i\epsilon)\ket{k},\\
	&(E_{i}+\hat{H}\pm i\epsilon)\ket{\psi^{\pm}}=(E_{i}-\hat{H}\pm i\epsilon)\ket{k}+\hat{V}\ket{k},
\end{split}
\end{equation}
可以得到另一种形式的李普曼-史温格方程,
\begin{equation}
\label{LSeq2}
\ket{\psi^{\pm}}=\ket{k}+\frac{1}{E_{i}-\hat{H}\pm i\epsilon}\hat{V}\ket{k}.
\end{equation}
\subsection{$\hat{T}$算符}
在散射理论中,散射振幅为
\begin{equation}
	f(\theta,\phi)=%-\frac{(2\pi)^{\frac{3}{2}}}{4\pi}\frac{2m}{\hbar^2}\int e^{-ik_{f}\vdot r'}V(r')\psi^{\dagger}_{P_{i}}(r')\dd r'=\
	-\frac{4\pi^2m}{\hbar^2}\mel{k_{f}}{V}{\psi^{+}_{P_{i}}}.
\end{equation}
定义$\hat{T}$算符
\begin{equation}
\label{T operators def}
	\hat{T}^{\pm}\ket{k_{i}}=\hat{V}\ket{\psi^{\pm}_{P_{i}}},
\end{equation}
散射振幅就可以写成
\begin{equation}
	f(\theta,\phi)=-\frac{4\pi^2m}{\hbar^2}\mel{k_{f}}{T^{+}}{k_{i}}.
\end{equation}
用算符$\hat{V}$左乘公式\eqref{LSeq}可以得到
\begin{equation}
	\hat{V}\ket{\psi_P^{\pm}}=\hat{V}\ket{k}+\hat{V}\hat{G}_{0}\hat{V}\ket{\psi_P^{\pm}},
\end{equation}
再把$\hat{T}$算符的定义代入,得到
\begin{equation}
	\hat{T}^{\pm}\ket{k}=\hat{V}\ket{k}+\hat{V}\hat{G}_{0}\hat{T}^{\pm}\ket{k},
\end{equation}
即
\begin{equation}
	\hat{T}^{\pm}=\hat{V}+\hat{V}\hat{G}_{0}\hat{T}^{\pm}.
\end{equation}
其中$\hat{G}_{0}=\frac{1}{E_{i}-\hat{H}_{0}\pm i\epsilon}$。同理把$\hat{V}$左乘公式\eqref{LSeq2}可得
\begin{equation}
	\hat{T}^{\pm}=\hat{V}+\hat{V}\hat{G}\hat{V},
\end{equation}
其中$\hat{G}=\frac{1}{E_{i}-\hat{H}\pm i\epsilon}$。
\subsection{$\hat{S}$算符}
定义两个摩勒算符$\hat{\Omega}^{\pm}$,
\begin{equation}
\label{moller operations}
\hat{\Omega}^{\pm}\ket{k}=\ket{\psi_P^{\pm}},
\end{equation}
摩勒算符可以把平面波态变成散射态。摩勒算符可写成
\begin{equation}
	\label{moller operations def}
	\hat{\Omega}^{\pm}=\sum_{k}\dyad{\psi_P^{\pm}}{k}.
\end{equation}
还可用时间演化算符来表示\cite{weinberg1995quantum}
\begin{equation}
	\hat{U}(\tau,t)\ket{\psi_P^{+}(t)}=\hat{U}_{0}(\tau,t)\ket{k},
\end{equation}
式中$\tau\to-\infty$,时间演化算符分别为
\begin{equation}
\begin{split}
	\hat{U}(\tau,t)=e^{-\frac{i}{\hbar}\hat{H}(\tau-t)},\\
	\hat{U}_{0}(\tau,t)=e^{-\frac{i}{\hbar}\hat{H}_{0}(\tau-t)}.
\end{split}
\end{equation}
因此我们可以得到
\begin{equation}
	\hat{\Omega}^{+}=\lim_{\tau\to-\infty}e^{\frac{i}{\hbar }\hat{H}\tau}e^{-\frac{i}{\hbar }\hat{H}_{0}\tau}.
\end{equation}
同理,另一个摩勒算符为
\begin{equation}
	\hat{\Omega}^{-}=\lim_{\tau\to+\infty}e^{\frac{i}{\hbar }\hat{H}\tau}e^{-\frac{i}{\hbar }\hat{H}_{0}\tau}.
\end{equation}\par
现在引入散射算符$\hat{S}$的定义,
\begin{equation}
\label{S matrix}
S_{fi}=\mel{k_{f}}{\hat{S}}{k_{i}}=\braket{\psi^{-}_{Pf}}{\psi^{+}_{Pi}},
\end{equation}
可以用摩勒算符把$\hat{S}$算符表示出来
\begin{equation}
	\hat{S}=(\hat{\Omega}^{-})^{\dagger}\hat{\Omega}^{+}.
\end{equation}
然后可以计算$\hat{S}^{\dagger}\hat{S}$和$\hat{S}\hat{S}^{\dagger}$,
\begin{equation}
	\begin{aligned}[b]
	\hat{S}^{\dagger}\hat{S}=&(\hat{\Omega}^{+})^{\dagger}\hat{\Omega}^{-}(\hat{\Omega}^{-})^{\dagger}\hat{\Omega}^{+}\\
	=&\lim_{\tau\to+\infty}(e^{-\frac{i}{\hbar }\hat{H}_{0}\tau}e^{\frac{i}{\hbar }\hat{H}\tau})(e^{\frac{i}{\hbar }\hat{H}\tau}e^{-\frac{i}{\hbar }\hat{H}_{0}\tau})(e^{\frac{i}{\hbar }\hat{H}_{0}\tau}e^{-\frac{i}{\hbar }\hat{H}\tau})(e^{-\frac{i}{\hbar }\hat{H}\tau}e^{\frac{i}{\hbar }\hat{H}_{0}\tau})\\
	=&1.
	\end{aligned}
\end{equation}
同理,可以得到$\hat{S}\hat{S}^{\dagger}=1$,所以$\hat{S}$算符是幺正算符。还可以利用公式\eqref{moller operations def}得到算符$\hat{S}$是幺正算符的结论,
\begin{equation}
	\begin{aligned}[b]
	\hat{S}^{\dagger}\hat{S}=&(\hat{\Omega}^{+})^{\dagger}\hat{\Omega}^{-}(\hat{\Omega}^{-})^{\dagger}\hat{\Omega}^{+}\\
	=&\sum_{k_{1}k_{2}k_{3}k_{4}}\big(\dyad{k_{1}}{\psi^{+}_{P_1}}\big)\big(\dyad{\psi^{-}_{P_{2}}}{k_{2}}\big)\big(\dyad{k_{3}}{\psi^{-}_{P_{3}}}\big)\big(\dyad{\psi^{+}_{P_{4}}}{k_{4}}\big)\\
	=&\sum_{k_{1}k_{2}k_{4}}\ket{k_{1}}\braket{\psi^{+}_{P_{1}}}{\psi^{-}_{P_{2}}}\braket{\psi^{-}_{P_{2}}}{\psi^{+}_{P_{4}}}\bra{k_{4}}\\
	=&\sum_{k_{1}k_{2}k_{4}}\ket{k_{1}}\braket{\psi^{+}_{P_{1}}}{\psi^{+}_{P_{2}}}\braket{\psi^{+}_{P_{2}}}{\psi^{+}_{P_{4}}}\bra{k_{4}}\\
	=&\sum_{k}\dyad{k}{k}\\
	=&1,
	\end{aligned}
\end{equation}
其中利用到了$\sum_{P}\dyad{\psi^{+}_{P}}{\psi^{+}_{P}}=\sum_{P}\dyad{\psi^{-}_{P}}{\psi^{-}_{P}}$这个关系式。\par
把式\eqref{LSeq}、\eqref{LSeq2}和\eqref{T operators def}代入\eqref{S matrix}式中,就可以得到$S$算符和$T$算符的关系
\begin{equation}
	\begin{aligned}[b]
	S_{fi}=&\braket{\psi^{-}_{Pf}}{\psi^{+}_{Pi}}\\
	=&\Big[\bra{k_{f}}+\bra{k_{f}}\hat{V}\frac{1}{E_{f}-\hat{H}+i\epsilon}\Big]\ket{\psi^{+}_{Pi}}\\
	=&\braket{k_{f}}{\psi^{+}_{pi}}+\frac{1}{E_{f}-E_{i}+i\epsilon}\mel{k_{f}}{\hat{V}}{\psi^{+}_{Pi}}\\
	=&\bra{k_{f}}\Big[\ket{k_{i}}+\frac{1}{E_{i}-\hat{H}_{0}+i\epsilon}\hat{V}\ket{\psi^{+}_{Pi}}\Big]+\frac{1}{E_{f}-E_{i}+i\epsilon}\mel{k_{f}}{\hat{V}}{\psi^{+}_{Pi}}\\
	=&\braket{k_{f}}{k_{i}}+\Big[\frac{1}{E_{i}-E_{f}+i\epsilon}+\frac{1}{E_{f}-E_{i}+i\epsilon}\Big]\mel{k_{f}}{\hat{V}}{\psi^{+}_{Pi}}\\
	=&\delta_{fi}-\frac{2i\epsilon}{(E_{i}-E_{f})^2+\epsilon^2}\mel{k_{f}}{\hat{T}^{+}}{k_{i}}.
	\end{aligned}
\end{equation}
再利用狄拉克函数的关系式
\begin{equation}
	\lim_{\epsilon\to0}\frac{\epsilon}{x^2+\epsilon^2}=\pi\delta(x),
\end{equation}
就可以得到
\begin{equation}
	S_{fi}=\delta_{fi}-2\pi i\delta(E_{f}-E_{i})T^{+}_{fi},
\end{equation}
即
\begin{equation}
	\hat{S}=1-2\pi i \hat{T}^{+}.
\end{equation}
\subsection{Bethe-Salpeter (BS)方程}
为了解决手征微扰论的局限性,扩大其应用范围,可以使用BS方程\cite{OLLER1997438,KAISER1995325,OSET199899,KAISER199523,NIEVES199930,NIEVES200057}。
BS方程是一个积分方程,其形式为
\begin{equation}
	T=V+\int\frac{\dd[4]{q}}{(2\pi)^{4}}VGT,
\end{equation}
式中的$T$是相互作用全振幅,不可约振幅$V$是方程的核,$G$是传播子圈函数。
\section{手征幺正法}
把BS方程中的核取为手征微扰论中的$\order{p^2}$阶树图级振幅,
\begin{equation}
	V=\mel{f}{(-\mathcal{L}_{2})}{i}.
\end{equation}
把手征振幅的动量设为在壳的,离壳的效应归于重定义的衰变常数和质量中\cite{OLLER1997438},就可以把BS方程转化为代数方程
\begin{equation}
\begin{split}
	T=&V+VGT\\
	=&(1-VG)^{-1}V.
\end{split}
\end{equation}
\begin{figure}[h]
	\centering
	\includegraphics[width=10cm]{bs}
	\caption[BS方程示意图]{BS方程示意图。}
	\label{BS}
\end{figure}
图\ref{BS}给出了BS方程的示意图。在介子-介子散射过程中,由两介子传播子构成的传播子圈函数为
\begin{equation}
	\label{eq2-34}
	G(s)=i\int\frac{\dd[4]{q}}{(2\pi)^4}\frac{1}{(q^2-m_{1}^2+i\epsilon)[(P-q)^2-m_{2}^2+i\epsilon]},
\end{equation}
其中$m_{1}$和$m_{2}$是两个介子的质量,$s=P^2$是体系质心能量的平方,$q$是其中一个传播子所携带的动量转移。由于$q$是自由的四动量,\eqref{eq2-34}式的积分是发散的,需要用三动量截断法或者维数正规化的方法来处理。三动量截断法的过程如下,
\begin{equation}
	\label{G}
	G(s)=i\int\frac{\dd{\vec{q}}}{(2\pi)^{3}}\frac{\dd{q_{0}}}{2\pi}\frac{1}{(q_{0}^2-\vec{q}\,^2-m^2_{1}+i\epsilon)[(P_{0}-q_{0})^2-(\vec{P}-\vec{q})^2-m_{2}^2+i\epsilon]}.
\end{equation}
利用关系式$(A-B+i\epsilon)(A+B-i\epsilon)=A^2-B^2+2i\epsilon B+\epsilon^2\approx A^2-B^2+i\epsilon$,其中$\epsilon\to0$,因此
\begin{equation}
	q_{0}^2-\vec{q}\,^2-m^2_{1}+i\epsilon=(q_{0}-\sqrt{\vec{q}\,^2+m^2_{1}}+i\epsilon)(q_{0}+\sqrt{\vec{q}\,^2+m^2_{1}}-i\epsilon).
\end{equation}
在质心系中$\vec{P}=0$、$s=P_{0}^2$,所以另一部分可以写为
\begin{equation}
	(P_{0}-q_{0})^2-\vec{q}\,^2-m_{2}^2+i\epsilon=(P_{0}-q_{0}-\sqrt{\vec{q}\,^2+m^2_{2}}+i\epsilon)(P_{0}-q_{0}+\sqrt{\vec{q}\,^2+m^2_{2}}-i\epsilon),
\end{equation}
于是\eqref{G}式的$G$函数可以写成
\begin{equation}
\label{Gc}
\begin{split}
	G(s)=i\int\frac{\dd{\vec{q}}}{(2\pi)^{3}}&\frac{\dd{q_{0}}}{2\pi}\frac{1}{(q_{0}-\omega_1+i\epsilon)(q_{0}+\omega_1-i\epsilon)}\\[1ex]
	&\cdot\frac{1}{(P_{0}-q_{0}-\omega_2+i\epsilon)(P_{0}-q_{0}+\omega_2-i\epsilon)},
\end{split}
\end{equation}
式中$\omega_{1}=\sqrt{\vec{q}\,^2+m_{1}^2}$、$\omega_{2}=\sqrt{\vec{q}\,^2+m_{2}^2}$。我们先对$q_{0}$积分,利用留数定理
\begin{equation}
	\oint_{L}f(z)\dd{z}=2\pi i\sum_{k}\text{Res}f(b_{k}),
\end{equation}
对于$\int_{-\infty}^{+\infty}f(z)\dd{z}$类型的积分需要先构造环路积分,因为
\begin{equation}
	\lim_{q_{0}\to\infty}\frac{q_{0}}{(q_{0}-\omega_1+i\epsilon)(q_{0}+\omega_1-i\epsilon)(P_{0}-q_{0}-\omega_2+i\epsilon)(P_{0}-q_{0}+\omega_2-i\epsilon)}=0,
\end{equation}
所以我们可以构造上半平面的积分。在式\eqref{Gc}中总共有四个极点分别为
\begin{equation}
	q_{0}=\omega_{1}-i\epsilon,\quad q_{0}=-\omega_{1}+i\epsilon,\quad q_{0}=P_{0}-\omega_{2}+i\epsilon,\quad q_{0}=P_{0}+\omega_{2}-i\epsilon.
\end{equation}
在上半平面极点的留数为
\begin{equation}
	\begin{aligned}[b]
		\text{Res}_{1}=&\eval{\frac{(q_{0}+\omega_1-i\epsilon)}{(q_{0}-\omega_1+i\epsilon)(q_{0}+\omega_1-i\epsilon)(P_{0}-q_{0}-\omega_2+i\epsilon)(P_{0}-q_{0}+\omega_2-i\epsilon)}}_{q_{0}\to-\omega_{1}+i\epsilon}\\[1ex]
	=&\frac{1}{(-2\omega_{1}+i\epsilon)(P_{0}+\omega_{1}-\omega_{2})(P_{0}+\omega_{1}+\omega_{2}-i\epsilon)},
	\end{aligned}
\end{equation}
\begin{equation}
	\begin{aligned}[b]
		\text{Res}_{2}=&\eval{\frac{(q_{0}-P_{0}+\omega_2-i\epsilon)}{(q_{0}-\omega_1+i\epsilon)(q_{0}+\omega_1-i\epsilon)(P_{0}-q_{0}-\omega_2+i\epsilon)(P_{0}-q_{0}+\omega_2-i\epsilon)}}_{q_{0}\to P_{0}-\omega_{2}+i\epsilon}\\[1ex]
	=&\frac{1}{-(P_{0}-\omega_{1}-\omega_{2}+i\epsilon)(P_{0}+\omega_{1}-\omega_{2})(2\omega_{2}-i\epsilon)},
	\end{aligned}
\end{equation}
\eqref{Gc}式的$G$函数对$q_{0}$积分后得
\begin{equation}
	\begin{aligned}[b]
		G(s)=&-\int\frac{\dd[3]{q}}{(2\pi)^{3}}(\text{Res}_{1}+\text{Res}_{2})\\[1ex]
	=&\int\frac{\dd[3]{q}}{(2\pi)^{3}}\frac{\omega_{1}+\omega_{2}}{2\omega_{1}\omega_{2}[P_{0}-(\omega_{1}+\omega_{2})+i\epsilon][P_{0}+(\omega_{1}+\omega_{2})]}.
	\end{aligned}
\end{equation}
接下来把直角坐标系转换成球坐标系,因为被积函数与$\theta,\phi$无关,所以对角度部分的积分得到$4\pi$,传播子$G$函数可以写为
\begin{equation}
\label{Gi}
\begin{split}
	G(s)=&\frac{1}{4\pi^2}\int^{q_{\text{max}}}_{0}\dd{\abs{\vec{q\,}}}\vec{q\,}^2\frac{\omega_{1}+\omega_{2}}{\omega_{1}\omega_{2}[s-(\omega_{1}+\omega_{2})^2+i\epsilon]}\\[1ex]
	=&\frac{1}{16\pi^2s}\Biggl\{\sigma\left(\arctan\frac{s+\Delta}{\sigma\lambda_{1}}+\arctan\frac{s-\Delta}{\sigma\lambda_{2}}\right)\\
	 &\quad-\left[(s+\Delta)\ln\frac{(1+\lambda_{1})q_{\text{max}}}{m_{1}}+(s-\Delta)\ln\frac{(1+\lambda_{2})q_{\text{max}}}{m_{2}}\right]\Biggr\},
\end{split}
\end{equation}
其中$\sigma$、$\Delta$、$\lambda$分别为
\begin{equation}
\begin{split}
	\sigma=&\sqrt{-[s-(m_{1}+m_{2})^2][s-(m_{1}-m_{2})^2]},\\[1ex]
	\Delta=&m_{1}^2-m_{2}^2,\\[1ex]
	\lambda_{i}=&\sqrt{1+\frac{m_{i}^2}{q_{\text{max}}}}\ .
\end{split}
\end{equation}
\par
仔细观察式\eqref{Gi},仍然可能有一个极点,极点的位置方程为
\begin{equation}
	s-(\omega_{1}+\omega_{2})^2=0.
\end{equation}
我们可以注意到$\omega_{1}+\omega_{2}\ge m_{1}+m_{2}$。所以当的$\sqrt{s}< m_{1}+m_{2}$时,$s-(\omega_{1}+\omega_{2})^2<0$,此时不存在极点,$G(s)$为实数。但是当$\sqrt{s}>m_{1}+m_{2}$的时候,就会存在一个极点,$G(s)$变成复数。极点方程可写为
\begin{equation}
	s-\left(\sqrt{\vec{q}\,^2+m_{1}^2}+\sqrt{\vec{q}\,^2+m_{2}^2}\right)^2=0,
\end{equation}
此方程的解为
\begin{equation}
	\abs{\vec{q}\,}=\pm\frac{\sqrt{[s-(m_{1}+m_{2})^2][s-(m_{1}-m_{2})^2]}}{2\sqrt{s}}=\pm q_{\text{cm}}.
\end{equation}
其中 $q_{\text{cm}}$是介子-介子质心系中介子的三动量大小。引入一个关系式
\begin{equation}
	\frac{1}{x+i\epsilon}=\frac{x-i\epsilon}{(x+i\epsilon)(x-i\epsilon)}=\frac{x}{x^2+\epsilon^2}+i\frac{-\epsilon}{x^2+\epsilon^2}=\frac{x}{x^2+\epsilon^2}-i\pi\delta(x),
\end{equation}
上式利用了
\begin{equation}
	\lim_{\epsilon \to 0^{+}}\frac{\epsilon}{x^2+\epsilon^2}=\pi\delta(x).
\end{equation}
利用了这个关系式,\eqref{Gi}式就可以写为
\begin{equation}
	\label{Gri}
	\begin{aligned}[b]
		G(s)=&\frac{1}{4\pi^2}\int^{q_{\text{max}}}_{0}\dd{\abs{\vec{q\,}}}\vec{q\,}^2\frac{\omega_{1}+\omega_{2}}{\omega_{1}\omega_{2}[s-(\omega_{1}+\omega_{2})^2+i\epsilon]}\\[1ex]
		=&\frac{1}{4\pi^2}\int^{q_{\text{max}}}_{0}\dd{\abs{\vec{q\,}}}\vec{q\,}^2\frac{(\omega_{1}+\omega_{2})[s-(\omega_{1}+\omega_{2})^2]}{\omega_{1}\omega_{2}\{[s-(\omega_{1}+\omega_{2})^2]^2+\epsilon^2\}}\\[1ex]
	 &-\frac{i}{4\pi^2}\int^{q_{\text{max}}}_{0}\dd{\abs{\vec{q\,}}}\vec{q\,}^2\frac{\omega_{1}+\omega_{2}}{\omega_{1}\omega_{2}}\pi\delta[s-(\omega_{1}+\omega_{2})^2]\\[1ex]
	=&\Re(G)+i\Im(G).
	\end{aligned}
\end{equation}
接下来分析 $\Im(G)$。我们先来处理 $\delta[s-(\omega_{1}+\omega_{2})^2]$,利用关系式
\begin{equation}
	\delta[f(x)]=\sum_{i}\frac{\delta(x-x_{i})}{\abs{f'(x_{i})}}=\sum_{i}\frac{\delta(x-x_{i})}{\abs{f'(x)}},
\end{equation}
其中 $x_{i}$是方程 $f(x)=0$的单根。因此
\begin{equation}
	\begin{aligned}[b]
		\delta[s-(\omega_{1}+\omega_{2})^2]=&\frac{\delta(q-q_{\text{cm}})}{\abs{\dv{[s-(\omega_{1}+\omega_{2})^2]}{q}}}+\frac{\delta(q+q_{\text{cm}})}{\abs{\dv{[s-(\omega_{1}+\omega_{2})^2]}{q}}}\\[1ex]
	=&\abs{\frac{\omega_{1}\omega_{2}}{2q(\omega_{1}+\omega_{2})^2}}\big[\delta(q-q_{\text{cm}})+\delta(q+q_{\text{cm}})\big],
	\end{aligned}
\end{equation}
把上式带入\eqref{Gri}式的$\Im(G)$中,得到
\begin{equation}
	\begin{aligned}[b]
		\Im(G)=&-\frac{1}{4\pi^2}\int^{q_{\text{max}}}_{0}\dd{\abs{\vec{q\,}}}\vec{q\,}^2\frac{\omega_{1}+\omega_{2}}{\omega_{1}\omega_{2}}\pi\delta[s-(\omega_{1}+\omega_{2})^2]\\[1ex]
	=&-\frac{1}{4\pi^2}\int^{q_{\text{max}}}_{0}\dd{\abs{\vec{q\,}}}\vec{q\,}^2\frac{\omega_{1}+\omega_{2}}{\omega_{1}\omega_{2}}\pi\abs{\frac{\omega_{1}\omega_{2}}{2q(\omega_{1}+\omega_{2})^2}}\big[\delta(q-q_{\text{cm}})+\delta(q+q_{\text{cm}})\big]\\[1ex]
	=&-\frac{1}{4\pi^2}\int^{q_{\text{max}}}_{0}\dd{\abs{\vec{q\,}}}\vec{q\,}^2\frac{\omega_{1}+\omega_{2}}{\omega_{1}\omega_{2}}\pi\frac{\omega_{1}\omega_{2}}{2q(\omega_{1}+\omega_{2})^2}\delta(q-q_{\text{cm}})\\[1ex]
	=&-\frac{1}{8\pi}\eval{\frac{q}{\omega_{1}+\omega_{2}}}_{q=q_{\text{cm}}}\\[1ex]
	=&-\frac{q_{cm}}{8\pi\sqrt{s}}.
	\end{aligned}
\end{equation}
当$\sqrt{s}>m_{1}+m_{2}$的时候,要把传播子解析延拓到第\uppercase\expandafter{\romannumeral2}黎曼面上,
\begin{equation}
	\begin{aligned}[b]
	G^{b}(E+i\epsilon)=&G^{a}(E-i\epsilon)\\
	=&G^{a*}(E+i\epsilon)\\
	=&G^{a}(E+i\epsilon)-2i\Im[G^{a}(E+i\epsilon)].
	\end{aligned}
\end{equation}
上式中使用了连续性条件和Schwarz反射定理 $f(z^{*})=f^{*}(z)$。然后第\uppercase\expandafter{\romannumeral2}黎曼面的传播子为
\begin{equation}
	\begin{aligned}
		G^{\uppercase\expandafter{\romannumeral2}}(s)=&\frac{1}{16\pi^2s}\Biggl\{\sigma\left(\arctan\frac{s+\Delta}{\sigma\lambda_{1}}+\arctan\frac{s-\Delta}{\sigma\lambda_{2}}\right)&\\[1ex]
		&\quad-\left[(s+\Delta)\ln\frac{(1+\lambda_{1})q_{\text{max}}}{m_{1}}+(s-\Delta)\ln\frac{(1+\lambda_{2})q_{\text{max}}}{m_{2}}\right]\Biggr\}+\frac{i}{4\pi}\frac{q_{\text{cm}}}{\sqrt{s}}.
	\end{aligned}
\end{equation}
\par
对于介子-重子散射过程,介子-重子传播子构成的传播子圈函数为
\begin{equation}
	G(s)=i2M\int\frac{\dd[4]{q}}{(2\pi)^4}\frac{1}{(q^2-m_{1}^2+i\epsilon)[(P-q)^2-M_{2}^2+i\epsilon]},
\end{equation}
类似于前述的介子-介子散射过程,写出三动量截断的介子-重子传播子圈函数为
\begin{equation}
	\label{eqcG}
\begin{split}
	G(s)=&\frac{2M}{16\pi^2s}\Biggl\{\sigma\left(\arctan\frac{s+\Delta}{\sigma\lambda_{1}}+\arctan\frac{s-\Delta}{\sigma\lambda_{2}}\right)\\[1ex]
	     &\quad-\left[(s+\Delta)\ln\frac{(1+\lambda_{1})q_{\text{max}}}{M}+(s-\Delta)\ln\frac{(1+\lambda_{2})q_{\text{max}}}{m}\right]\Biggr\},
\end{split}
\end{equation}
其中$M$是重子的质量,$m$是介子的质量,$\sigma$、$\Delta$、$\lambda$分别为
\begin{equation}
\begin{split}
	\sigma=&\sqrt{-[s-(M+m)^2][s-(M-m)^2]}\ ,\\[1ex]
	\Delta=&M^2-m^2,\\[1ex]
	\lambda_{1}=&\sqrt{1+\frac{M^2}{q_{\text{max}}}}\ ,\\[1ex]
	\lambda_{2}=&\sqrt{1+\frac{m^2}{q_{\text{max}}}}\ .
\end{split}
\end{equation}
第\uppercase\expandafter{\romannumeral2}黎曼面的传播子圈函数为
\begin{equation}
	\label{eqcG2}
	\begin{aligned}
		G^{\uppercase\expandafter{\romannumeral2}}(s)=&\frac{2M}{16\pi^2s}\Biggl\{\sigma\left(\arctan\frac{s+\Delta}{\sigma\lambda_{1}}+\arctan\frac{s-\Delta}{\sigma\lambda_{2}}\right)\\[1ex]
							      &\quad-\left[(s+\Delta)\ln\frac{(1+\lambda_{1})q_{\text{max}}}{M}+(s-\Delta)\ln\frac{(1+\lambda_{2})q_{\text{max}}}{m}\right]\Biggr\}+\frac{i}{4\pi}\frac{2Mq_{\text{cm}}}{\sqrt{s}}.
	\end{aligned}
\end{equation}\par
此外,还可以采用维数正规化方法对传播子圈函数进行重整化,得到的介子-重子圈函数为\cite{JIDO2003181}
\begin{equation}
	\label{eqdG}
\begin{split}
	G_{l}(s)=&\frac{2M_{l}}{16\pi^2}\Big\{a_{l}(\mu)+\ln\frac{M_{l}^2}{\mu^2}+\frac{m_{l}^2-M_{l}^2+s}{2s}\ln\frac{m_{l}^2}{M_{l}^2}\\[1ex]
		 &+\frac{q_{l}}{\sqrt{s}}\big[\ln(s-(M_{l}^2-m_{l}^2)+2q_{l}\sqrt{s})+\ln(s+(M_{l}^2-m_{l}^2)+2q_{l}\sqrt{s})\\[1ex]
		 &-\ln(-s+(M_{l}^2-m_{l}^2)+2q_{l}\sqrt{s})-\ln(-s-(M_{l}^2-m_{l}^2)+2q_{l}\sqrt{s})\big]\Big\},
\end{split}
\end{equation}
其中$\mu$是重整化能标,$a_{l}(\mu)$为减除常数。第\uppercase\expandafter{\romannumeral2}黎曼面的介子-重子传播子$G$函数为
\begin{equation}
	\label{eqdG2}
	G_{l}^{\uppercase\expandafter{\romannumeral2}}(s)=G_{l}(s)+\frac{i}{4\pi}\frac{2M_{l}q_{\text{cm}}}{\sqrt{s}}.
\end{equation}
\section{$\Xi(1620)$和$\Xi(1690)$的动力学产生}
最近Belle合作组对奇异-粲重子$\Xi_{c}^{+}$的衰变过程$\Xi_{c}^{+}\to\Xi^{-}\pi^{+}\pi^{+}$进行了观测\cite{PhysRevLett.122.072501},将末态的两个$\pi^+$介子区分为:动量较低的$\pi^+_L$和动量较高的$\pi^+_H$;测量了衰变末态$\Xi^-\pi^+_L$和$\Xi^-\pi^+_H$的不变质量分布,在$\Xi^-\pi^+_L$不变质量谱中观测到了较明显的$\Xi(1530)(\frac{3}{2}^+)$、$\Xi(1620)(\frac{1}{2}^-)$和$\Xi(1690)(\frac{1}{2}^{-})$共振态的产生信号,见文献\cite{PhysRevLett.122.072501}的图Fig.\,2(a)。\par
另一方面,文献\cite{PhysRevLett.89.252001}在手征幺正法理论框架下研究了$\pi\Xi$(及其耦合道,包括$\bar{K}\Lambda$、$\bar{K}\Sigma$和$\eta\Xi$)相互作用,采用维数正规化方法处理介子-重子传播子圈函数的非物理发散问题,通过求解介子-重子散射的BS方程,动力学重现了$\Xi(1620)$共振态,计算结果显示,$\pi\Xi $是与$\Xi(1620)$耦合最强的反应道。文献\cite{PhysRevLett.89.252001}的研究结果表明,超子激发态$\Xi(1620)$的自旋-宇称为$J^P={\frac{1}{2}}^-$,它被解释为介子-重子分子态,其主要成分为$\pi\Xi$。然而,在文献\cite{PhysRevLett.89.252001}的研究中,另一个$J^P={\frac{1}{2}}^-$的超子激发态$\Xi(1690)$并未从$\pi\Xi$及其耦合道的相互作用过程中动力学地产生出来。\par
本文将在手征幺正法的理论框架下研究$\Xi_{c}^{+}\to\Xi^{-}\pi^{+}\pi^{+}$衰变过程,考虑末态$\Xi^-\pi^+$重散射机制的贡献,计算出$\Xi^-\pi^+$不变质量分布并与Belle的实验结果对比,进而检验$\Xi(1620)(\frac{1}{2}^{-})$和$\Xi(1690)(\frac{1}{2}^{-})$共振态的分子态结构特性。\par
由于我们的研究工作中将考虑$\pi\Xi$的重散射过程,需要用到$\pi\Xi$重散射的全散射振幅,为此需要重复文献\cite{PhysRevLett.89.252001}中关于$\pi\Xi$散射及作为散射中间共振态而动力学产生$\Xi $超子激发态的计算。我们对文献\cite{PhysRevLett.89.252001}工作的重复包括两方面:1)采用与文献\cite{PhysRevLett.89.252001}完全相同的理论公式和参数,重现文献\cite{PhysRevLett.89.252001}的计算结果;2)采用另一种处理介子-重子传播子圈函数发散的重整化方法——三动量截断法来计算,通过调整理论计算中唯一的自由参数——三动量截断$q_{\text{max}}$的取值来考察是否能同时动力学产生$\Xi(1620)(\frac{1}{2})^-$和$\Xi(1690)(\frac{1}{2})^-$共振态,并计算动力学产生态与各反应道的耦合常数来给出动力学产生态的主要成分。
%文献\cite{PhysRevLett.89.252001}研究了$\Xi(1620)$的动力学产生,在该文献中认为$\Xi(1620)$的同位旋为$I=\frac{1}{2}$,由$\pi\Xi$,$\bar{K}\Lambda$,$\bar{K}\Sigma$,$\eta\Xi$相互作用动力产生的介子重子分子态。文献中使用了维数正规化的方法,并且给出了$\Xi(1620)$的极点和耦合常数,但是在该文献中并没有$\Xi(1690)$。在本文中认为$\Xi(1690)$有着同样的动力学产生机制,使用三动量截断的方法来重新计算$\Xi(1620)$和$\Xi(1690)$的极点和耦合常数。
\subsection{与$\pi^{+}\Xi^{-}$相耦合的电荷反应道及同位旋反应道}
$\pi^{+}\Xi^{-}$$s$-波相互作用体系的量子数如下。由于$\pi^{+}$和$\Xi^{-}$的$J^{P}$量子数分别为$0^{-}$和$\frac{1}{2}^{+}$,$s$-波$\pi^{+}\Xi^{-}$体系的$J^{P}$量子数为$\frac{1}{2}^{-}$。该体系的奇异数为$S=-1$。对同位旋量子数$I$而言,由于$\pi^{+}$和$\Xi^{-}$的同位旋分别为$1$和$\frac{1}{2}$,由同位旋耦合规则可知,$\pi^{+}\Xi^{-}$体系的总同位旋有两种可能的取值$I=\frac{1}{2},\frac{3}{2}$,其分量$I_{z}=0$。$s$-波$\pi^{+}\Xi^{-}$体系的量子数可总结如下,
\begin{equation*}
	\pi^{+}\Xi^{-}\text{:电荷}Q=0;\ J^{P}=\frac{1}{2}^{-};\ \text{奇异数}S=-1;\ \text{同位旋}I=\frac{1}{2},\ \frac{3}{2};\ I_{z}=0.
\end{equation*}
具有如上量子数的赝标介子-重子反应道(也称为物理反应道或电荷反应道)为
\begin{equation}
	\label{eqeb}
	\pi^{+}\Xi^{-},\quad\pi^{0}\Xi^{0},\quad \bar{K}^{0}\Lambda,\quad \bar{K}^{0}\Sigma^{0},\quad K^{-}\Sigma^{-},\quad \eta\Xi^{0}.
\end{equation}
由于\eqref{eqeb}式中的电荷反应道的同位旋有两种可能取值:$I=\frac{1}{2},\frac{3}{2}$,因而这些电荷反应道的$s$-波散射既可以研究$J^{P}=\frac{1}{2}^{-},I=\frac{1}{2}$的$\Xi$超子激发态,也可以研究$J^{P}=\frac{1}{2}^{-},I=\frac{3}{2}$的超子激发态。由于本文中我们只关于$\Xi$超子激发态的性质和结构,因此我们只考虑$I=\frac{1}{2}$的赝标介子-重子同位旋反应道,
\begin{equation}
	\label{eqib}
	I=\frac{1}{2}:\quad \pi\Xi,\quad \bar{K}\Lambda,\quad \bar{K}\Sigma,\quad \eta\Xi\,.
\end{equation}\par
以同位旋态$\ket{I,I_{z}}$来表示物理粒子,并采用与文献\cite{OSET199899}相同的同位旋相位约定,即
\begin{equation}
\begin{split}
	\ket{\pi^{+}}=&-\ket{I=1,I_{z}=1},\quad
	\ket{K^{-}}=-\ket{I=\frac{1}{2},I_{z}=-\frac{1}{2}},\\[1ex]
	\ket{\Sigma^{+}}=&-\ket{I=1,I_{z}=1},\quad
	\ket{\Xi^{-}}=-\ket{I=\frac{1}{2},I_{z}=-\frac{1}{2}},
\end{split}
\end{equation}
\vspace{-0.5cm}
可以写出相关物理粒子的同位旋态表示,
\begin{equation*}
	\ket{\pi^{0}}=\ket{1,0},\quad \ket{\bar{K}^{0}}=\ket{\frac{1}{2},\frac{1}{2}},\quad \ket{\Lambda}=\ket{0,0},\quad \ket{\Xi^{0}}=\ket{\frac{1}{2},\frac{1}{2}},\quad \ket{\Sigma^{0}}=\ket{1,0}.
\end{equation*}\par
\vspace{-0.5cm}
利用同位旋耦合的C.G.系数,可以得到\eqref{eqeb}式中各物理反应道的同位旋态表示为
\begin{equation}
%\label{ib1}
\begin{aligned}[b]
	\label{eqib1}
	\ket{\pi^{+}\Xi^{-}}=&\ket{-(I_{1}=1,I_{1z}=1);-(I_{2}=\frac{1}{2},I_{2z}=-\frac{1}{2})}\\[2ex]
	=&\sqrt{\frac{1}{3}}\ket{I=\frac{3}{2},I_{z}=\frac{1}{2}}+\sqrt{\frac{2}{3}}\ket{I=\frac{1}{2},I_{z}=\frac{1}{2}},
\end{aligned}
\end{equation}
\begin{equation}
\begin{aligned}[b]
	\label{eqib2}
	\ket{\pi^{0}\Xi^{0}}=&\ket{I_{1}=1,I_{1z}=0;I_{2}=\frac{1}{2},I_{2z}=\frac{1}{2}}\\[2ex]
	=&\frac{2}{3}\ket{I=\frac{3}{2},I_{z}=\frac{1}{2}}-\frac{1}{3}\ket{I=\frac{1}{2},I_{z}=\frac{1}{2}},
\end{aligned}
\end{equation}
\begin{equation}
\begin{aligned}[b]
	\label{eqib3}
	\ket{K^{-}\Sigma^{+}}=&\ket{-(I_{1}=\frac{1}{2},I_{1z}=-\frac{1}{2});-(I_{2}=1,I_{2z}=1)}\\[2ex]
	=&(-1)^{\frac{3}{2}-\frac{1}{2}-1}\sqrt{\frac{1}{3}}\ket{I=\frac{3}{2},I_{z}=\frac{1}{2}}\\[2ex]
	 &+(-1)^{\frac{3}{2}-\frac{1}{2}-1}\sqrt{\frac{2}{3}}\ket{I=\frac{1}{2},I_{z}=\frac{1}{2}}\\[2ex]
	=&\sqrt{\frac{1}{3}}\ket{I=\frac{3}{2},I_{z}=\frac{1}{2}}-\sqrt{\frac{2}{3}}\ket{I=\frac{1}{2},I_{z}=\frac{1}{2}},
\end{aligned}
\end{equation}\newpage
\begin{equation}
\begin{aligned}[b]
	\label{eqib4}
	\ket{\bar{K}^{0}\Sigma^{0}}=&\ket{I_{1}=\frac{1}{2},I_{1z}=\frac{1}{2};I_{2}=1,I_{2z}=0}\\[2ex]
	=&(-1)^{\frac{3}{2}-\frac{1}{2}-1}\sqrt{\frac{2}{3}}\ket{I=\frac{3}{2},I_{z}=\frac{1}{2}}\\[2ex]
	 &+(-1)^{\frac{1}{2}-\frac{1}{2}-1}(-\sqrt{\frac{1}{3}})\ket{I=\frac{1}{2},I_{z}=\frac{1}{2}}\\[2ex]
	=&\sqrt{\frac{2}{3}}\ket{I=\frac{3}{2},I_{z}=\frac{1}{2}}+\sqrt{\frac{1}{3}}\ket{I=\frac{1}{2},I_{z}=\frac{1}{2}},
\end{aligned}
\end{equation}
\begin{equation}
\begin{aligned}[b]
	\label{eqib5}
	\ket{\bar{K}^{0}\Lambda}=&\ket{I_{1}=\frac{1}{2},I_{1z}=\frac{1}{2};I_{2}=0,I_{2z}=0}\\[2ex]
	=&\ket{I=\frac{1}{2},I_{z}=\frac{1}{2}},
\end{aligned}
\end{equation}
\begin{equation}
\begin{aligned}[b]
	\label{eqib6}
	\ket{\eta\Xi^{0}}=&\ket{I_{1}=0,I_{1z}=0;I_{2}=\frac{1}{2},I_{2z}=\frac{1}{2}}\\[2ex]
	=&\ket{I=\frac{1}{2},I_{z}=\frac{1}{2}},
\end{aligned}
\end{equation}
其中$\ket{K^{-}\Sigma^{+}}$和$\ket{\bar{K}^{0}\Sigma^{0}}$的表示中用到了关系式
\begin{equation}
\braket{j_{1}j_{2}m_{1}m_{2}}{j_{1}j_{2}JM}=(-1)^{J-j_{1}-_{2}}\braket{j_{2}j_{1}m_{2}m_{1}}{j_{2}j_{1}JM}.
\end{equation}\par
由式\eqref{eqib1}--\eqref{eqib6}可以得到,式\eqref{eqeb}中的同位旋反应道可用电荷反应道表示如下,
%\begin{equation}
%%\label{ib2}
%\begin{aligned}[b]
%	\label{eq2ib1}
%	\ket{\pi\Xi,I=\frac{1}{2}}=&\sqrt{\frac{2}{3}}\ket{1,1;\frac{1}{2},-\frac{1}{2}}-\sqrt{\frac{1}{3}}\ket{1,0;\frac{1}{2},\frac{1}{2}}\\[1ex]
%	=&\sqrt{\frac{2}{3}}\ket{\pi^{+}\Xi^{-}}-\sqrt{\frac{1}{3}}\ket{\pi^{0}\Xi^{0}},
%\end{aligned}
%\end{equation}
%\begin{equation}
%\begin{aligned}[b]
%	\ket{\bar{K}\Sigma,I=\frac{1}{2}}=&(-1)^{\frac{1}{2}-\frac{1}{2}-1}\sqrt{\frac{2}{3}}\ket{\frac{1}{2},-\frac{1}{2};1,1}\\[1ex]
%							    &+(-1)^{\frac{1}{2}-\frac{1}{2}-1}\left(-\sqrt{\frac{1}{3}}\right)\ket{\frac{1}{2},\frac{1}{2}1,0}\\[1ex]
%	=&\sqrt{\frac{1}{3}}\ket{\bar{K}^{0}\Sigma^{0}}-\sqrt{\frac{2}{3}}\ket{K^{-}\Sigma^{+}},
%\end{aligned}
%\end{equation}
%\begin{equation}
%	\ket{\bar{K}\Lambda,I=\frac{1}{2}}=\ket{\bar{K}^{0}\Lambda},
%\end{equation}
%\begin{equation}
%	\ket{\eta\Xi,I=\frac{1}{2}}=\ket{\eta,\Xi^{0}}.
%\end{equation}
\begin{align}
	\label{eq2ib1}
	\ket{\pi\Xi,I=\frac{1}{2}}=&\sqrt{\frac{2}{3}}\ket{\pi^{+}\Xi^{-}}-\sqrt{\frac{1}{3}}\ket{\pi^{0}\Xi^{0}},\\[2ex]
	\ket{\bar{K}\Sigma,I=\frac{1}{2}}=&\sqrt{\frac{1}{3}}\ket{\bar{K}^{0}\Sigma^{0}}-\sqrt{\frac{2}{3}}\ket{K^{-}\Sigma^{+}},\\[2ex]
	\ket{\bar{K}\Lambda,I=\frac{1}{2}}=&\ket{\bar{K}^{0}\Lambda},\\[2ex]
	\ket{\eta\Xi,I=\frac{1}{2}}=&\ket{\eta,\Xi^{0}}.
\end{align}
\newpage
\subsection{BS方程的核$V_{ij}$}
$V_{ij}$可取为第$i$个反应道跃迁到第$j$个反应道的跃迁过程$i\to j$的最低阶跃迁势,
\begin{equation}
\label{def v}
V_{ij}=\mel{j}{(-\mathcal{L}_{2})}{i},
\end{equation}
其中$\mathcal{L}_{2}$为描写赝标介子与$J^{P}=\frac{1}{2}^{+}$重子基态间相互作用的最低阶手征拉氏量。由文献\cite{OSET199899,Pich_1995,ECKER19951,BERNARD_1995}可以知道,赝标介子九重态和$J^{P}=\frac{1}{2}^{+}$基态重子八重态相互作用的最低阶手征拉氏量为
\begin{equation}
	\mathcal{L}_{2}=\langle\bar{B}i\gamma^{\mu}\grad_{\mu}B\rangle-M_{B}\langle\bar{B}B\rangle+\frac{1}{2}D\langle\bar{B}\gamma^{\mu}\gamma_{5}\{u_{\mu},B\}\rangle+\frac{1}{2}F\langle\bar{B}\gamma^{\mu}\gamma_{5}[u_{\mu},B]\rangle,
\end{equation}
\vspace{-0.5cm}
其中
\vspace{-0.5cm}
\begin{equation}
\begin{split}
	&\grad_{\mu}B=\partial_{\mu}B+[\Gamma_{\mu},B],\\[1ex]
	&\Gamma_{\mu}=\frac{1}{2}(u^{\dagger}\partial_{\mu}u+u\partial_{\mu}u^{\dagger}),\\[1ex]
	&U=u^2=\exp(i\sqrt{2}\phi/f),\\[1ex]
	&u_{\mu}=iu^{\dagger}\partial_{\mu}Uu^{\dagger},
\end{split}
\end{equation}
$\phi$是赝标介子八重态矩阵,具体形式为式\eqref{meson8},$B$为$J^{P}=\frac{1}{2}^{+}$的基态重子八重态矩阵,表达式为式\eqref{baryon8}。相互作用拉氏量来自$\Gamma_{\mu}$的项,所以拉氏量为
\begin{equation}
	\mathcal{L}_{2}=\langle\bar{B}i\gamma^{\mu}[\Gamma_{\mu},B]\rangle.
\end{equation}
把拉氏量处理一下,其中$u$和$u^{\dagger}$为
%\begin{equation}
%\begin{aligned}
%	u=&\exp(\frac{i\sqrt{2}}{2f}\phi)&u^{\dagger}=&\exp(\frac{-i\sqrt{2}}{2f}\phi)\\
%	=&1+\frac{i\sqrt{2}}{2f}\phi-\frac{1}{4f^2}\phi\phi+\cdots,&=&1-\frac{i\sqrt{2}}{2f}\phi-\frac{1}{4f^2}\phi\phi+\cdots,
%\end{aligned}
%\end{equation}
\begin{align}
	u=&\exp(\frac{i\sqrt{2}}{2f}\phi)=1+\frac{i\sqrt{2}}{2f}\phi-\frac{1}{4f^2}\phi\phi+\cdots,\\[1ex]
	u^{\dagger}=&\exp(\frac{-i\sqrt{2}}{2f}\phi)=1-\frac{i\sqrt{2}}{2f}\phi-\frac{1}{4f^2}\phi\phi+\cdots,
\end{align}
其中$\phi^{\dagger}=\phi$,只保留到$\phi$的平方项,则$\partial_{\mu}u$和$\partial_{\mu}u^{\dagger}$为
\begin{equation}
\begin{split}
	\partial_{\mu}u=&\frac{i\sqrt{2}}{2f}\partial_{\mu}\phi-\frac{1}{4f^2}\partial_{\mu}\phi\phi-\frac{1}{4f^2}\phi\partial_{\mu}\phi,\\[1ex]
	\partial_{\mu}u^{\dagger}=&\frac{-i\sqrt{2}}{2f}\partial_{\mu}\phi-\frac{1}{4f^2}\partial_{\mu}\phi\phi-\frac{1}{4f^2}\phi\partial_{\mu}\phi.
\end{split}
\end{equation}
现在可以把$\Gamma_{\mu}$表示出来,同样保留到$\phi$的平方项,\vspace{-0.5cm}
\begin{equation}
\begin{aligned}[b]
	\Gamma_{\mu}=&\frac{1}{2}(u^{\dagger}\partial_{\mu}u+u\partial_{\mu}u^{\dagger})\\
	=&\frac{1}{2}[(1-\frac{i\sqrt{2}}{2f}\phi)(\frac{i\sqrt{2}}{2f}\partial_{\mu}\phi-\frac{1}{4f^2}\partial_{\mu}\phi\phi-\frac{1}{4f^2}\phi\partial_{\mu}\phi)\\[1ex]
	 &+(1+\frac{i\sqrt{2}}{2f}\phi)(\frac{-i\sqrt{2}}{2f}\partial_{\mu}\phi-\frac{1}{4f^2}\partial_{\mu}\phi\phi-\frac{1}{4f^2}\phi\partial_{\mu}\phi)]\\[1ex]
	=&\frac{1}{2}(-\frac{1}{4f^2}\partial_{\mu}\phi\phi-\frac{1}{4f^2}\phi\partial_{\mu}\phi+\frac{1}{2f^2}\phi\partial_{\mu}\phi-\frac{1}{4f^2}\partial_{\mu}\phi\phi-\frac{1}{4f^2}\phi\partial_{\mu}\phi+\frac{1}{2f^2}\phi\partial_{\mu}\phi)\\[1ex]
	=&\frac{1}{4f^2}(\phi\partial_{\mu}\phi-\partial_{\mu}\phi\phi)\\
	=&\frac{1}{4f^2}[\phi,\partial_{\mu}\phi].
\end{aligned}
\end{equation}
最终的拉氏量可以简化为
\begin{equation}
	\label{eql2}
	\mathcal{L}_{2}=\frac{1}{4f^2}\langle\bar{B}i\gamma^{\mu}[[\phi,\partial_{\mu}\phi],B]\rangle.
\end{equation}\par
从拉氏量中可以看出$V_{ij}$具有类似的结构,
\begin{equation}
\begin{split}
	V_{ij}=&\mel{j}{(\mathcal{-L}_{2})}{i}\\[1ex]
	=&-C_{ij}\frac{1}{4f^2}\bar{u}(p')\gamma^{\mu}u(p)(k_{\mu}+k'_{\mu}),
\end{split}
\end{equation}
其中,$f=\SI{93}{MeV}$为$\pi$介子的衰变常数,$p$和$k$分别是第$i$反应道中重子和介子的四动量,$p'$和$k'$分别时第$j$反应道中重子和介子的四动量。
$V_{ij}$的具体计算过程详见附录\ref{appendices a},因此BS方程的核为
\begin{equation}
	\label{eqvij}
	V_{ij}=-C_{ij}\frac{1}{4f^2}\sqrt{\frac{E_{i}+M_{i}}{2M_{i}}}\sqrt{\frac{E_{j}+M_{j}}{2M_{j}}}(2\sqrt{s}-M_{i}-M_{j}),
\end{equation}
其中$\sqrt{s}$为介子-重子体系的质心系总能量,即反应的有效能量;$m_{i}$和$M_{i}$分别表示第$i$道中介子和重子的质量;$E_{i}$表示第$i$反应道中重子的能量,
\begin{equation}
	E_{i}=\sqrt{M_{i}^2+\vec{p\,}_{i}^2},
\end{equation}
上式中$\vec{p}_{i}$为第$i$反应道中的某个粒子在质心系中三动量,其大小为
\begin{equation}
	\abs{\vec{p}_{i}}=\frac{\lambda^{\frac{1}{2}}(s,M_{i}^2,m_{i}^2)}{2\sqrt{s}}.
\end{equation}
当能量$\sqrt{s}$小于第$i$个反应道的阈值$m_{i}+M_{i}$时,$\lambda(s,M_{i}^2,m_{i}^2)<0$,此时,取$\abs{\vec{p}_{i}}=0$。
式\eqref{eqvij}中的$C_{ij}$系数可由最低阶手征拉氏量式\eqref{eql2}来求出。当取式\eqref{eqeb}所示的电荷反应道时,所得的$C_{ij}$系数列于表\ref{eCij}中。当取式\eqref{eqib}所示的$I=\frac{1}{2}$同位旋反应道时,所得的$C_{ij}$系数列于表\ref{iCij}中。具体的求解过程见附录\ref{appendices a}。
\begin{table}[ht]
\centering
%\caption{Coefficients $C_{ij}$ of the meson baryon amplitudes for isospin $I=\frac{1}{2}(C_{ji}=C_{ij})$}
\caption[电荷基下的介子-重子振幅系数$C_{ij}$]{电荷基下的介子-重子振幅系数$C_{ij}$($C_{ji}=C_{ij}$)。}
\label{eCij}
\begin{spacing}{1.5}
%\scalebox{1.5}{
\setlength{\tabcolsep}{7mm}
\begin{tabular}{c|rrrrrr}
	\toprule
	\hline
	~ & $\pi^{+}\Xi^{-}$ &$\pi^{0}\Xi^{0}$& $\bar{K}^{0}\Lambda$ &$\bar{K}^{0}\Sigma^{0}$& $K^{-}\Sigma^{+}$ & $\eta\Xi^{0}$\\\hline
	$\pi^{+}\Xi^{-}$ & 1 & $-\sqrt{2}$ & $-\frac{3}{\sqrt{6}}$ &$-\frac{1}{\sqrt{2}}$&0&0 \\
	$\pi^{0}\Xi^{0}$ & ~ & 0 & $\frac{\sqrt{3}}{2}$ &$-\frac{1}{2}$& $-\frac{1}{\sqrt{2}}$&0 \\
	$\bar{K}^{0}\Lambda$ & ~ & ~ & 0 &0&0&$-\frac{3}{2}$ \\
	$\bar{K}^{0}\Sigma^{0}$ & ~ & ~ & ~ &0&$-\sqrt{2}$& $\frac{\sqrt{3}}{2}$ \\
	$K^{-}\Sigma^{+}$ & ~ & ~ & ~ &~&1& $-\frac{3}{\sqrt{6}}$ \\
	$\eta\Xi^{0}$ & ~ & ~ & ~ &~&~&0 \\
	\Xhline{1pt}
\end{tabular}
%}
\end{spacing}
\end{table}
\begin{table}[h]
\centering
\caption[同位旋$I=\frac{1}{2}$的介子-重子振幅系数$C_{ij}$]{同位旋$I=\frac{1}{2}$的介子-重子振幅系数$C_{ij}$($C_{ji}=C_{ij}$)。}
\label{iCij}
\begin{spacing}{1.5}
%\scalebox{1.5}{
\setlength{\tabcolsep}{10mm}
\begin{tabular}{c|rrrr}
	\toprule
	\hline
	~ & $\pi\Xi$ & $\bar{K}\Lambda$ & $\bar{K}\Sigma$ & $\eta\Xi$\\\hline
	$\pi\Xi$ & 2 & $-\frac{3}{2}$ & $-\frac{1}{2}$ &0 \\
	$\bar{K}\Lambda$ & ~ & 0 & 0 &$-\frac{3}{2}$ \\
	$\bar{K}\Sigma$ & ~ & ~ & 2 &$\frac{3}{2}$ \\
	$\eta\Xi$ & ~ & ~ & ~ &0 \\
	\Xhline{1pt}
\end{tabular}
%}
\end{spacing}
\end{table}
\subsection{极点和耦合常数}
$\pi\Xi$散射$T$矩阵所满足的耦合道BS方程为
\begin{equation}
	\label{eqbs}
	T=\frac{V}{I-VG}.
\end{equation}
当考虑式\eqref{eqib}所示的$I=\frac{1}{2}$的赝标介子-重子同位旋反应道时,由于共有4个相互耦合的反应道,BS方程\eqref{eqbs}式是一个$4\times4$的矩阵方程,其中$I$是$4\times4$的单位矩阵,$V$矩阵的矩阵元如式\eqref{eqvij}所示,相应的$C_{ij}$系数如表\ref{iCij}所示。对角矩阵$G$的矩阵元为介子-重子传播子圈函数,既可取为维数正规化下的式\eqref{eqdG}和\eqref{eqdG2},也可取三动量截断重整化下的式\eqref{eqcG}和\eqref{eqcG2}。\par
通过求解$\det(I-VG)=0$可以得到$T$矩阵的极点,
\begin{equation}
	s_{pole}=M_{R}+i\frac{\Gamma_{R}}{2}.
\end{equation}
极点的实部是共振态的质量,虚部为共振态的半宽度。\par
%\begin{table}[h]
%\centering
%\caption{不同$q_{max}$下的极点}
%\begin{spacing}{1.5}
%\setlength{\tabcolsep}{5mm}
%\begin{tabular}{|c|c|c|c|}
%	\Xhline{1pt}
%	\hline
%	$q_{max}$&670MeV & 690MeV & 710MeV \\\Xhline{0.8pt}
%	\multirow{2}{*}{电荷基}&1567.55+114.007i& 1565.7+108.584i& 1563.62+103.426i  \\
%	~&1681.34+1.1337i&1679.51+1.45883i&1677.25+1.72387i\\\hline
%	\multirow{2}{*}{同位旋基}& 1566.49+114.363i & 1564.65+108.941i&1562.59+103.782i\\
%	~& 1684.97+1.36415i &1682.92+1.63574i&1680.51+1.87713i\\
%	\Xhline{0.8pt}
%	$q_{max}$&730MeV & 750MeV & 770MeV \\\Xhline{0.8pt}
%	\multirow{2}{*}{电荷基}&1561.38+98.5369i&1559+93.9122i& 1556.53+89.5437i  \\
%	~&1674.64+1.94279i&1671.72+2.12178i&1668.53+2.26399i\\\hline
%	\multirow{2}{*}{同位旋基}& 1560.36+98.8912i & 1557.99+94.2645i& 1555.52+89.8939i\\
%	~& 1677.78+2.08683i & 1674.75+2.26405i& 1671.48+2.40858i\\
%	\Xhline{1pt}
%\end{tabular}
%\end{spacing}
%\end{table}\par
散射振幅$T_{ij}$可以在极点处做洛朗级数展开
\begin{equation}
	T_{ij}=\frac{g_{i}g_{j}}{s-s_{pole}}+a_{0}+a_{1}(s-s_{pole})+\cdots,
\end{equation}
其中$g_{i}$和$g_{j}$分别是动力学产生的共振态与第$i$和第$j$反应道的耦合常数。因此
\begin{equation}
	g_{i}g_{j}=\lim_{s\to s_{pole}}(s-s_{pole})T_{ij}.
\end{equation}
还可以通过留数定理来求解耦合常数,
\begin{equation}
	g_{i}g_{j}=\frac{1}{2\pi i}\oint_{\abs{s-s_{pole}}=\varepsilon} T_{ij}\dd{s},
\end{equation}
其中$\varepsilon$是一个小量,确保圈积分内只有一个极点。\par
%\begin{table}[h]
%\centering
%\caption{$\Xi(1620)$在电荷基下的耦合常数($q_{max}=770\si{MeV}$)}
%\label{1620eb}
%\begin{spacing}{1.2}
%\fontsize{10pt}{\baselineskip}\selectfont
%\setlength{\tabcolsep}{5mm}
%\begin{tabular}{|c|c|c|c|}
%	\Xhline{1pt}
%	\hline
%	电荷基 & $\pi^{+}\Xi^{-}$ &$\pi^{0}\Xi^{0}$& $\bar{K}^{0}\Lambda$\\\Xhline{0.5pt}
%	$g_{i}$&1.70451+1.0023i&-1.21372-0.72100i&-1.8151-0.62984i\\\hline
%	$\abs{g_{i}}$&1.97734&1.41172&1.92127\\\Xhline{0.7pt}
%	电荷基 &$\bar{K}^{0}\Sigma^{0}$& $\bar{K}^{-}\Sigma^{+}$ & $\eta\Xi^{0}$\\\Xhline{0.5pt}
%	$g_{i}$&-0.38819-0.21018i&0.56482+0.28176i&0.27395+0.31477i\\\hline
%	$\abs{g_{i}}$&1.97734&1.41172&1.92127\\
%	\Xhline{1pt}
%\end{tabular}
%\end{spacing}
%\end{table}
%\begin{table}[h]
%\centering
%\caption{$\Xi(1620)$在同位旋基下的耦合常数($q_{max}=770\si{MeV}$)}
%\label{1620ib}
%\begin{spacing}{1.2}
%\fontsize{10pt}{\baselineskip}\selectfont
%\setlength{\tabcolsep}{2mm}
%\begin{tabular}{|c|c|c|c|c|}
%	\Xhline{1pt}
%	\hline
%	同位旋基 & $\pi\Xi$&$\bar{K}\Lambda$&$\bar{K}\Sigma$&$\eta\Xi$\\\hline
%	$g_{i}$&2.09263+1.23698i&-1.81332-0.63143i&-0.67932-0.34245i&0.27356+0.32159i\\\hline
%	$\abs{g_{i}}$&2.43089&1.92011&0.76075&0.42221\\
%	\Xhline{1pt}
%\end{tabular}
%\end{spacing}
%\end{table}
%$\Xi(1690)$的耦合常数如表\ref{1690eb}和表\ref{1690ib}所示。
%\begin{table}[h]
%\centering
%\caption{$\Xi(1690)$在电荷基下的耦合常数($q_{max}=770\si{MeV}$)}
%\label{1690eb}
%\begin{spacing}{1.2}
%\fontsize{10pt}{\baselineskip}\selectfont
%\setlength{\tabcolsep}{5mm}
%\begin{tabular}{|c|c|c|c|}
%	\Xhline{1pt}
%	电荷基 & $\pi^{+}\Xi^{-}$ &$\pi^{0}\Xi^{0}$& $\bar{K}^{0}\Lambda$\\\Xhline{0.5pt}
%	$g_{i}$&0.11211-0.05167i&-0.09591+0.03279i&0.39895+0.10587i\\\hline
%	$\abs{g_{i}}$&0.12345&0.10136&0.41277\\
%	\Xhline{0.7pt}
%	电荷基 &$\bar{K}^{0}\Sigma^{0}$& $\bar{K}^{-}\Sigma^{+}$ & $\eta\Xi^{0}$\\\Xhline{0.5pt}
%	$g_{i}$&-1.28878+0.02306i&1.80487-0.03966i&-1.44526+0.05789i\\\hline
%	$\abs{g_{i}}$&1.28899&1.80531&1.44642\\\Xhline{1pt}
%\end{tabular}
%\end{spacing}
%\end{table}
%\begin{table}[t]
%\centering
%\caption{$\Xi(1690)$在同位旋基下的耦合常数($q_{max}=770\si{MeV}$)}
%\label{1690ib}
%\begin{spacing}{1.2}
%\fontsize{10pt}{\baselineskip}\selectfont
%\setlength{\tabcolsep}{2mm}
%\begin{tabular}{|c|c|c|c|c|}
%	\Xhline{1pt}
%	\hline
%	同位旋基 & $\pi\Xi$&$\bar{K}\Lambda$&$\bar{K}\Sigma$&$\eta\Xi$\\\hline
%	$g_{i}$&0.15403-0.05574i&0.40335+0.10674i&-2.24129+0.04456i&-1.46171+0.06328i\\\hline
%	$\abs{g_{i}}$&0.16381&0.41723&2.24173&1.46308\\
%	\Xhline{1pt}
%\end{tabular}
%\end{spacing}
%\end{table}\par
%通过耦合常数可以看到,$\Xi(1620)$对$\pi\Xi$和$\bar{K}\Lambda$的耦合强度较强$\bar{K}\Sigma$和$\eta\Xi$较弱,$\Xi(1690)$与之相反。
下面分别采用维数正规化的$G$函数和三动量截断的$G$函数来求解$T$矩阵的极点和耦合常数$g_{i}$。
\subsubsection{采用维数正规化的$G$函数}
现在我们采用式\eqref{eqdG}和\eqref{eqdG2}所示的维数正规化的$G$函数,重现文献\cite{PhysRevLett.89.252001}的计算结果。\par
我们在计算中采用与文献\cite{PhysRevLett.89.252001}中完全相同的正规化能标$\mu=\SI{630}{MeV}$以及各反应道的减除常数$a_{l}$,表\ref{tabal}中列出了文献\cite{PhysRevLett.89.252001}所采用的5套不同的$a_{l}$参数。
\begin{table}[b]
\centering
\caption[5套不同的$a_{l}$参数]{文献\cite{PhysRevLett.89.252001}所采用的5套不同的$a_{l}$参数}
\label{tabal}
\begin{spacing}{1.5}
%\scalebox{1.5}{
\setlength{\tabcolsep}{8mm}
\begin{tabular}{cccccc}
	\toprule
	\hline
	~ & Set 1& Set 2& Set 3 & Set 4& Set 5\\\midrule
	$a_{\pi\Xi}$ & $-2.0$ & $-2.2$& $-2.0$&$-2.5$&$-3.1$ \\
	$a_{\bar{K}\Lambda}$ & $-2.0$& $-2.0$ & $-2.2$ &$-1.6$&$-1.0$ \\
	$a_{\bar{K}\Sigma}$ & $-2.0$& $-2.0$ & $-2.0$ & $-2.0$ &$-2.0$ \\
	$a_{\eta\Xi}$ & $-2.0$& $-2.0$& $-2.0$& $-2.0$& $-2.0$ \\
	\bottomrule
\end{tabular}
%}
\end{spacing}
\end{table}
利用表\ref{tabal}中的参数Set1、Set4和Set5分别计算$\pi\Xi\to\pi\Xi$跃迁过程的振幅$T_{11}$的模方$\abs{T_{11}}^2$随能量$\sqrt{s}=M_{\pi\Xi}$变化的分布,如图\ref{T11}所示,这一结果很好地重现了文献\cite{PhysRevLett.89.252001}中Fig.\,1的结果。
\begin{figure}[t]
	\centering
	\includegraphics[width=.8\linewidth]{T11}
	\caption[$\pi\Xi\to\pi\Xi$跃迁振幅的模方]{$\pi\Xi\to\pi\Xi$跃迁振幅的模方$\abs{T_{11}}^2$随$\sqrt{s}=M_{\pi\Xi}$的分布图(完好重现了文献\cite{PhysRevLett.89.252001}中的Fig.\,1)。}
	\label{T11}
\end{figure}
\begin{table}[t]
\begin{spacing}{1.2}
\centering
\caption[$\pi \Xi$ 散射振幅的极点及其耦合常数]{\vadjust{\vspace{0.2pt}}不同参数计算得到的极点及其耦合常数。 (很好重现了文献\cite{PhysRevLett.89.252001}中的Tab. II。)}\label{tab:a2}
\begin{tabular*}{1.05\textwidth}{@{\extracolsep{\fill}}lccccc}
\toprule
\hline
                                     & Set 1           & Set 2            & Set 3           & Set 4          & Set 5 \\
\midrule
    $M\!+\!i \frac{\Gamma}{2}$(MeV)  & $1607.6+140.2i$ & $1597.3+117.3i$  & $1596.2+133.8i$ & $1604.0+98.2i$ & $1605.7+66.1i$  \\
    $g_{\pi \Xi}$                    & $-2.3-1.6i$   & $-2.3-1.4i$    & $-2.2-1.6i$   & $-2.3-1.1i$  & $-2.2-0.5i$      \\
    $g_{\bar K \Lambda}$             & $2.2+0.5i$    & $2.1+0.5i$     & $2.0+0.5i$    &$2.3+0.3i$    & $2.5-0.1i$      \\
     $g_{\bar K \Sigma}$             & $0.6+0.5i$    & $0.7+0.4i$     & $0.6+0.4i$    &$0.8+0.4i$    & $0.9+0.2i$      \\
     $g_{\eta \Xi}$                  & $-0.1-0.6i$   & $-0.2-0.5i$    & $-0.3-0.6i$   &$-0.01-0.4i$   & $0.4-0.3i$      \\
     $|g_{\pi \Xi}|^2$               & $7.9$          & $7.2$           & $7.5$          & $6.8$         & $5.3 $      \\
   $|g_{\bar K \Lambda}|^2$          & $5.0$          & $4.6$           & $4.3$          & $5.5$         & $6.3 $      \\
     $|g_{\bar K \Sigma}|^2$         & $0.6$          & $0.6$           & $0.6$          & $0.7$         & $0.8 $      \\
      $|g_{\eta \Xi}|^2$             & $0.3$          & $0.3$           & $0.4$          & $0.1$         & $0.2$      \\
  \hline
  \bottomrule
\end{tabular*}
\end{spacing}
\end{table}\par
由图\ref{T11}可见,$\abs{T_{11}}^2$在$M_{\pi\Xi}$为$\SI{1610}{MeV}$附近有一个明显的峰,而在$\SI{1690}{MeV}$附近则表现为一个很小的尖峰结构。\par
利用表\ref{tabal}的5组参数求解BS方程,所得到的极点$M+i\frac{\Gamma}{2}$及其与各反应道的耦合常数列于表\ref{tab:a2}中,这些结果很好地重现了文献\cite{PhysRevLett.89.252001}中的Tab.\,\uppercase\expandafter{\romannumeral2}。
由表\ref{tab:a2}可以看到,不同参数计算所得到的$T$振幅都只有一个极点,对应于一个宽的共振态。参数Set5所得到的极点所对应的质量和宽度更接近$\Xi(1620)$的质量和宽度。该动力学产生态的主要耦合道是$\pi\Xi$和$\bar{K}\Lambda$,与$\bar{K}\Sigma$和$\eta\Xi$的耦合较弱。因此文献\cite{PhysRevLett.89.252001}认为,$\Xi(1620)$可由$\pi\Xi$(及其耦合道)相互作用动力学产生,是一个成分主要为$\pi\Xi$和$\bar{K}\Lambda$的介子-重子分子态。\par
下面我们采用三动量截断重整化的$G$函数来计算$T$矩阵的极点和耦合常数。
\begin{table}[h]
\caption[$q_{\text{max}}$取不同值得到的极点]{$q_{\text{max}}$~取不同值得到的极点。(以MeV为单位)}
\centering
\begin{spacing}{1.2}
\begin{tabular}{c c c c c}
\toprule\hline
$q_{\text{max}}$ & $605$                  & $630$                   & $650$                   & $680$ \\
\hline
\multirow{2}*{poles}      &$1570.5+133.3i$ & $1569.4+125.7i$ & $1568.1+119.8i$ & $1565.6+111.4i$ \\
           & $1688.8 + 0.3i$  & $1687.9 + 0.7i$      & $1686.6+1.1i$     & $1684.0 + 1.5i$ \\
\hline\hline
$q_{\text{max}}$ & $700$                 & $710$                 &  $720$                &$730$ \\
\hline
\multirow{2}*{poles}      & $1563.7 + 106.1i$ & $1562.6 + 103.5i$ & $1561.5 + 101.1i$ & $1560.4 + 98.7i$ \\
           & $1681.8 + 1.8i $    & $1680.5 + 1.9i$     &  $1679.2 + 2.0i$    &  $1677.8 + 2.1i$ \\
\hline\hline
$q_{\text{max}}$ & $740$               & $750$                & $760$                & $770$ \\
\hline
\multirow{2}*{poles}      & $1559.2 + 96.3i$ & $1558.0 + 94.0i$ & $1556.8 + 91.8i$ & $1555.6 + 89.7i$ \\
           & $1676.3 + 2.2i $  & $1674.8 + 2.3i$    &  $1673.2 + 2.3i$   &  $ 1671.5 + 2.4i$ \\
\hline\hline
$q_{\text{max}}$ & $780$                & $790$               &$800$                 & $810$ \\
\hline
\multirow{2}*{poles}      & $1554.3 + 87.6i$ & $1553.0 + 85.5i$ & $1551.7 + 83.6i$ & $1550.4 + 81.6i$ \\
           & $1669.8 + 2.5i$    &  $1668.0 + 2.5i$  & $1666.2 + 2.6i$    & $1664.3 + 2.6i$ \\
\bottomrule
\end{tabular}
\end{spacing}
\label{tab:a3}
\end{table}\par
\subsubsection{利用三动量截断的$G$函数}
现在我们采用式\eqref{eqcG}和\eqref{eqcG2}所示的三动量截断重整化的$G$函数,求解耦合道BS方程,寻找$T$矩阵的极点。计算的唯一自由参数是三动量截断$q_{\text{max}}$。取$q_{\text{max}}$的不同值,计算所得到的极点列于表\ref{tab:a3}中。
由表\ref{tab:a3}可见,当$q_{\text{max}}$取不同值时,都能找到$T$振幅的两个极点,其中第一个极点的质量较小但宽度很大,第二个极点的质量较大而宽度很窄。当$q_{\text{max}}$在$[605,810]\;\si{MeV}$范围内变化时,第二个极点宽度改变很小,其质量从$\SI{1688.8}{MeV}$减小为$\SI{1664.3}{MeV}$,变化量为$\SI{24.5}{MeV}$。总体而言,第二个极点对参数$q_{\text{max}}$的改变并不敏感。对于第一个极点,当$q_{\text{max}}$在$[605,810]\;\si{MeV}$范围内变化时,第一个极点的质量从$\SI{1570.5}{MeV}$减小为$\SI{1550.4}{MeV}$,改变量约为$\SI{20}{MeV}$,可以说第一个极点的位置对参数$q_{\text{max}}$的取值并不敏感,但其宽度的改变较大。\par
以$q_{\text{max}}=\SI{630}{MeV}$的情况为例,第二个极点为$1687.9+i\SI{0.7}{MeV}$,其质量和宽度与$\Xi(1690)$相符,可认为第二个极点对应于$\Xi(1690)$共振态。两个极点与各反应道的耦合常数如表\ref{tab:a4}所示。可以看到,第二个极点主要耦合到$\bar{K}\Sigma$道,与$\eta\Xi$道的耦合也较强,但与$\pi\Xi$和$\bar{K}\Lambda$的耦合很弱。相反地,第一个极点与$\pi\Xi$和$\bar{K}\Lambda$强耦合,而与$\bar{K}\Sigma$和$\eta\Xi$的耦合较弱,这与表\ref{tab:a2}中的极点性质相一致。因此,我们认为第一个极点对应于$\Xi(1620)$共振态。\par
\begin{table}[t]
	\caption[两个动力学产生态与各反应道的耦合常数]{当$q_{\text{max}}=\SI{630}{MeV}$时,两个动力学产生态与各反应道的耦合常数。}
\label{tab:a4}
\centering
\begin{spacing}{1.2}
\begin{tabular}{ccccc}
\hline\hline
   $ \bm{1569.4+i125.7}$ & $\pi \Xi$          & $\bar K\Lambda$        & $\bar K\Sigma$    & $\eta \Xi$ \\
   \hline
   $g_i$                   & $-2.0-1.6i$  & $1.9+0.9i$  & $0.7+0.5i$  & $-0.1-0.4i$ \\
   $|g_i|^2$               & $6.6$             & $4.5$            & $0.6$            & $0.1$ \\
   \hline\hline
   $\bm{1687.9 +i 0.7}$  & $\pi \Xi$          & $\bar K\Lambda$        & $\bar K\Sigma$    &$\eta \Xi$ \\
   \hline
   $g_i$                   & $0.1-0.1i$       & $0.2+0.1i$      & $-1.2+0.2i$   & $-0.8+0.1i$ \\
   $|g_i|^2$               & $0.01$             & $0.04$            & $1.5$            & $0.6$ \\
   \hline\hline
\end{tabular}
\end{spacing}
\end{table}
本章中,我们在手征幺正法的理论框架下,重新研究了同位旋$I=\frac{1}{2}$的$\pi\Xi$及其耦合道散射,分别采用了维数正规化方法和三动量截断法来处理发散的介子-重子传播子圈函数$G$。研究结果表明:当采用维数正规化的$G$函数时,只能找到散射振幅$T$的一个极点,该极点对应于$\Xi(1620)$超子激发态;当利用三动量截断的$G$函数时,散射振幅$T$存在两个极点,分别对应于$\Xi(1620)$和$\Xi(1690)$共振态。$\Xi(1620)$与$\pi\Xi$和$\bar{K}\Lambda$强耦合,而与$\bar{K}\Sigma$和$\eta\Xi$道的耦合较弱,$\Xi(1620)$被认为是以$\pi\Xi$和$\bar{K}\Lambda$为主要成分的介子-重子分子态。$\Xi(1690)$与$\bar{K}\Sigma$道的耦合最强,而与$\pi\Xi$和$\bar{K}\Lambda$的耦合很弱,$\Xi(1690)$被认为是以$\bar{K}\Sigma$为主要成分的介子-重子分子态。\par
由于Belle合作组在$\Xi_{c}^{+}\to\Xi^{-}\pi^{+}\pi^{+}$衰变的$\Xi^{-}\pi^{+}$不变质量谱中观测到了明显的$\Xi(1620)$和$\Xi(1690)$产生信号,因此,我们在第三章中对$\Xi_{c}^{+}\to\Xi^{-}\pi^{+}\pi^{+}$衰变进行理论研究时,将采用三动量截断的$G$函数来计算$\pi\Xi$二体重散射振幅,并给出$\Xi^{-}\pi^{+}$不变质量谱的理论预言,考察$\Xi(1620)$和$\Xi(1690)$的分子态结构效应。
