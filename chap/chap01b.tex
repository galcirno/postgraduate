\chapter{引言}
\label{chap01}
粒子物理主要研究比原子更小的亚原子粒子的性质、结构、相互作用及转化规律。自1897年发现电子以来,人们陆陆续续发现了为数不少的亚原子粒子,例如中子和质子等等(中子和质子是由夸克组成的重子)。微观粒子具有波粒二象性,我们用希尔伯特空间中的态矢量来描写粒子的状态。为了更好的描述粒子的产生、湮灭过程,结合经典场论,我们使用量子场论来描述微观粒子\cite{braibant2011particles}。\par
二十世纪下半叶科学家们建立起了粒子物理的标准模型(Standard Model)。这一标准模型中包括了四种基本相互作用中的三种:电磁相互作用、弱相互作用和强相互作用,只有引力仍然没有包括进来。因此许多科学家在寻找一个能够包含四种基本力的大一统理论,但也有一些科学家认为引力是一种特殊的力,因此不存在大一统的理论。\par
电磁相互作用是至今为止人们了解得最为清楚的一种基本相互作用,其规律由量子电动力学(QED)描写。关于弱相互作用,最初人们建立起弱相互作用的唯象理论,如四费米子弱相互作用理论。后来,人们认识到弱相互作用和电磁相互作用有统一的来源,建立起了电弱统一理论。强相互作用的作用荷是色荷,夸克、胶子等微观粒子具有非零的色荷,它们之间存在强相互作用。量子色动力学(QCD)被认为是描写强相互作用的基本理论。因此,电弱统一理论和QCD构成了粒子物理的标准模型。
\section{基本粒子}
标准模型里面有61种基本粒子,如表\ref{Elementary Particles}所示,这61种基本粒子构成了其它数百种粒子\cite{braibant2011particles}。代(Generations)是基本粒子的一种划分,不同代之间粒子的味道量子数和质量不同,但是电磁相互作用和强相互作用是一样的。表\ref{Classification of matter}列出了夸克和轻子的代。
\begin{table}[h]
\centering
\caption{\label{Elementary Particles}基本粒子}
%\label{Elementary Particles}
\setlength{\tabcolsep}{8mm}
\begin{spacing}{1.5}
\begin{tabular}{|c|c|c|c|c|c|}
	\Xhline{1pt}
	 & 类 & 代 & 反粒子 & 颜色 & 总计\\\hline
	夸克 & \multirow{2}*{2} & \multirow{2}*{3} & 反夸克 & 3 & 36\\\cline{1-1}\cline{4-6}
	轻子 & ~ & ~ & 反轻子 & 无 & 12\\\hline
	胶子 & \multirow{5}*{1} & \multirow{5}*{无} & 自身 & 8 & 8\\\cline{1-1}\cline{4-6}
	光子 & ~ & ~ & 自身 & \multirow{4}*{无} & 1\\\cline{1-1}\cline{4-4}\cline{6-6}
	Z 玻色子 & ~ & ~ & 自身 & ~ & 1\\\cline{1-1}\cline{4-4}\cline{6-6}
	W 玻色子 & ~ & ~ & 反W 玻色子 & ~ & 2\\\cline{1-1}\cline{4-4}\cline{6-6}
	希格斯子 & ~ & ~ & 自身 & ~ & 1\\\hline
	\multicolumn{5}{|c|}{总计} & 61\\
	\Xhline{1pt}
\end{tabular}
\end{spacing}
\end{table}
\begin{table}[h]
\centering
\caption{夸克和轻子的代}
\label{Classification of matter}
\setlength{\tabcolsep}{8mm}
\begin{spacing}{1.5}
\begin{tabular}{|c|c|c|c|}
	\Xhline{1pt}
	类 & 第一代 & 第二代 & 第三代\\\hline
	\multicolumn{4}{|c|}{夸克}\\\hline
	上 & 上夸克(up, $u$) & 粲夸克(charm, $c$) & 顶夸克(top, $t$)\\\hline
	下 & 下夸克(down, $d$) & 奇夸克(strange, $s$) & 底夸克(bottom, $b$)\\\hline
	\multicolumn{4}{|c|}{轻子} \\\hline
	带电 & 电子($e^{-}$) & $\mu$子($\mu^{-}$) &$\tau$子($\tau^{-}$) \\\hline
	中性 & 电子中微子($\nu_{e}$) & $\mu$子中微子($\nu_{\mu}$) & $\tau$子中微子($\nu_{\tau}$)\\
	\Xhline{1pt}
\end{tabular}
\end{spacing}
\end{table}
后来,人们发现在现有的自由度下仍然无法解释一些现象,夸克应该有额外的量子数。所以引入了三个自由度称为色(Colours),分别为红($r$)、绿($g$)、蓝($b$)。自然界观测到的强子(质子、中子等直接参与强相互作用的亚原子粒子)都处于色单态,在数学上类似于自旋单态\cite{griffiths2004introduction}:
\begin{equation*}
	\frac{r\bar{r}+b\bar{b}+g\bar{g}}{\sqrt{3}}.
\end{equation*}
\newpage
而胶子并不是色单态,是颜色混合的八重态,常见的表示为:
\begin{equation*}
\begin{split}
	&\frac{r\bar{b}+b\bar{r}}{\sqrt{2}},\qquad -i\frac{r\bar{b}-b\bar{r}}{\sqrt{2}},\qquad
	\frac{r\bar{g}+g\bar{r}}{\sqrt{2}},\qquad -i\frac{r\bar{g}-g\bar{r}}{\sqrt{2}},\\[2ex]
	&\frac{b\bar{g}+g\bar{b}}{\sqrt{2}},\qquad -i\frac{b\bar{g}-g\bar{b}}{\sqrt{2}},\qquad
	\frac{r\bar{r}-b\bar{b}}{\sqrt{2}},\qquad \frac{r\bar{r}+\bar{b}-2g\bar{g}}{\sqrt{6}}.
\end{split}
\end{equation*}
可以用如下8个Gell-Mann矩阵$\lambda_{i}$来描述,
\begin{align*}
	&\lambda_{1}=
	\begin{pmatrix}
		0 & 1 & 0 \\
		1 & 0 & 0 \\
		0 & 0 & 0 \\
	\end{pmatrix},\quad
	\lambda_{2}=
	\begin{pmatrix}
		0 & -i & 0 \\
		i & 0 & 0 \\
		0 & 0 & 0 \\
	\end{pmatrix},\quad
	\lambda_{3}=
	\begin{pmatrix}
		1 & 0 & 0 \\
		0 & -1 & 0 \\
		0 & 0 & 0 \\
	\end{pmatrix},\quad
	\lambda_{4}=
	\begin{pmatrix}
		0 & 0 & 1 \\
		0 & 0 & 0 \\
		1 & 0 & 0 \\
	\end{pmatrix},\\
	&\lambda_{5}=
	\begin{pmatrix}
		0 & 0 & -i \\
		0 & 0 & 0 \\
		i & 0 & 0 \\
	\end{pmatrix},\quad
	\lambda_{6}=
	\begin{pmatrix}
		0 & 0 & 0 \\
		0 & 0 & 1 \\
		0 & 1 & 0 \\
	\end{pmatrix},\quad
	\lambda_{7}=
	\begin{pmatrix}
		0 & 0 & 0 \\
		0 & 0 & -i \\
		0 & i & 0 \\
	\end{pmatrix},\quad
	\lambda_{8}=\frac{1}{\sqrt{3}}
	\begin{pmatrix}
		1 & 0 & 0 \\
		0 & 1 & 0 \\
		0 & 0 & -2 \\
	\end{pmatrix}.
\end{align*}\par
%在标准模型中所有的基本粒子见图\ref{smep}。
%\begin{figure}[h]
%	\centering
%	\includegraphics[width=15cm]{smepa}
%	\caption{标准模型中的基本粒子}
%	\label{smep}
%\end{figure}
%\subsection{标准模型的拉氏量}
%标准模型的拉氏量构造的方法是:先得到系统的对称性,然后观测粒子场的对称性写出普适的可归一化的拉氏量
%\subsubsection{量子色动力项}
%量子色动力学是描述夸克和胶子之间的相互作用,夸克与胶子场耦合的拉氏量为
%\begin{equation}
%	\mathcal{L}_{QCD}=\bar{\psi}_{i}(i(\gamma^{\mu}D_{\mu})_{ij}-m\delta_{ij})\psi_{j}-\frac{1}{4}G_{\mu\nu}^{a}G^{\mu\nu}_{a}.
%\end{equation}
%其中$\psi_{i}$是夸克场,下标代表夸克颜色的分量($r,g,b$),$\gamma^{\mu}$是狄拉克矩阵,Weyl表象的形式如公式\eqref{gamma matrix}所示\cite{peskin2018introduction}。拉氏量中的 $D_{\mu}$是规范协变微分,表达式为公式\eqref{gaue convariant derivative of quantum chromodynamics},$G^{a}_{\mu\nu}$是规范不变的胶子场强张量,与电磁场强张量类似,具体形式如公式\eqref{gluon field strength tensor}所示\cite{Eidemuller:1999mx,Greiner:1995jn}。公式中的 $\mathcal{A}^{a}_{\mu}$是胶子场, $g_{s}$是强相互作用的耦合常数, $\lambda_{a}$是Gell-Mann矩阵。$f^{a}_{\ bc}$是一个结构常数,见公式\eqref{fabc}。
%\begin{equation}
%\label{gamma matrix}
%\gamma^{0}=
%\begin{pmatrix}
%	0 & 1 \\
%	1 & 0 \\
%\end{pmatrix};\qquad
%\gamma^{i}=
%\begin{pmatrix}
%	0 &\sigma^{i} \\
%	-\sigma^{i} & 0 \\
%\end{pmatrix}
%\end{equation}
%\begin{equation}
%\label{gaue convariant derivative of quantum chromodynamics}
%D_{\mu}=\partial_{\mu}-ig_{s}\mathcal{A}^{a}_{\mu}\lambda_{a}/2
%\end{equation}
%\begin{equation}
%\label{gluon field strength tensor}
%G^{a}_{\mu\nu}=\partial_{\mu}\mathcal{A}^{a}_{\nu}-\partial_{\nu}\mathcal{A}^{a}_{\mu}+g_{s}f^{a}_{\ bc}\mathcal{A}^{b}_{\mu}\mathcal{A}^{c}_{\nu}
%\end{equation}
%\begin{equation}
%\label{fabc}
%[\lambda_{i},\lambda_{j}]=2if_{ijk}\lambda_{k}
%\end{equation}
%\subsubsection{其他项}
%在电弱对称破缺前,我们的拉氏量为
%\begin{equation}
%	\mathcal{L}=\mathcal{L}_{EW}+L_{H}+L_{Y},
%\end{equation}
%其中电弱项$\mathcal{L}_{EW}$为
%\begin{equation}
%\begin{split}
%	\mathcal{L}_{EW}=&\mathcal{L}_{g}+\mathcal{L}_{f}\\
%	=&-\frac{1}{4}W^{\mu\nu}_{a}W^{a}_{\mu\nu}-\frac{1}{4}B^{\mu\nu}B_{\mu\nu}+\bar{\psi}_{j}i\gamma^{\mu}D_{\mu}\psi_{j},
%\end{split}
%\end{equation}
%其中 $W^{a\mu\nu}(a=1,2,3)$和$B^{\mu\nu}$ 分别是弱同位旋规范场和弱超荷规范场的场强张量。这里的规范协变微分$D_{\mu}$为
%\begin{equation}
%\label{ewD}
%	D_{\mu}=\partial_{\mu}-i\frac{g'}{2}YB_{\mu}-i\frac{g}{2}T_{j}W^{j}_{\mu},
%\end{equation}
%上式中$Y$是弱超荷,$T_{j}$是泡利矩阵 $B_{\mu}$和$W_{\mu}^{j}$分别是U(1)和SU(2)的规范场,$g'$和$g$分别是U(1)和SU(2)的耦合常数。\par
%Higgs项$\mathcal{L}_{H}$为
%\begin{equation}
%	\mathcal{L}_{H}=\abs{D_{\mu}\psi}^2-\frac{\lambda^2}{4}(\psi^{\dagger}\psi-v^2)^2,
%\end{equation}
%Yukawa项$\mathcal{L}_{Y}$为
%\begin{equation}
%	\mathcal{L}_{Y}=-y_{uij}\epsilon^{ab}h^{\dagger}_{b}\bar{Q}_{ia}u^{c}_{j}-y_{dij}h\bar{Q}_{i}d^{c}_{j}-y_{eij}h\bar{L}_{i}e_{j}^{c}+h.c.\ .
%\end{equation}

\section{强相互作用的非微扰研究方法}
质子和中子统称为核子,它们紧密地束缚成致密而稳定的原子核。由于质子带一个单位的正电荷,而中子是电中性的,因而核子间不可能通过电磁相互作用紧密地束缚成原子核,肯定存在一个比电磁相互作用强得多的力——强相互作用使核子束缚成原子核。\par
描写强相互作用动力学的基本理论是量子色动力学(QCD)。QCD是颜色空间的SU(3)$_c$非阿贝尔规范场论\cite{PhysRevLett.30.1343,PhysRevLett.30.1346,greensite2011introduction},带色的胶子是该理论中的规范玻色子。渐近自由和色禁闭(也称为夸克禁闭)是QCD理论的最主要特征。在大动量转移情况下(相当于两个夸克间距离很小时),QCD耦合常数趋于零,即夸克间的相互作用变弱而趋于自由粒子,QCD的这一性质称为渐近自由。这一性质已由轻子-核子深度非弹性散射等高能标度下的实验所证实。色禁闭意味着,带色的夸克和胶子被禁闭在强子内部,强子表现为白色或无色的色单态。色禁闭的机制至今仍然是未知的。\par
渐近自由是QCD微扰论的基础,高能区的微扰QCD取得了巨大的成功。强子的能量标度大多在GeV的量级,属于QCD的低能标度。在低能区,QCD是高度非线性的,不能严格求解,微扰QCD方法也不再适用。因此,要研究强子物理问题,需要发展QCD的非微扰途径。目前发展出来的QCD非微扰途径包括格点QCD、有效场论方法以及其它的唯象方法。\par
格点QCD是最具QCD精神的非微扰QCD途径。这一方法将欧几里德时空离散化为格点,夸克场定义在格点上,胶子场则连接相邻的格点\cite{PhysRevLett.92.022001,PhysRevD.10.2445},QCD拉氏量以格点规范形式表示。格点QCD面临的问题是随着晶格间距的缩小计算量急剧地增加,这使得对硬件的需求很高,给格点QCD的应用带来限制。\par
有效理论旨在只对确定的某种效果建模,而不考虑该效果的内部机制\cite{stamatescu2007approaches}。有效理论作为一种科学的理论有许多重要的成果,如费米的$\beta$衰变理论、BCS超导理论等等。在强子物理中有效理论有手征微扰论和手征幺正法等。
手征微扰论是一种手征对称QCD的有效场论,因此在手征对称破缺标度下的情况可以使用手征微扰论\cite{Leutwyler:2012,LEUTWYLER1994165,scherer2011primer,GASSER1984142,GASSER1985465}。由于夸克禁闭的效应,在低能的情况下自由度就变成强子而不再是QCD中的夸克场和胶子场。手征微扰论是一种不可重整的理论,而且无法用来描述共振态。因为手征微扰论的局限性,所以文献\cite{OLLER1997438}基于手征微扰论构造出了手征幺正法。\par
手征幺正法利用了手征微扰论中的$\order{p^2}$阶的手征拉氏量\cite{KAISER1995325},把得到的最低阶的振幅作为散射动力学方程Bethe-Salpeter (BS)方程的核。手征幺正法可以很好地描述共振态,文献\cite{OLLER1997438}中在第二黎曼面得到的标量介子共振态$\sigma$、$a_{0}(980)$、$f_{0}(980)$和$\kappa$的质量和宽度与实验数据符合得很好。\vspace{-1cm}
\section{强子的结构模型}%\vspace{-1cm}
质子和中子是人类最早观测到的强子,此后人们陆续发现了$\pi$介子,$\Lambda$超子等越来越多的强子。到上个世纪六十年代,实验上发现了一百多种强子。人们逐渐认识到数目如此庞大的强子不可能都是基本粒子,它们有内部结构,由更基本的基元构成。\vspace{-0.5cm}
\subsection{夸克模型下的强子态}
1964年Gell-Mann和G. Zweig分别独立地提出强子结构的夸克模型\cite{GellMann:1964nj,Zweig:1981pd,Zweig:1964jf},对当时实验上观测到的众多强子进行了成功分类,这一模型认为,强子由更基本的夸克($q$)和反夸克($\bar{q}$)构成,$q$和$\bar{q}$都是自旋为$\frac{1}{2}$费米子,它们都具有分数电荷。夸克的重子数为$\frac{1}{3}$,反夸克的重子数为$-\frac{1}{3}$。介子由一对正反夸克($q\bar{q}$)组成,重子由三个夸克($qqq$)组成。
目前已证实,自然界存在6种不同类型的夸克,也称为6种不同味道的夸克,分别是:$u,d,s,c,b,t$。$u,d,s$夸克质量较小,被称为轻夸克。本文仅限于讨论由轻夸克构成的强子,表\ref{uds quark}列出了轻夸克$u,d,s$的量子数,其中,$I$是同位旋量子数,$I_{3}$是同位旋第三分量,$B$是重子数,$S$是奇异数,$Y$是超荷,$Q$是电荷数。
夸克模型在解释自旋-宇称$J^{PC}=0^{-+}$赝标介子基态、$J^{PC}=1^{--}$矢量介子基态、$J^{P}=\frac{1}{2}^{+}$和$J^{P}=\frac{3}{2}^{+}$的重子基态方面取得了成功。
\begin{table}[h]
\centering
\caption{轻夸克的量子数}
\label{uds quark}
\begin{spacing}{1.5}
\setlength{\tabcolsep}{8mm}
\begin{tabular}{crrrrrr}
\toprule
\hline
	夸克的味道 & $I$ & $I_{3}$ & $B$ & $S$ & $Y$ & $Q$\\
	\hline
	$u$ & $\frac{1}{2}$ & $\frac{1}{2}$ & $\frac{1}{3}$ & 0 & $\frac{1}{3}$ & $\frac{2}{3}$\\
	$d$ & $\frac{1}{2}$ & $-\frac{1}{2}$ & $\frac{1}{3}$ & 0 & $\frac{1}{3}$ & $-\frac{1}{3}$\\
	$s$ & 0 & 0 & $\frac{1}{3}$ & -1 & $-\frac{2}{3}$ & $-\frac{1}{3}$\\
	\hline
\bottomrule
\end{tabular}
\end{spacing}
\end{table}
\subsubsection{赝标介子}
对于由$u,d,s$构成的$q\bar{q}$介子态,可以得到一个味道SU(3)八重态和一个味道SU(3)单态。定义一个不可约张量$M^{i}_{j}$,\vspace{-0.4cm}
\begin{equation}
\begin{split}
	M^{i}_{j}=&q^{i}\bar{q}_{j}-\frac{1}{3}\delta^{i}_{j}q^{k}\bar{q}_{k}\\[2ex]
	=&
	\begin{pmatrix}
		\dfrac{2u\bar{u}-d\bar{d}-s\bar{s}}{3} & u\bar{d} & u\bar{s} \\
		d\bar{u} & \dfrac{2d\bar{d}-u\bar{u}-s\bar{s}}{3} & d\bar{s} \\
		s\bar{u} & s\bar{d} & \dfrac{2s\bar{s}-u\bar{u}-d\bar{d}}{3} \\
	\end{pmatrix}.
\end{split}
\end{equation}
对于赝标介子($J^{P}=0^{-}$),根据量子数可得
\begin{equation}
\begin{split}
	&K^{0}=d\bar{s},\qquad\qquad K^{+}=u\bar{s},\quad \pi^{+}=u\bar{d},\quad\pi^{-}=d\bar{u},\\[2ex]
	&\pi^{0}=\frac{-d\bar{d}+u\bar{u}}{\sqrt{2}},\quad K^{-}=s\bar{u},\quad \bar{K}^{0}=s\bar{d},\quad \eta_{8}=\frac{u\bar{u}+d\bar{d}-2s\bar{s}}{\sqrt{6}},
\end{split}
\end{equation}
最后可得到赝标介子八重态矩阵,
\begin{equation}
\label{meson8}
	\Phi=
	\begin{pmatrix}
		\dfrac{\eta_{8}}{\sqrt{6}}+\dfrac{\pi^{0}}{\sqrt{2}} & \pi^{+} & K^{+}  \\
		\pi^{-} & \dfrac{\eta_{8}}{\sqrt{6}}-\dfrac{\pi^{0}}{\sqrt{2}} & K^{0} \\
		K^{-} & \bar{K}^{0} & -\dfrac{2\eta_{8}}{\sqrt{6}} \\
	\end{pmatrix},
\end{equation}
单态为
\begin{equation}
\label{meson1}
	\eta_{1}=\frac{u\bar{u}+d\bar{d}+s\bar{s}}{\sqrt{3}}.
\end{equation}
\subsubsection{矢量介子}
对于矢量介子,只要做以下的变换:$\pi\to\rho,K\to K^{*},\eta_{8}\to\omega_{8},\eta_{1}\to\omega_{1}$.
\subsubsection{重子}
$J^{P}=\frac{1}{2}^{+}$的基态重子八重态为
\begin{equation}
\label{baryon quark}
\begin{aligned}
	&\Sigma^{+}=\frac{1}{\sqrt{2}}(suu-usu),&\quad& p=\frac{1}{\sqrt{2}}(udu-duu),\quad \ \Sigma^{-}=\frac{1}{\sqrt{2}}(dsd-sdd),\\[2ex]
	&n=\frac{1}{\sqrt{2}}(udd-dud),&\quad& \Xi^{-}=\frac{1}{\sqrt{2}}(dss-sds),\quad\Sigma^{0}=\frac{1}{2}(usd+dsu-sud-sdu),\\[2ex]
	&\Xi^{0}=\frac{1}{\sqrt{2}}(sus-uss),&\quad&\Lambda^{0}=\frac{1}{\sqrt{12}}\big[(ds-sd)u+(su-us)d-2(ud-du)s\big].
\end{aligned}
\end{equation}
于是$J^{P}=\frac{1}{2}^{+}$的基态重子SU(3)八重态矩阵为
\begin{equation}
\label{baryon8}
	B=
	\begin{pmatrix}
		\frac{1}{\sqrt{2}}\Sigma^{0}+\frac{1}{\sqrt{6}}\Lambda & \Sigma^{+}&p  \\
		\Sigma^{-} & -\frac{1}{\sqrt{2}}\Sigma^{0}+\frac{1}{\sqrt{6}}\Lambda & n \\
		\Xi^{-} & \Xi^{0} & -\frac{2}{\sqrt{6}}\Lambda \\
	\end{pmatrix}.
\end{equation}
$J^{P}=\frac{3}{2}^{+}$的基态重子十重态:
\begin{equation}
\begin{aligned}
	&\Omega^{-}=sss,\quad \Xi^{*0}=\frac{1}{\sqrt{3}}(uss+sus+ssu),\quad \Xi^{*-}=\frac{1}{\sqrt{3}}(dss+sds+ssd),\\[2ex]
	&\Sigma^{*+}=\frac{1}{\sqrt{3}}(suu+usu+uus),\quad \Sigma^{-}=\frac{1}{\sqrt{3}}(sdd+dsd+dds),\\[2ex]
	&\Sigma^{0}=\frac{1}{\sqrt{6}}(sud+dsu+uds+sdu+dus+usd),\quad\Delta^{++}=uuu,\\[2ex]
	&\Delta^{+}=\frac{1}{\sqrt{3}}(uud+udu+duu),\quad\Delta^{0}=\frac{1}{\sqrt{3}}(udd+dud+ddu)\quad\Delta^{-}=ddd.
\end{aligned}
\end{equation}
%强相互作用具有电荷无关性,因此在强相互作用下系统满足同位旋守恒定律,同位旋量子数$I$及其第三分量$I_{3}$都是不变的。根据实验得知,在强相互作用下系统有许多守恒量,例如奇异数、重子数、粲数、底数。
\subsection{奇特强子态及其候选者}
除了SU(3)夸克模型外,还有其他的色单态结构也应该是被允许的,如两个夸克和两个反夸克组成的介子还有四个夸克和一个反夸克组成的重子。\par
对于四夸克的介子和五夸克的重子除了可能是紧密的结合在一起,还有一种可能的结构是强子分子态,即两个强子形成类似分子的结构组成松散的结合态。分子态的结合能为两个强子的阈值与质量的差。四夸克的介子可以是两个介子构成的介子-介子分子态,对于五夸克的重子来说,就可能是一个介子与一个重子形成介子-重子分子态。对于强子分子态中的强子-强子相互作用通常使用有效场论来研究。\par
在2003年,BaBar、CLEO和Belle合作组分别发现了无法用SU(3)夸克模型解释的粲-奇介子$D^{*}_{s0}(2317)$,$D_{s1}(2460)$和类粲偶素$X(3872)$。Godfrey和Isgur预测的质量要比实验低许多\cite{godfrey1985mesons}。$D^{*}_{s0}(2317)$和$D_{s1}(2460)$理论预测可能的结构有$DK$束缚态、四夸克态等,$X(3872)$可能是分子态、四夸克态和混杂态等。\par
在实验上也发现了许多无法用传统夸克模型解释的重子。
2015年,LHCb在$\Lambda_b\rightarrow J/\psi K^-p$衰变过程中观测到了$J/\psi p$不变质量谱的两个峰引起了一大热议,它们就是被认为是五夸克态的$P_{c}^{+}(4380)$和$P_{c}^{+}(4450)$\cite{PhysRevLett.115.072001,Aaij_2016}。在文献\cite{Aaij2014}中LHCb合作组还在$\Lambda_{b}^{0}\to J/\psi p\pi^{-}$的衰变过程中$J/\psi p$的质量分布中发现了相同质量的峰。文献\cite{PhysRevD.93.094001}采用了文献\cite{PhysRevD.92.094003}相同的方法证明了两个峰的一致性。
文献\cite{Roca2015}使用手征幺正法研究了$\Lambda_b\rightarrow J/\psi K^-p$并预测了在$K^{-}p$和$\pi\Sigma$谱中的$\Lambda(1405)$\cite{Dalitz2000},其中计算的$K^{-}p$分布与实验数据吻合的很好。\par
手征幺正法作为一种低能QCD有效场论方法,广泛应用于介子-介子和介子-重子强相互作用体系,可以动力学产生具有强子分子态结构特性的共振态。手征幺正法在$S=-1$的领域取得很好的结果,现在想推广到$S=-2$的情况。在$\Xi$的共振态中,我们知道$\Xi(1530)$的$I(J^{P})=\frac{1}{2}(\frac{3}{2}^{+})$,但接下来的$\Xi(1620)$和$\Xi(1690)$,它们的自旋-宇称量子数并不清楚。
二十世纪七十年代人们在$K^{-}p$相互作用过程中发现了$\Xi(1620)\to\Xi\pi$的衰变\cite{deBellefon1975,PhysRevD.16.2706,ROSS1972177}。在$K^{-}p$相互作用过程中和$\Xi^{-}N$相互作用过程中,$\Xi^{-}$衰变和 $\Lambda_{c}^{+}$衰变过程中都发现了$\Xi(1690)$\cite{DIONISI1978145,Biagi1981,Biagi1987,ABE200233}。\par
$\Xi(1620)$和$\Xi(1690)$共振态的质量太低了,以至于大多数的夸克模型无法解释,在那些模型中产生$\Xi$共振态的能量要在$\SI{1800}{MeV}$以上\cite{PhysRevD.34.2809,Blask:1990ez}。
在$\Lambda_{c}^{+}\to \Xi^{-}\pi^{+}K^{+}$的衰变过程中,找到了一些证据证明$\Xi(1690)$的$J^{P}$量子数为$1/2^{-}$。本文中,在手征幺正法的理论框架下,认为$\Xi(1620)$和$\Xi(1690)$是$\pi\Xi$(及其耦合道)$s$-波相互作用动力学产生的介子-重子分子态,它们的性质为$I(J^{P})=\frac{1}{2}(\frac{1}{2}^{-})$。我们将使用SU(3)的手征拉氏量\cite{Pich_1995,ECKER19951,BERNARD_1995,Meissner_1993},计算赝标介子八重态和重子八重态的耦合道。\par
本文的结构如下:第二章先将手征幺正法应用于$S=-2$、$I=\frac{1}{2}$的$\pi\Xi$及其耦合道的$s$-波散射,使用三动量截断法对介子-重子传播子圈函数进行重整化。最后求解Bethe-Salpeter (BS)方程动力学重现两个超子激发态$\Xi(1620)$和$\Xi(1690)$,并分别给出它们的极点和与各耦合道的耦合常数。第三章将结合Belle合作组在文献\cite{PhysRevLett.122.072501}中的实验数据,分析具体的衰变过程($\Xi_{c}^{+}\to\Xi^{-}\pi^{+}\pi^{+}$)。计算了这两个超子激发态对$\Xi^{-}\pi^{+}$不变质量谱的贡献和背景道的贡献,并把它们与实验数据进行比较。最后一章是对全文进行的一个总结。
