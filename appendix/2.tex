\subsection{$C_{ij}$的计算过程}
\label{appendices a}
\subsubsection{$\pi\Xi\to\pi\Xi$}
由式\eqref{ib2}和\eqref{def v}可知,
\begin{equation}
\begin{split}
	V_{\pi\Xi\rightarrow\pi\Xi}=&\mel{\pi\Xi}{-\mathcal{L}}{\pi\Xi}\\
	=&\left(\sqrt{\frac{2}{3}}\bra{\pi^{+},\Xi^{-}}-\sqrt{\frac{1}{3}}\bra{\pi^{0},\Xi^{0}}\right)-\mathcal{L}\left(\sqrt{\frac{2}{3}}\ket{\pi^{+},\Xi^{-}}-\sqrt{\frac{1}{3}}\ket{\pi^{0},\Xi^{0}}\right)\\
	=&\frac{2}{3}V_{\pi^{+}\Xi^{-}\rightarrow\pi^{+}\Xi^{-}}-\frac{\sqrt{2}}{3}V_{\pi^{0}\Xi^{0}\rightarrow\pi^{+}\Xi^{-}}-\frac{\sqrt{2}}{3}V_{\pi^{+}\Xi^{-}\rightarrow\pi^{0}\Xi^{0}}+\frac{1}{3}V_{\pi^{0}\Xi^{0}\rightarrow\pi^{0}\Xi^{0}}
\end{split}
\end{equation}
下面对各个衰变道进行计算。
\subsubsection{$V_{\pi^{+}\Xi^{-}\to\pi^{+}\Xi^{-}}$}
首先要从拉氏量中挑选出与该衰变道相关的项,
\begin{equation}
	\mathcal{L}=-\frac{1}{4f^2}\bar{\Xi}^{-}i\gamma^{\mu}\Xi^{-}(\pi^{+}\partial_{\mu}\pi^{-}-\partial_{\mu}\pi^{+}\pi^{-})
\end{equation}
然后可以写出
\begin{equation}
	V_{\pi^{+}\Xi^{-}\rightarrow\pi^{+}\Xi^{-}}=\mel{\pi^{+}(k')\Xi^{-}(p')}{-\mathcal{L}}{\pi^{+}(k)\Xi^{-}(p)}
\end{equation}
由于交叉对称性,可以把末态的粒子变成初态上动量相反的反粒子\cite{peskin2018introduction},即
\begin{equation}
	V_{\pi^{+}\Xi^{-}\rightarrow\pi^{+}\Xi^{-}}=\mel{0}{-\mathcal{L}}{\pi^{-}(-k')\Xi^{+}(-p')\pi^{+}(k)\Xi^{-}(p)}
\end{equation}
在计算时还需要注意对称因子,在这里没有对称因子。最后把拉氏量代入可得
\begin{equation}
\begin{split}
	V_{\pi^{+}\Xi^{-}\rightarrow\pi^{+}\Xi^{-}}=&\frac{1}{4f^2}\mel{0}{\bar{\Xi}^{-}i\gamma^{\mu}\Xi^{-}(\pi^{+}\partial_{\mu}\pi^{-}-\partial_{\mu}\pi^{+}\pi^{-})}{\pi^{-}(-k')\bar{\Xi}^{-}(-p')\pi^{+}(k)\Xi^{-}(p)}\\
	=&\frac{1}{4f^2}\bar{u}(p')i\gamma^{\mu}u(p)(-i(-k_{\mu}')-(-i)k_{\mu})\\
	%=&\boxed{-\frac{1}{4f^2}\bar{u}(p')\gamma^{\mu}u(p)(k+k')}
	=&-\frac{1}{4f^2}\bar{u}(p')\gamma^{\mu}u(p)(k+k')
\end{split}
\end{equation}
同理,可以依次计算其他的衰变道。
\subsubsection{$V\pi^{0}\Xi^{0}\to\pi^{+}\Xi^{-}$}
\begin{equation}
	\mathcal{L}=-\sqrt{2}\frac{1}{4f^2}\bar{\Xi}^{-}i\gamma^{\mu}\Xi^{0}(\pi^{-}\partial_{\mu}\pi^{0}-\pi^{0}\partial_{\mu}\pi^{-})
\end{equation}
\begin{equation}
	V_{\pi^{0}\Xi^{0}\to\pi^{+}\Xi^{-}}=\left(-\sqrt{2}\right)\left(-\frac{1}{4f^2}\bar{u}(p')\gamma^{\mu}u(p)(k_{\mu}+k'_{\mu})\right)
\end{equation}
\subsubsection{$V_{\pi^{+}\Xi^{-}\to\pi^{0}\Xi^{0}}$}
\begin{equation}
	\mathcal{L}=-\frac{1}{4f^2}\sqrt{2}\bar{\Xi}^{0}i\gamma^{\mu}\Xi^{-}(\pi^{0}\partial_{\mu}\pi^{+}-\pi^{+}\partial_{\mu}\pi^{0})
\end{equation}
\begin{equation}
	V_{\pi^{+}\Xi^{-}\to\pi^{0}\Xi^{0}}={\left(-\sqrt{2}\right)\left(-\frac{1}{4f^2}\bar{u}(p')\gamma^{\mu}u(p)(k_{\mu}+k_{\mu}')\right)}
\end{equation}
\subsubsection{$V_{\pi^{0}\Xi^{0}\to\pi^{0}\Xi^{0}}$}
\begin{equation}
	\mathcal{L}=0
\end{equation}
\begin{equation}
	V_{\pi^{0}\Xi^{0}\rightarrow\pi^{0}\Xi^{0}}=0
\end{equation}
综上,
\begin{equation}
	V_{\pi\Xi\rightarrow\pi\Xi}=\frac{2}{3}V_{0}-\frac{\sqrt{2}}{3}\times(-\sqrt{2})V_{0}-\frac{\sqrt{2}}{3}\times(-\sqrt{2})=2V_{0}
\end{equation}
其中
\begin{equation}
V_{0}=\left(-\frac{1}{4f^2}\bar{u}(p')\gamma^{\mu}u(p)(k_{\mu}+k_{\mu}')\right)
\end{equation}
\subsubsection{$\pi\Xi\to \bar{K}\Lambda$}
\begin{equation}
	V_{\pi\Xi\to \bar{K}\Lambda}=\sqrt{\frac{2}{3}}V_{\pi^{+}\Xi^{-}\rightarrow \bar{K}\Lambda}-\sqrt{\frac{1}{3}}V_{\pi^{0}\Xi^{0}\rightarrow \bar{K}\Lambda}
\end{equation}
\subsubsection{$\pi^{+}\Xi^{-}\to \bar{K}\Lambda$}
\begin{equation}
	\mathcal{L}=\frac{1}{4f^2}\frac{3}{\sqrt{6}}\Lambda i\gamma^{\mu}\Xi^{-}(\pi^{+}\partial_{\mu}K^{0}-\partial_{\mu}\pi^{+}K^{0})
\end{equation}
\begin{equation}
V_{\pi^{+}\Xi^{-}\to \bar{K}\Lambda}=-\frac{3}{\sqrt{6}}\left(-\frac{1}{4f^2}\bar{u}(p')\gamma^{\mu}u(p)(k_{\mu}+k'_{\mu})\right)
\end{equation}
\subsubsection{$\pi^{0}\Xi^{0}\to \bar{K}\Lambda$}
\begin{equation}
	\mathcal{L}=\frac{1}{4f^2}\frac{3}{\sqrt{6}}\Lambda i\gamma^{\mu}\Xi^{0}(-\frac{1}{\sqrt{2}}\pi^{0}\partial_{\mu}K^{0}+\frac{1}{\sqrt{2}}\partial_{\mu}\pi^{0}K^{0})
\end{equation}
\begin{equation}
	V_{\pi^{0}\Xi^{0}\rightarrow \bar{K}^{0}\Lambda}=\frac{\sqrt{3}}{2}\left(-\frac{1}{4f^2}\bar{u}(p')\gamma^{\mu}u(p)(k_{\mu}+k'_{\mu})\right)
\end{equation}
\begin{equation}
\begin{split}
	V_{\pi\Xi\rightarrow \bar{K}\Lambda}=&\sqrt{\frac{2}{3}}(-\frac{3}{\sqrt{6}})V_{0}-\sqrt{\frac{1}{3}}(\frac{\sqrt{3}}{2})V_{0}\\
	=&-\frac{3}{2}V_{0}
\end{split}
\end{equation}
\subsubsection{$\pi\Xi\rightarrow \bar{K}\Sigma$}
\begin{equation}
	V_{\pi\Xi\rightarrow \bar{K}\Sigma}=\frac{\sqrt{2}}{3}V_{\pi^{+}\Xi^{-}\rightarrow \bar{K}^{0}\Sigma^{0}}-\frac{1}{3}V_{\pi^{0}\Xi^{0}\rightarrow \bar{K}^{0}\Sigma^{0}}-\frac{2}{3}V_{\pi^{+}\Xi^{-}\rightarrow K^{-}\Sigma^{+}}+\frac{\sqrt{2}}{3}V_{\pi^{0}\Xi^{0}\rightarrow K^{-}\Sigma^{+}}
\end{equation}
\subsubsection{$\pi^{+}\Xi^{-}\to \bar{K}^{0}\Sigma^{0}$}
\begin{equation}
	\mathcal{L}=\frac{1}{4f^2}\frac{1}{\sqrt{2}}\Sigma^{0}i\gamma^{\mu}\Xi^{-}(\pi^{+}\partial_{\mu}K^{0}-\partial_{\mu}\pi^{+}K^{0})
\end{equation}
\begin{equation}
	V_{\pi^{+}\Xi^{-}\rightarrow \bar{K}^{0}\Sigma^{0}}=\left(-\frac{1}{\sqrt{2}}\right)\left(-\frac{1}{4f^2}\bar{u}(p')\gamma^{\mu}u(p)(k_{\mu}+k'_{\mu})\right)
\end{equation}
\subsubsection{$\pi^{0}\Xi^{0}\to \bar{K}^{0}\Sigma^{0}$}
\begin{equation}
	\mathcal{L}=\frac{1}{4f^2}\frac{1}{2}\Sigma^{0}i\gamma^{\mu}\Xi^{0}(\pi^{0}\partial_{\mu}K^{0}-\partial_{\mu}\pi^{0}K^{0})
\end{equation}
\begin{equation}
	V_{\pi^{0}\Xi^{0}\rightarrow \bar{K}^{0}\Sigma^{0}}=\left(-\frac{1}{2}\right)\left(-\frac{1}{4f^2}\bar{u}(p')\gamma^{\mu}u(p)(k_{\mu}+k'_{\mu})\right)
\end{equation}
\subsubsection{$\pi^{+}\Xi^{-}\rightarrow K^{-}\Sigma^{+}$}
\begin{equation}
	\mathcal{L}=0
\end{equation}
\subsubsection{$\pi^{0}\Xi^{0}\rightarrow K^{-}\Sigma^{+}$}
\begin{equation}
	\mathcal{L}=\frac{1}{4f^2}\frac{1}{\sqrt{2}}\bar{\Sigma}^{+}i\gamma^{\mu}\Xi^{0}(\pi^{0}\partial_{\mu}K^{+}-\partial_{\mu}\pi^{0}K^{+})
\end{equation}
\begin{equation}
	V_{\pi^{0}\Xi^{0}\rightarrow \bar{K}^{-}\Sigma^{+}}=\left(-\frac{1}{\sqrt{2}}\right)\left(-\frac{1}{4f^2}\bar{u}(p')\gamma^{\mu}u(p)(k_{\mu}+k'_{\mu})\right)
\end{equation}
\begin{equation}
\begin{split}
	V_{\pi\Xi\rightarrow \bar{K}\Sigma}=&\frac{\sqrt{2}}{3}(-\frac{1}{\sqrt{2}})V_{0}-\frac{1}{3}(-\frac{1}{2})V_{0}+\frac{\sqrt{2}}{3}(-\frac{1}{\sqrt{2}})V_{0}\\
	=&-\frac{1}{2}V_{0}
\end{split}
\end{equation}
\subsubsection{$\pi\Xi\rightarrow\eta\Xi$}
\begin{equation}
	V_{\pi\Xi\rightarrow\eta\Xi}=\sqrt{\frac{2}{3}}V_{\pi^{+}\Xi^{-}\rightarrow\eta\Xi^{0}}-\sqrt{\frac{1}{3}}V_{\pi^{0}\Xi^{0}\rightarrow\eta\Xi^{0}}
\end{equation}
\subsubsection{$\pi^{+}\Xi^{-}\rightarrow\eta\Xi^{0}$}
\begin{equation}
	\mathcal{L}=0
\end{equation}
\subsubsection{$\pi^{0}\Xi^{0}\rightarrow\eta\Xi^{0}$}
\begin{equation}
	\mathcal{L}=0
\end{equation}
\begin{equation}
	V_{\pi\Xi\to\eta\Xi}=0
\end{equation}
\subsubsection{$\bar{K}\Lambda\rightarrow \bar{K}\Lambda$}
\begin{equation}
	V_{\bar{K}\Lambda\rightarrow \bar{K}\Lambda}=V_{\bar{K}^{0}\Lambda\rightarrow \bar{K}^{0}\Lambda}
\end{equation}
\begin{equation}
	\mathcal{L}=0
\end{equation}
\begin{equation}
	V_{\bar{K}\Lambda\to \bar{K}\Lambda}=0
\end{equation}
\subsubsection{$\bar{K}\Lambda\rightarrow \bar{K}\Sigma$}
\begin{equation}
	V_{\bar{K}\Lambda\rightarrow \bar{K}\Sigma}=\sqrt{\frac{1}{3}}V_{\bar{K}^{0}\Lambda\rightarrow \bar{K}^{0}\Sigma^{0}}-\sqrt{\frac{2}{3}}V_{\bar{K}^{0}\Lambda\rightarrow K^{-}\Sigma^{+}}
\end{equation}
\subsubsection{$\bar{K}^{0}\Lambda\rightarrow \bar{K}^{0}\Sigma^{0}$}
\begin{equation}
	\mathcal{L}=0
\end{equation}
\subsubsection{$\bar{K}^{0}\Lambda\rightarrow K^{-}\Sigma^{+}$}
\begin{equation}
	\mathcal{L}=0
\end{equation}
\begin{equation}
	V_{\bar{K}\Lambda\to \bar{K}\Sigma}=0
\end{equation}
\subsubsection{$\bar{K}\Lambda\rightarrow\eta\Xi$}
\begin{equation}
	V_{\bar{K}\Lambda\rightarrow\eta\Xi}=V_{\bar{K}^{0}\Lambda\rightarrow\eta\Xi^{0}}
\end{equation}
\begin{equation}
	\mathcal{L}=\frac{1}{4f^2}\frac{3}{2}\Xi^{0}i\gamma^{\mu}\Lambda(\bar{K}^{0}\partial_{\mu}\eta-\eta\partial_{\mu}\bar{K}^{0})
\end{equation}
\begin{equation}
	V_{\bar{K}^{0}\Lambda\rightarrow\eta\Xi^{0}}=\left(-\frac{3}{2}\right)\left(-\frac{1}{4f^2}\bar{u}(p')\gamma^{\mu}(k_{\mu}+k'_{\mu})\right)
\end{equation}
\subsubsection{$\bar{K}\Sigma\rightarrow \bar{K}\Sigma$}
\begin{equation}
	V_{\bar{K}\Sigma\rightarrow \bar{K}\Sigma}=\frac{1}{3}V_{\bar{K}^{0}\Sigma^{0}\rightarrow \bar{K}^{0}\Sigma^{0}}-\frac{\sqrt{2}}{3}V_{K^{-}\Sigma^{+}\rightarrow \bar{K}^{0}\Sigma^{0}}-\frac{\sqrt{2}}{3}V_{\bar{K}^{0}\Sigma^{0}\rightarrow K^{-}\Sigma^{+}}+\frac{2}{3}V_{K^{-}\Sigma^{+}\rightarrow K^{-}\Sigma^{+}}
\end{equation}
\subsubsection{$\bar{K}^{0}\Sigma^{0}\rightarrow \bar{K}^{0}\Sigma^{0}$}
\begin{equation}
	\mathcal{L}=0
\end{equation}
\subsubsection{$K^{-}\Sigma^{+}\rightarrow \bar{K}^{0}\Sigma^{0}$}
\begin{equation}
	\mathcal{L}=-\frac{1}{4f^2}\sqrt{2}\Sigma^{0}i\gamma^{\mu}\Sigma^{+}(K^{0}\partial_{\mu}K^{-}-K^{-}\partial_{\mu}K^{0})
\end{equation}
\begin{equation}
	V_{K^{-}\Sigma^{+}\rightarrow \bar{K}^{0}\Sigma^{0}}=\left(-\sqrt{2}\right)\left(-\frac{1}{4f^2}\bar{u}(p')\gamma^{\mu}u(p)(k_{\mu}+k'_{\mu})\right)
\end{equation}
\subsubsection{$\bar{K}^{0}\Sigma^{0}\rightarrow K^{-}\Sigma^{+}$}
\begin{equation}
	\mathcal{L}=-\frac{1}{4f^2}\sqrt{2}\bar{\Sigma}^{+}i\gamma^{\mu}\Sigma^{0}(K^{+}\partial_{\mu}\bar{K}^{0}-\bar{K}^{0}\partial_{\mu}K^{+})
\end{equation}
\begin{equation}
	V_{\bar{K}^{0}\Sigma^{0}\rightarrow K^{-}\Sigma^{+}}=\left(-\sqrt{2}\right)\left(-\frac{1}{4f^2}\bar{u}(p')\gamma^{\mu}u(p)(k_{\mu}+k'_{\mu})\right)
\end{equation}
\subsubsection{$K^{-}\Sigma^{+}\rightarrow K^{-}\Sigma^{+}$}
\begin{equation}
	\mathcal{L}=\frac{1}{4f^2}\bar{\Sigma}^{+}i\gamma_{\mu}\Sigma^{+}(K^{+}\partial_{\mu}K^{-}-K^{-}\partial_{\mu}K^{+})
\end{equation}
\begin{equation}
	V_{K^{-}\Sigma^{+}\rightarrow K^{-}\Sigma^{+}}=-\frac{1}{4f^2}\bar{u}(p')\gamma^{\mu}u(p)(k_{\mu}+k'_{\mu})
\end{equation}
\begin{equation}
\begin{split}
	V_{\bar{K}\Sigma\rightarrow \bar{K}\Sigma}=&-\frac{\sqrt{2}}{3}(-\sqrt{2}V_{0})-\frac{\sqrt{2}}{3}(-\sqrt{2}V_{0})+\frac{2}{3}V_{0}\\
	=&2V_{0}
\end{split}
\end{equation}
\subsubsection{$\bar{K}\Sigma\rightarrow\eta\Xi$}
\begin{equation}
	V_{\bar{K}\Sigma\rightarrow\eta\Xi}=\sqrt{\frac{1}{3}}V_{\bar{K}^{0}\Sigma^{0}\rightarrow\eta\Xi^{0}}-\sqrt{\frac{2}{3}}V_{K^{-}\Sigma^{+}\rightarrow\eta\Xi^{0}}
\end{equation}
\subsubsection{$\bar{K}^{0}\Sigma^{0}\rightarrow\eta\Xi^{0}$}
\begin{equation}
	\mathcal{L}=-\frac{1}{4f^2}\frac{\sqrt{3}}{2}\bar{\Xi}^{0}i\gamma^{\mu}\Sigma^{0}(\bar{K}^{0}\partial_{\mu}\eta-\eta\partial_{\mu}\bar{K}^{0})
\end{equation}
\begin{equation}
	V_{\bar{K}^{0}\Sigma^{0}\rightarrow\eta\Xi^{0}}=\left(\frac{\sqrt{3}}{2}\right)\left(-\frac{1}{4f^2}\bar{u}(p')\gamma^{\mu}u(p)(k_{\mu}+k'_{\mu})\right)
\end{equation}
\subsubsection{$K^{-}\Sigma^{+}\rightarrow\eta\Xi^{0}$}
\begin{equation}
	\mathcal{L}=\frac{1}{4f^2}\frac{3}{\sqrt{6}}\bar{\Xi}^{0}i\gamma^{\mu}\Sigma^{+}(K^{-}\partial_{\mu}\eta-\eta\partial_{\mu}K^{-})
\end{equation}
\begin{equation}
	V_{K^{-}\Sigma^{+}\rightarrow\eta\Xi^{0}}=\left(-\frac{3}{\sqrt{6}}\right)\left(-\frac{1}{4f^2}\bar{u}(p')\gamma^{\mu}u(p)(k_{\mu+k'_{\mu}})\right)
\end{equation}
\begin{equation}
	V_{\bar{K}\Sigma\rightarrow\eta\Xi}=\sqrt{\frac{1}{3}}\frac{\sqrt{3}}{2}V_{0}-\sqrt{\frac{2}{3}}(-\frac{3}{\sqrt{6}})V_{0}=\frac{3}{2}V_{0}
\end{equation}
\subsubsection{$\eta\Xi\rightarrow\eta\Xi$}
\begin{equation}
	V_{\eta\Xi\rightarrow\eta\Xi}=V_{\eta\Xi^{0}\rightarrow\eta\Xi^{0}}
\end{equation}
\begin{equation}
	\mathcal{L}=0
\end{equation}
\subsection{$V_{0}$的简化}
根据文献\cite{OSET199899},在低能的时候可以忽略空间部分,只保留$\gamma^{0}$项。因此可得
\begin{equation}
\begin{split}
	V_{0}=&-\frac{1}{4f^2}\bar{u}(p')\gamma^{\mu}u(p)(k_{\mu}+k_{\mu}')\\
	=&-\frac{1}{4f^2}\bar{u}_{0}\frac{\gamma^{0}p'_{0}+m_{1}}{\sqrt{2m_{1}(p'_{0}+m_{1})}}\gamma^{0}\frac{\gamma^{0}p_{0}+m_{1}}{\sqrt{2m_{1}(p_{0}+m_{1})}}u_{0}(k_{0}+k_{0}')\\
	=&-\frac{1}{4f^2}u^{\dagger}_{0}\sqrt{\frac{p'_{0}+m_{1}}{2m_{1}}}\sqrt{\frac{p_{0}+m_{1}}{2m_{1}}}u_{0}(k_{0}+k_{0}')\\
	=&-\frac{1}{4f^2}\sqrt{\frac{p'_{0}+m_{1}}{2m_{1}}}\sqrt{\frac{p_{0}+m_{1}}{2m_{1}}}(k_{0}+k_{0}')
\end{split}
\end{equation}
其中
\begin{equation*}
	u(k)=\frac{\slashed{k}+m}{\sqrt{2m(k^{0}+m)}}u_{0}
\end{equation*}
\begin{equation*}
\begin{aligned}
	\bar{u}(k)\equiv&u^{\dagger}(k)\gamma^{0}&\hspace{2cm}\gamma^{0}\slashed{k}^{\dagger}\gamma^{0}=&\gamma^{0}(\gamma^{\mu})^{\dagger}k_{\mu}\gamma^{0}\\
	=&u_{0}^{\dagger}\frac{\slashed{k}^{\dagger}+m}{\sqrt{2m(k^{0}+m)}}\gamma^{0}&=&\gamma^{0}\gamma^{0}k_{0}\gamma^{0}-\gamma^{0}\gamma^{i}k_{i}\gamma^{0}\\
	=&u_{0}^{\dagger}\gamma^{0}\frac{\gamma^{0}\slashed{k}^{\dagger}\gamma^{0}+m}{\sqrt{2m(k^{0}+m)}}&=&\gamma^{\mu}k_{\mu}\gamma^{0}\gamma^{0}\\
	=&\bar{u}_{0}\frac{\slashed{k}+m}{\sqrt{2m(k^{0}+m)}}&=&\slashed{k}
\end{aligned}
\end{equation*}
然后用能量守恒定律
\begin{equation}
	p_{0}+k_{0}=p'_{0}+k'_{0}=\sqrt{s}
\end{equation}
因此
\begin{equation}
	k_{0}+k'_{0}=2\sqrt{s}-p_{0}-p'_{0}
\end{equation}
在低能的条件下,重子的能量主要来自重子的质量。
最终$V_{0}$可写为
\begin{equation}
	V_{0}=-\frac{1}{4f^2}\sqrt{\frac{E_{i}+M_{i}}{2M_{i}}}\sqrt{\frac{E_{j}+M_{j}}{2M_{j}}}(2\sqrt{s}-M_{i}-M_{j})
\end{equation}

